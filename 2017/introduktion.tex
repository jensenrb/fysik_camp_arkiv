\chapter{Introduktion}
\label{cha:introduktion}

Velkommen til UNF Fysik Camp 2017. Dette kompendium er en introduktion
til de emner, som vi skal arbejde med i løbet af campen. Programmet,
som I skal igennem, indeholder flere forskellige emner, som alle
giver et indblik i, hvor vigtig og alsidig fysikken er. Emnerne er i år laserfysik, astrofysik, rotationel mekanik, relativitetsteori, elektromagnetiske bølger og kerne- og partikelfysik, hvor I blandt de sidste fire selv får lov til at vælge, hvilke to I ønsker at arbejde med. Alle er de relevante for vores verden, og vi håber, at I
vil finde dem lige så spændende, som vi selv gør.

I kommer alle med forskellig undervisningsbaggrund, og vi kræver
derfor ikke, at I bare forstår alt med det samme. Under campen vil der
være rig mulighed for at snakke mere om det, der står i dette
kompendie og virkelig dykke ned i stoffet. Kompendiet indeholder alt,
hvad I vil få brug for til at forstå og arbejde med emnerne under
campen, og I opfordres derfor kraftigt til at læse det. Særligt bør I
forsøge at læse de to første kapitler om laserfysik og astrofysik,
da disse emner vil indgå i det obligatoriske program, hvorimod de
andre emner er valgfrie.

Da I kommer fra forskellige klassetrin, har I ikke alle modtaget
undervisning i al den matematik, som I måske skal bruge. I opfordres
derfor også til at læse appendikset bagerst i kompendiet, som
forklarer den matematik, vi skal arbejde med på campen. Den relevante matematik for campens hovedemner er dækket i Afsnit A.2 og A.3, hvorfor disse er vigtigst at læse. 

God fornøjelse, og velkommen i fysikkens verden.

\begin{flushright}
På vegne af det faglige team, \\
\textit{Dorte Thrige Plauborg, fagligt ansvarlig } 
\end{flushright}

%%% Local Variables: 
%%% mode: latex
%%% TeX-master: "main"
%%% End: 
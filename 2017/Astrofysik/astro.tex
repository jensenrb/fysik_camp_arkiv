%Du begynder bare at skrive
\chapter{Astrofysik}
\section{Kosmologi}
Kosmologi er læren om universets skabelse, udvikling og endeligt. Når der snakkes om universets størrelse, er det normalt det synlige univers, der menes. Dette er den del af universet, hvor lys er nået hen til os, så vi kan observere det. Længere ude er informationen ikke nået frem til os endnu, så vi ved ikke, hvor stort hele universet er. 

Det stemmer overens med \emph{det kosmologiske princip}, der siger, at universets love og konstanter er ens overalt og massen er ligevægt fordelt (set fra en stor nok skala). Det kan deles op i to postulater: Universet er homogent (ens i hvert område) og isotropt (ser ens ud, uanset hvilken retning man kigger i), som illustreret i Figur \ref{isohomo}. Det kosmologiske princip behøver ikke gælde, men vi har ikke observeret noget, der bryder med det. Princippet antages derfor normalt at gælde, da det er det simpleste -- vi har ingen grund til at tro, at universets egenskaber pludselig ændrer sig et sted. Det er en god approksimation på skalaer fra 200 Mpc (megaparsec) og opefter. 1 pc er cirka 3 lysår, så det er $3\cdot 10^8$ lysår. På små skalaer holder det selvfølgelig ikke. For eksempel er du tættere (har større massefylde) end luften omkring dig, og Solen kan udpege en særlig retning inden for solsystemet.

Antagelsen om, at universet er homogent og isotropt, er understøttet af
kortlægningen af den \emph{kosmiske mikrobølgebaggrund} (eng: CMB – cosmic microwave
background). Baggrunden består af stråling fra det tidspunkt, hvor universet blev
gennemsigtigt (ca. 380 000 år efter Big Bang) – altså hvor temperaturen af plasmaet
og strålingen dannet efter Big Bang er aftaget tilstrækkeligt, så det er muligt for frie elektroner at 
kombinere med atomkerner og danne hydrogen og helium
og lyset kunne undslippe. Af historiske årsager kaldes dette for \emph{rekombinationen}, men man kalder det også for \emph{foton--afkoblingen}. Den kosmiske mikrobølgebaggrund er derfor det ældste
lys i universet! Dengang var strålingen i UV-området, men udvidelsen af Universet
(Afsnit \ref{12}) har nu kølet den til en temperatur på 2.73 K og flyttet den til mikrobølgeområdet. Den kosmiske mikrobølgebaggrund er blevet kortlagt af flere
missioner, først COBE (Cosmic Background Explorer), siden WMAP (Wilkinson
Microwave Anisotropy Probe) og senest Planck, se Figur \ref{CMB}. Det ses, at selv med høj opløsning er kortet uniformt – beviset på et næsten
homogent og isotropt tidligt univers. Men små fluktuationer i tætheden er stadig til
stede, og var det ikke for disse, ville den gravitationelle tiltrækning ikke have kunnet
skabt de galakser, vi observerer (og bebor!) i dag.

\begin{figure}[h]
	\centering
	\includegraphics[width=0.7\textwidth]{Astrofysik/Astrofig/CMB.jpg}
	\caption{Kortlægningen af den kosmiske mikrobølgebaggrund af COBE, WMAP og
		Planck. Mælkevejen, som ellers ville være til stede i billedet, er redigeret ud. }
	\label{CMB}
\end{figure}
%Fra http:
%		//www.planetastronomy.com/astronews/astrn-2013/04/astron3.jpg

\begin{figure}[h]
	\centering
	\includegraphics[width=0.7\textwidth]{Astrofysik/Astrofig/isohomo.png}
	\caption{Venstre: Illustration af isotropi. Fra midten ser verden ens ud i alle retninger. Højre: Illustration af homogenitet. Kigger man på et stort nok udsnit af verden, vil det se ud som alle andre udsnit.}
	\label{isohomo}
\end{figure}

%\section{Big Bang og CMB}
\section{Rødforskydning}
\label{12}
\subsection{Dopplerforskydning}
Du kender nok til, at når en ambulance kører forbi, så lyder sirenens tone højere når den nærmer sig, og dybere når den kører væk. Det skyldes \emph{Dopplerforskydning}. %Find gerne lydklip til undervisningen
Det skyldes, at lydbølgerne skubbes sammen og strækkes, afhængigt af hvilken fart de udsendes med i forhold til lytteren. Hvis en ambulance kører mod dig, og du står stille, vil du høre bølgerne sammenpresset. Men hvis du selv kører med samme hastighed foran ambulancen, så vil du høre dem på samme måde, som de bliver udsendt - altså på samme måde som hvis begge biler står stille, da det er den relative hastighed, som er afgørende.

\begin{figure}[h!]
	\centering
	\includegraphics[width=0.7\textwidth]{Astrofysik/Astrofig/doppler.jpg}
	\caption{Dopplerforskydning af lyden fra en bil, der kører mod venstre. Person A vil høre end højere tone end person B, og personen i bilen vil høre noget et sted derimellem. Ringene viser et fast punkt på lydbølgerne.}
	\label{shapes}
\end{figure}

Det samme sker for lys. Hvis en ambulance kører væk, vil både tonen blive dybere og lyset fra den en smule rødere. Bemærk dog, at selvom fotoner og lydbølger har højere energi ved korte bølgelængder, så mister de ikke energi ved Dopplerforskydning - der er bare sket et skift i perspektivet, man ser bølgerne fra. Fra bølgens eget synspunkt (hvis man følger den) har den samme bølgelængde og energi hele tiden.
\subsection{Kosmologisk Rødforskydning}
Edwin Hubble opdagede i 1920'erne, at galakser langt borte ser rødforskudte ud, men dette skyldes \emph{ikke} Dopplerforskydning fra galaksernes egenhastighed i forhold til os. Så ville vi have forventet, at lige så mange galakser var rødforskudte som blåforskudte, hvis de starter med en tilfældig hastighed. Og det gør de jo - for hvorfor skulle de have en særlig retning i forhold til os? Det ville bryde med isotropien. 

Hubble opdagede, at galakserne oftere er rødforskudt end blåforskudt, og jo længere væk de er, desto mere rødforskudte er de også. Han formulerede \emph{Hubbles lov}, der beskriver den direkte proportionalitet mellem afstand og fart af en galakse:
\begin{align}
v=H_0 D, \label{Hubbleslaw}
\end{align}
hvor $v$ er farten, $D$ er afstanden til galaksen, og $H_0=67.7\pm0.5$ er Hubbles konstant, som er den nuværende værdi af Hubble-parameteren (der ikke er konstant).

Galaksers relative hastighed til os er altså større, desto længere væk de er. Og sådan vil det se ud fra ethvert punkt i universet. Derfor må rødforskydningen stamme fra, at alting bevæger sig længere væk fra hinanden, som rosiner i en hævende bolle. Det er netop, hvad der sker - universets dej dvs. selve rumtiden "hæver". Galakserne har lige ofte egenhastigheder der går mod os som fra os, men hastigheden, som selve rumtiden udvider sig med, er vigtigere for galakser langt væk. For fjerne galakser skal lyset bevæge sig gennem mere rum for at nå frem til os. Derfor når lyset at blive strukket mere end ved nære galakser. 

Ved rødforskydning fra universets udvidelse mister lyset rent faktisk energi i modsætning til almindelig Dopplerforskydning. Dette er selvfølgelig et brud på energibevarelse, men det er egentlig ikke et problem, da man ikke kan betragte universet som et lukket system, fordi det udvider sig. Energibevarelse behøver kun at gælde i lukkede systemer. Nogle mener dog, at ændringen i energi fra universets udvidelse (og den mørke energi der opstår, se afsnit \ref{bestanddele}) udlignes af, at den gravitationelle energi falder til endnu lavere negative niveauer, således at universets totale energi altid er 0.

Måden, man måler rødforskydningen på, er ved at opsplitte lyset i dets forskellige bølgelængder. Dvs. man tager spektrer af fjerne objekter, og derefter genkender man mønstre fra jordiske laboratorier. Niels Bohr opdagede, at elektroner kun kan eksistere i bestemte baner om en atomkerne, men ikke mellem disse. Hver bane har en bestemt energi, så elektronernes energi i atomer er kvantiseret, dvs. de findes kun i bestemte pakker. Når en elektron henfalder til en lavere tilstand, kommer den af med overskydende energi ved at udsende en foton. Og hvis en foton med passende energi rammer en elektron, kan fotonen blive absorberet, så elektronen kommer op i en højere energitilstand. 

For lysspektrer gælder 3 love kaldet Kirchoffs love (ikke at forveksle med Kirchoffs love for elektriske kredsløb), som er illustreret i Figur \ref{kirchoff}.
\begin{itemize}
	\item Varme, uigennemsigte objekter udsender lys kontinuert over hele spektret. Ideelt set ville det give spektret for sortlegemestråling, og det er en særligt god approksimation for varme stjerner. Mindre stjernes lys er mere "forurenet" af effekter fra molekylær hydrogen (det er relativt koldt) og andre stoffer.
	\item Varme, gennemsigte gasser udsender lys og danner emissionsspektrer.
	\item Kolde gasser danner absorptionslinjer. Hvis en stjerne ligger bagved og sender lys mod os, vil skyen absorbere lyset og udsende det senere i en tilfældig retning. Det er meget usandsynligt, at retningen er mod os igen, så vi ser mindre lys ved denne bølgelængde.
\end{itemize}

\begin{figure}[h]
	\centering
	\includegraphics[width=0.7\textwidth]{Astrofysik/Astrofig/kirchoffslaws.jpg}
	\caption{Kontinuert spektrum (til højre), absorptions-spektrum (midtfor) og
		emissions-spektrum (til venstre). }
	\label{kirchoff}
\end{figure}
%fra http://astro.psu.edu/public-outreach/fireworks-masks-1/absorption-and-emission-spectra

\begin{figure}[h!]
	\centering
	\includegraphics[width=0.7\textwidth]{Astrofysik/Astrofig/spektrum.png}
	\caption{Et typisk stjernespektrum. Y-aksen skal forstås som intensitet, mens der
		på X-aksen er bølgelængde i ångstrøm (Å) (1 Å = $10^{-10}$ m). Dykkene i intensitet
		er angivet med en overgang tilhørende et grundstof, som er identificeret i stjernens
		atmosfære. }
	\label{shapes}
\end{figure}
%fra http://i.stack.imgur.com/RMHmB.gif

Har man en galakse med en stjerne, som udsender et bredt spektrum af lys, vil lyset både bevæge sig gennem stjernens ydre "kolde" lag og galaksens gasskyer, før det når os. Stjerner og skyer består af forskellige stoffer såsom hydrogen. Når de belyses, absorberer hydrogenet fotoner med de energier, der svarer til energiforskellen mellem banerne i hydrogen. Der dannes derfor et helt bestemt mønster af absorbtionslinjer i spektret, som er unikt for f.eks. hydrogen. For en rødforskudt galakse vil mønsteret ligge ved længere bølgelængder, end det vi måler for hydrogen på Jorden, men det er stadig genkendeligt.

Når vi kan genkende et mønster af absorptionsliner eller emissionslinjer, selvom det ligger forskudt ved andre bølgelængder end normalt, så kan vi finde rødforskydningen. Den er defineret som forskellen mellem observeret bølgelængde $\lambda_{obs}$ og laboratoriebølgelængde $\lambda_{lab}$ i forhold til laboratoriebølgelængden. Det er altså den relative forskydning i forhold til den oprindelige bølge. Lad os opskrive det som

\begin{align}
z=\frac{\lambda_{obs}-\lambda_{lab}}{\lambda_{lab}}.
\end{align}
Rødforskydningen $z$ er relateret til farten $v$ således:
\begin{align}
z+1=\sqrt{\frac{1+\frac{v}{c}}{1-\frac{v}{c}}}.
\end{align}
For hastigheder meget langsommere end lysets hastighed i vakuum $c$, kan det forsimples til:
\begin{align}
z\approx\frac{v}{c}.
\end{align}

Rødforskydning er altså en form for afstandsmål, men også et tidsmål, da lyset har rejst i lang tid, hvis det har nået at passere en stor afstand og er blevet meget udtrukket. Hvordan rødforskydningen relaterer til andre måder at måle afstand afhænger af, hvordan rumtiden strækker lyset. Og det afgøres af Universets form via generel relativitetsteori.

\section{Universets Form}

Universet har samlet set en form. Vi har 3 tydelige rumdimensioner omkring os, og vi er vant til, at hvis man tegner to parallelle linjer, så vil de aldrig krydse, og en trekant har altid 180 grader. Men dette gælder kun i fladt rum! Forestil dig f.eks. en trekant tegnet på en globus; den vil faktisk have mere end 180 grader. På samme måde vil en trekant tegnet på en saddel have mindre end 180 grader, som på Figur \ref{shapes}. Hvis universet er kugleformet, har det en positiv krumning, og hvis det er saddelformet, har det en negativ krumning. I et fladt univers er krumningen 0.

\begin{figure}[h!]
	\centering
	\includegraphics[width=0.7\textwidth]{Astrofysik/Astrofig/universe_geometry.png}
	\caption{Trekanter i forskellige geometrier har forskellige vinkelsummer. }
	\label{shapes}
\end{figure}

Hvis et objekt er i frit fald, eller hvis man har en lysstråle, vil de bevæge sig langs en ret linje i rumtiden (kaldet en geodæt). Men hvis selve rumtiden krummer, så gør den "rette"  linje det også. Derfor kan man se universets krumning ved at kigge på lyset fra objekter langt væk. Den viser sig ved, om ting ser forstørrede eller formindskede ud (om strålerne spredes eller samles som i en linse), og disse målinger viser, at universet er fladt med en præcision på 0.5 \%. Det vender vi tilbage til i afsnit \ref{bestanddele}. %Kilde + illustration

Hvis universet var positivt krumt og småt nok, ville det også betyde, at lyset kunne nå hele vejen rundt, og vi ville se de samme objekter flere steder på himlen, hvilket heller ikke er observeret. Så det synlige univers er i hvert fald meget fladt -- men måske har rumtiden en svag krumning, der bare ikke kan ses her. Selvom Danmark ser fladt ud, kan hele jordkloden jo godt være rund.

Hvis universet er positivt krumt, er det endeligt, mens fladt eller negativt rum kan være uendeligt stort. Det er dog også muligt for universet både at være f.eks. fladt og endeligt, men så bryder man det kosmologiske princip.

Vi kan sammenligne universets størrelse ved forskellige tidspunkter gennem en skalafaktor $a(t)$. Vi definerer den nuværende faktor til 1, dvs. $a(t_0)=1$, hvor $t_0$ er tiden nu. Der gælder:
\begin{align}
\frac{a(t_0)}{a(t)}=1+z\\
a(t)=\frac{1}{1+z}.
\end{align}
Så man kan let omregne rødforskydning til skalafaktoren, fra dengang lyset blev udsendt.

Som beskrevet % i \ref{Hubble}
i forbindelse med ligning \ref{Hubbleslaw} er Hubblekonstanten ikke konstant, men blot den nuværende værdi af Hubbleparameteren. Hubbleparameteren er
\begin{align}
H=\frac{v}{D}=\frac{\dot{a}}{a},
\end{align}
hvor en prik betyder differentieret mht. tid, så det er den relative ændring i skalafaktoren. Hvis vi sætter det i anden og bruger noget generel relativitetsteori, får vi \emph{Friedmann-ligningen}:
\begin{align}
H^2=\left(\frac{\dot{a}}{a}\right)^2=\frac{8\pi G \rho}{3}-\frac{\kappa c^2}{a^2}+\frac{\Lambda}{3}. \label{friedmann}
\end{align}
Denne ligning er superinteressant, da den viser os, hvordan universet udvikler sig. Det er en andengradsligning med $a$ indeholdende konstanter som gravitationskonstanten $G$ og lysets fart i vakuum $c$. $\kappa$ kan være 1, 0 og -1 og dette afgør krumningen, der som bekendt kan være positiv, 0 eller negativ. $\Lambda$ er den kosmologiske kontant. Uden denne ville rumtiden trække sig sammen, fordi massen krummer det, så Einstein introducerede $\Lambda$ for at holde universet statisk. Det har han senere kaldt sin største fejl, efter Hubble opdagede, at universet er dynamisk. Man har dog genintroduceret konstanten for at accelerere udvidelsen, da man opdagede mørk energi i 1998. Lad os se på, hvad det egentlig er.

\section{Universets Komponenter} \label{bestanddele}
Universets udvikling og skæbne afhænger af dets indhold. Det består af:
\begin{itemize}
	\item Stråling/relativistisk stof: Fotoner og neutrinoer (fordi de har ingen eller meget lav masse samt høj hastighed)
	\item Stof: Almindeligt stof, antistof og mørkt stof har alle masse, så her kalder jeg dem samlet set "stof". Egenskaben masse afgør, hvor mange kræfter man skal bruge på at accelerere partiklerne, men også hvor meget de krummer rumtiden omkring sig. 
	\item Kosmologisk konstant: Den form for mørk energi, vi antager universet er fyldt med. Det får universets udvidelse til at accelerere, er ligeligt fordelt overalt og fortyndes ikke fra udvidelsen.
\end{itemize}

Disse komponenter påvirker universets form og udvikling forskelligt. Mængden af stof er nogenlunde konstant, men universet udvider sig i alle 3 rumdimensioner, så massedensiteten falder med:
\begin{align}
\rho_m \propto a^{-3}.
\end{align}
Stråling har ingen til næsten ingen hvilemasse, men ved høj fart får de det, hvad man kalder en relativistisk masse. Fotoner har jo energi og impuls, og det kan konverteres til masse. Det er derfor lys ikke kan undslippe sorte hullers masse, selvom lyset ikke har en hvilemasse. Den effektive masse bøjer rummet omkring sig, så stråling får universet til at trække sig sammen, ligesom stof. Stråling har dog en ekstra egenskab, nemlig at det rødforskydes. Derfor fortyndes energien af stråling både med universets udvidelse og en ekstra faktor fra rødforskydningen:
\begin{align}
\rho_R\propto a^{-4}.
\end{align}

Mørk energi ved man ikke særlig meget om, men man formoder ofte, det stammer fra energien i vakuum. Der er dog et ekstremt stort problem ved dette -- vakuumenergien burde være $10^{120}$ gange større! Dette kan ses som et af mange usandsynlige tilfælde, der gør, at universet akkurat passer til, at liv kan opstå. Denne problemstilling er kendt som \emph{the finetuning problem}, og der er mange mulige løsninger på det. De er ofte ganske farverige f.eks. multiverser, virtuelle universer og brud på det kosmologiske princip gennem variende konstanter på tværs af sted. %\cite{TheAccUniverse} 

I den simple antagelse, at mørk energi består af en "kosmologisk konstant", vil energien ikke fortyndes, så der hele tiden opstår mere mørk energi med udvidelsen og densiteten er konstant.
\begin{align}
\rho_\Lambda = \text{konstant}
\end{align}

Hvis universet er fladt, må det have en helt bestemt samlet densitet kaldet den kritiske densitet $\rho_c = 8.6\cdot 10^{-27} kg/m^3$. Lad os definere en densitetsparameter $\Omega$, som densitet i forhold til den kritiske densitet:
\begin{align}
\Omega=\frac{\rho}{\rho_c}.
\end{align}
Hvis vi indsætter densiteten af hver parameter får vi
\begin{align}
\Omega_{m,0}=0.3089\pm 0.0062\qquad
\Omega_{R,0}&=8.24\cdot 10^{-5}\qquad
\Omega_{\Lambda,0}=0.6911\pm 0.0062\\
\Omega_{\text{total},0}=\Omega_{m,0} + &\Omega_{R,0} + \Omega_{\Lambda,0}=1.000\pm0.005 \label{Omegatot}
\end{align}

0 er i subscript for at vise, at det er nuværende værdier. Den samlede densitet er altså lig eller meget tæt på den kritiske densitet, så det synlige univers lader til at være fladt. Som nævnt indeholder "stof" både synligt, baryonisk stof (og antistof) og mystisk mørkt stof, så vi kan videre opdele således:
\begin{align}
\Omega_{B,0}=0.049\pm 0.001\qquad
\Omega_{DM,0}=0.259\pm 0.006
\end{align}
Det stof, vi omgiver os med til hverdag udgør altså blot 5 \% af universets indhold, mens 26 \% er mørkt stof, som vi ikke ved meget om, og 69 \% er mørk energi, som vi ved næsten intet om. Og det hele går lige op, så den samlede densitet giver et fladt univers inden for bare en enkelt standardafvigelse. Mørkt stof er beskrevet nærmere i afsnit \ref{DM}.

\subsection{Komponenternes Udvikling}
Hver komponent i universet har en faktor $\omega$, som afgør hvor stort tryk $p$ de yder, hvilket er beskrevet ved \emph{tilstandsligningen}
\begin{align}
p=\omega \rho.
\end{align}

Vi kan desuden beskrive densiteten af hver komponent som
\begin{align}
\rho = \rho_0 a^{-3(1+\omega)}. \label{density}
\end{align}

For stråling er $\omega=\frac{1}{3}$, for stof er $\omega=0$ og for kosmologisk konstant er $\omega=-1$.


Skalafaktoren $a(t)$, der beskriver universets størrelse i enheder af dets nuværende størrelse, hænger også sammen med $\omega$. Bestod universet kun af én komponent, så ville
\begin{align}
a(t)=\left(\frac{t}{t_0}\right)^{2/(3+3\omega)},
\end{align}
hvor $t_0$ er universets alder nu. Hvis universets størrelse er monotont stigende, kan vi se $a$ som et tidsmål. Formlen kan selvfølgelig ikke gælde for kosmologisk konstant, da vi så ville dividere med 0. I stedet kan man løse Friedmann-ligningen (ligning \ref{friedmann}) og få
\begin{align}
	a(t)=e^{H_0(t-t_0)}.
\end{align}
Det ville være en dårlig model til data at antage, universet kun bestod af én ting. Så lad os hellere bruge Benchmark-modellen, hvor alle tre komponenter, beskrevet her, indgår.

Hvis vi kigger på ligning \ref{density}, så er det klart, at stråling og stof må have spillet en større rolle i universets barndom, end det gør i dag. Densiteten var højere og det udgjorde en større andel af den samlede densitet, når vi husker på, at mængden af kosmologisk konstant er konstant. Selvom energien fra stråling er meget lav i dag, så falder den også hurtigst, og det betyder, at den engang har været dominerende i universet. Der skal vi selvfølgelig meget langt tilbage. Helt til universet kun var $47$ tusind år gammelt. Kosmologisk konstant har domineret siden universet var $9.8$ mia. år gammelt, og i dag er det $13.8$ mia. år. Det er altså relativt kort tid siden mørk energi begyndte at dominere. De forskellige perioder og densiteter er indtegnet på Figur \ref{figDensity}.

\begin{figure}[h!]
	\centering
	\includegraphics[width=0.7\textwidth]{Astrofysik/Astrofig/density.jpg}
	\caption{Densitet af hver komponent af universet som funktion af tid. Bemærk akserne er logaritmiske. }
	\label{figDensity}
\end{figure}
%Billede fra http://pages.uoregon.edu/jimbrau/BrauImNew/Chap27/7th/AT\_7e\_Figure\_27\_01.jpg

Vi kan lave en approksimation og antage, at i hver fase vil skalafaktoren $a$ udvikle sig, som om universet kun består af den komponent, der dominerer.

Udover disse faser, skete der en voldsom inflation lige i starten omkring $10^{-35}$ sekund efter Big Bang. I dette tidsrum blev universet $10^{26}$ gange større. Det gik fra at være på størrelse med en proton til ca. en grapefrugt. Drivkraften har været en anden kosmologisk konstant, der dominerede netop der, men senere blev overgået af stråling og stof. Massen fik så universets udvidelse til at deaccelerere igen, indtil den kosmologiske konstant for nylig overtog. Baggrunden for denne teori er, at den forklarer hvorfor universet er så fladt (ujævnheder blev udjævnet) og hvorfor temperaturen er så ens overalt i det synlige univers. Man skulle tro hver ende af det synlige univers aldrig ville have været i kontakt, så de ville ikke kunne udveksle energi. Alligevel er temperaturfordelingen i baggrundsstrålingen som et sortlegeme. Men hvis alt lå virkelig tæt før inflationen, så kunne informationer og varme godt udveksles førhen. Det er altså en forklaring på, hvorfor det kosmologiske princip gælder i det synlige univers.

%\section{Universets skæbne}

\section{Mørkt Stof} \label{DM}
%Noget med rotationskurver af galakser, v af galakser i hobe, gravitational lensing

Ikke alene er det meste af det massen fra stof udetekterbart for vores øjne, men det er ikke nødvendigvis baryonisk heller (bundet i neutroner, protoner og lignende). Størstedelen af stoffet i universet er ikke--baryonisk mørkt stof, hvilket betyder, at det hverken absorberer, emitterer eller spreder lys ved en hvilken som helst bølgelængde. En måde hvorpå mørkt stof kan detekteres, er ved at se på dets gravitationelle indflydelse på synligt stof. En klassisk metode er at se på den orbitale hastighed af stjerner i spiralgalakser. Mælkevejen er f.eks. en spiralgalakse. Tag nu Solen. Den er i en afstand $R=8.5~\text{kpc}$ fra centrum af galaksen, og har en orbital hastighed på omkring $v=220~\text{km}~\text{s}^{-1}$. Solen vil opleve en acceleration
\begin{equation}
a = \frac{v^2}{R} 
\end{equation}
mod centrum af galaksen. Hvis accelerationen er givet ved gravitationel tiltrækning, så er
\begin{equation}
a = \frac{G M(R)}{R^2},
\end{equation}
hvor $M(R)$ er massen af galaksen indenfor en bestemt radius, $R$. De to ovenstående ligninger kan vi sætte lig hinanden for da at få et udtryk for hastigheden:
\begin{equation}
\frac{v^2}{R} = \frac{G M(R)}{R^2}
\end{equation}
eller
\begin{equation}
v = \sqrt{\frac{G M(R)}{R}}.
\end{equation}
Vi kan måle hastigheden af stjerner i en galakse ved hjælp af deres rødforskydning. Nu ved vi, at hastigheden er større, desto større en masse stjernen kredser om. Altså kan vi beregne fordelingen af masse ved at kigge på stjerner forskellige steder i galaksen.

Overfladelysstyrken, $I$, af disken i en spiralgalakse aftager med radius. Lysstyrken fortæller, os hvordan stjernerne dvs. synligt stof er fordelt.
\begin{equation}
I(R) = I(0) e^{-\frac{R}{R_s}},
\end{equation}
hvor $R_s$ er skalalængden, som typisk ligger indenfor et par kiloparsec. Vores galakse har $R_s\approx 4~\text{kpc}$. Så snart du er et par skalalængder fra centrum af en spiralgalakse begynder massen af stjernerne indenfor $R$ af være konstant -- længere ude er der nemlig næsten intet lys. Hvis stjernerne bidrog til alt eller det meste af massen i en galakse, ville hastigheden falde af som $v \propto 1/\sqrt{R}$ ved store radier. Men det ser vi ikke. Hastighederne holder sig nogenlunde konstante, som på Figur \ref{rotationskurve}, så der mangler noget masse. Faktisk mangler der mere masse, jo længere vi bevæger os ud (til en hvis grænse). Denne manglende masse kalder vi mørkt stof, da den ikke er synlig. At den ligger så langt ude er et tegn på, at massen ikke interagerer meget med hverken sig selv eller synligt stof, så det har ikke mistet energi ved kollisioner, udover at tyngdekraften hiver lidt i den. Derfor ligger det stadig langt væk med høje hastigheder, men er dog samlet omkring galaksers tyngdefelter. Denne komponent af galakser kalder vi deres \emph{dark matter halos}. De er sfæriske og ligger altså ikke kun i spiralgalaksens disk.

Man plejede at have to teorier for, hvad mørkt stof består af - WIMPs og MACHOs. WIMP står for Weakly Interacting Massive Particles og ville være en ny, tung type elementarpartikler. MACHO står for Massive Compact Halo Objects og er mere almindelige ting såsom sorte huller, svage dværgstjerner og "forældreløse" planeter, der er blevet slynget væk fra deres stjerner. Ting, der ikke lyser nok til, vi ville kunne se dem på lang afstand. Man har nu udelukket, at MACHOs kan udgøre en signifikant del af det mørke stof, da vi ville kunne se dets klumper af tyngdekraft deformere lyset af objekter bagved. Dette fænomen, kaldet \emph{gravitationslinseeffekten}, ses dog i mange andre sammenhænge f.eks. fra det mørke stof af en hel galakse. 

\begin{figure}[h!]
	\centering
	\includegraphics[width=0.7\textwidth]{Astrofysik/Astrofig/rotationskurve.jpg}
	\caption{Rotationskurve over Mælkevejen. Den solide kurve viser de observerede hastigheder, og den stiplede viser de forventede fra fordelingen af synligt stof. Afstanden mellem kurverne viser fordelingen af mørkt stof. 
		}
	\label{rotationskurve}
\end{figure}
%Billede fra http://pages.uoregon.edu/jimbrau/BrauImNew/Chap23/6th/23\_21Figure-F.jpg

\subsection{Gravitationelle Linser}

En af konsekvenserne af generel relativitetsteori er, at et massivt objekt kan fungere som en linse, der bøjer lyset fra en fjernere kilde. F.eks. kan lyset fra en kvasar blive bøjet af tyngdefeltet fra en galaksehob, der ligger mellem kilden og observatøren. Jo mere massiv en genstand er, jo større linseeffekt ser vi. Derfor kan vi fra effekten regne os frem til massen af det mellemliggende objekt. Hvis det er en galakse, kan vi se, at den bøjer rummet meget mere end hvad den burde ud fra det synlige lys. Igen mangler der altså masse, og mørkt stof må eksistere.


Stærk lensing er når forvrængningen af f.eks. baggrundsgalakser, danner flere billeder af den samme galakse eller en hel ring rundt om det tunge objekt. Dette har vi observeret omkring mange fjerne hobe, hvilket inkluderer den berømte Abell 1689, se Figur \ref{abell1689}. Ved måling af forvrængningsgeometrien (Eng: distortion geometry) kan massen af den mellemliggende hob findes. %I mange tilfælde, hvor man har gjort dette, opnåede man masse-til-lys forhold svarende til det dynamiske mørke stof målt i hoben.

Svag gravitationel lensing undersøger små forvrængninger ved hjælp af statistiske analyser fra store galakseundersøgelser. Ved at undersøge den tilsyneladende forskydningsdeformation af de tilstødende baggrundsgalakser, kan den gennemsnitlige fordeling af mørk stof karakteriseres.

Gravitationslinseeffekten kan også bruges til at opdage exoplaneter (planeter uden for Solsystemet). Det gælder, hvis lyset fra et objekt passerer en stjerne med en exoplanet, før det når frem til os. Så vil linsen forstærke lyset, og vi ser et ekstra lille bump fra planetens tyngdekraft.

\begin{figure}[h!]
	\centering
	\includegraphics[width=0.6\textwidth]{Astrofysik/Astrofig/abell1689.jpg}
	\caption{Stærk gravitationel lensing observeret med Hubble Space Telescope, hvilket er en indikator for mørkt stof. Billede fra Hubble Space Telescope.}
	\label{abell1689}
\end{figure}

\section{Afstande og Luminositet}
Afstand er et vigtigt begreb i astrofysikken og kosmologien, men ofte også svært at måle. Derfor har vi mange forskellige typer afstandsmål, og hvordan de hænger sammen, er ikke altid fastlagt. En af de mest brugte metoder til bestemmelse af afstande er gennem et objekts \textit{luminositet}. Et objekts absolutte luminositet $L$ er defineret som den energi, der er udsendt per sekund. Det har vist sig, at en stjernes luminositet, hvis vi antager, at det er et sortlegeme, er relateret til dets radius $R$ og temperatur $T$:
\begin{equation}
L = 4\pi\sigma R^2T^4 \propto R^2 T^4
\end{equation}
hvor $\sigma$ er Stefan-Boltzmann's konstant. \\
Hvis denne energi er udsendt uniformt i alle retninger og modtages i en afstand $d_L$ væk, er den modtagede \textit{tilsyneladende luminositet} - eller \emph{flux} - givet ved:
\begin{equation}
f = \frac{L}{4\pi d_L^2},
\end{equation}
hvor $d_L$ er luminositetsafstanden. 

Et andet eksempel på et afstandsmål, er vinkelafstanden. Her kigger man simpelthen på hvor stor en vinkel, $\Delta \theta$, objektet udspænder på himlen. Metoden kræver adgang til en stanardlineal dvs. et objekt hvis egenlængde, $l$, er kendt. Egenlængden er den længde, man ville måle med en lineal fra den ene til den anden side. Standardlinealen skal være bundet godt sammen af enten tyngdekraften eller gaffatape, eller måske en helt tredje kraft, som ikke tillader objektet at ekspandere med universet.  Vinkelafstanden, $d_A$, opskrives som
\begin{align}
	d_A=\frac{l}{\Delta \theta}
\end{align}

Udtrykkene for disse afstande er dog mest praktiske, når de er skrevet som funktion af $z$, da rødforskydningen altid er observerbar. De kan let blive skrevet som en funktion af skalafaktoren $a$ samt andre parametre ved at lave små og forholdsvis simple transformationer. For eksempel er de to udtryk for $d_L$ og $d_A$ næsten ens, men med en lille forskel:
\begin{align}
d_L(z) = (1+z)d_M(z) \\
d_A(z) =\frac{d_M(z)}{1+z},
\end{align}
hvor $d_M(z)$ afhænger af krumningen af universet.


Indenfor astronomiens verden beskrives et objekts lysstyrke ved \emph{magnitudesystemet}. Dette er et logaritmisk system, der rækker helt tilbage til Hipparchos i antikkens Grækenland. Af historiske årsager fungerer systemet således, at jo højere magnitude, desto svagere ser objektet ud på himlen og omvendt. Dengang fandtes der ikke teleskoper, som i dag, så alle målinger af stjerner blev taget per øjemål. Dengang gik skalaen fra 0 til og med 6, hvor 0 var det stærkeste på himlen og så fremdeles. Vi bruger i dag en skala, der er defineret således, at grækernes målinger stadig passer ind.

\subsection{Standard Candles}

Et standard candle er et astronomisk objekt, der har en kendt absolut magnitude. Disse er super vigtige for astronomer, da man ved hjælp af den tilsyneladende magnitude af et objekt kan bestemme afstanden ved at bruge
\begin{equation}
m-M=5\cdot \log_{10}(\frac{d_L}{pc}) - 5 ,
\end{equation}
hvor $m$ er den tilsyneladende magnitude (den lysstyrke, vi ville se fra Jordens overflade), $M$ er den absolutte magnitude (den faktiske lysstyrke af et objekt), og $d$ er afstanden til objektet målt i parsec. Forskellen kaldes objektets distancemodulus. Når der divideres med parsec, så er det blot for at fjerne længdeenheden, da man kun kan tage logaritmen til et tal uden enheder.

De mest anvendte standard candles indenfor astronomi er Cepheide variable stjerner og RR Lyræ stjerner. I begge tilfælde kan stjernernes absolutte magnituder bestemmes ud fra deres variabilitetsperioder.

Den nye dreng i klassen er Type Ia supernovaer. Disse kan også klassificeres som standard candles, men er i virkeligheden standardiserbare candles, da de ikke har præcist den samme maksimale lysstyrke. Forskellene i deres maksimale lysstyrke er imidlertid korreleret med hvor hurtigt, lyskurven aftager efter maksimal lysstyrke. Her ser man på differencen mellem den maksimale lysstyrke og lysstyrken 15 dage efter. Dette kaldes også $\Delta~m_{15}$. Hvis denne har en værdi mindre end $1$ er objektet lysstærkt, mens den ved en værdi over $1$ er lyssvag.

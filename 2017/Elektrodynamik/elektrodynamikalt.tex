\chapter{Elektromagnetiske bølger}

\section{Elektriske- og magnetiske felter}
I fysikken er felter et godt værktøj, og inden for elektrodynamik er de uundværlige. Vi vil især beskæftige os med vektor felter, der er en  funktion, der giver hvert punkt i rummet en tilhørende vektor. Et simpelt eksempel på et vektorfelt er tyngdeaccelerationen. 
$$
\v{g}(x,y,z) = -g\hat{z}
$$
Det elektriske felt kommer af Coulombs lov: $\v{F} = \frac{q_1q_2}{4\pi\varepsilon_0r^2}$, der beskriver kraften imellem to ladninger. Kraften på ladning 1 afhænger af begge ladninger. I fysik vil man gerne finde så gennerelle løsninger som muligt. Istedet for at finde kraften imellem de to ladninger finder man det elektriske felt $\v{E}$ fra hver ladning. Felterne er defineret så krafen på en ladning i feltet er $\v{F}=\v{E}q$. 

Den magnetiske kraft på en ladning er lidt anderledes end den elektriske kraft, den afhænger ikke kun af ladningen, men også hvor hurtigt og i hvilken retning ladningen bevæger sig. Man kan også beskrive kraften ved et felt: $\v{F} = q(\v{v}\times\v{B})$.




\section{Maxwells ligninger}
Dannelsen af $E$- og $B$-felterne beskrives af et sæt af fire ligninger, kaldet Maxwells ligninger. 
\begin{align*}
\v{\nabla}\cdot \v{E} &= \frac{\rho}{\varepsilon_0}\\
\v{\nabla}\cdot \v{B} &= 0\\
\v{\nabla}\times \v{E} &= -\frac{\partial \v{B}}{\partial t}\\
\v{\nabla}\times \v{B} &= \mu_0\left(\varepsilon_0\frac{\partial \v{E}}{\partial t} + \v{j}\right)
\end{align*}

Den første kaldes Gauss lov, og siger at $\v{E}$ felter udstråler fra elektriske ladninger. Her er $\rho$ tætheden af de elektriske ladninger.
Den anden har ikke et officielt navn, men kaldes ofte Gauss lov for magnetisme, p.g.a. ligheden imellem de to. Den siger at der ikke findes magnetiske ladninger (monopoler).
Den tredje kaldes Faradays lov, og siger at en ændring i et $\v{B}$-felt skaber et $\v{E}$-felt.
Den sidste, kaldet Amperes lov, siger at en ændring i et $\v{E}$ felt giver et $\v{B}$-felt. Den siger også at man kan danne et $\v{B}$-felt med en elektrisk strøm. $\v{j}$ strømtætheden, altså  strømstyrke per volumen.

Alt i elektrodynamik kan udledes fra disse fire love, og en femte der beskriver hvordan ladede partikler påvirkes af elektromagnetiske felter kaldet Lorentz kraften: $\v{F} = q(\v{E}+\v{v}\times\v{B})$.
\section{Bølgeligningen}
De fleste har nok hørt at lys også kaldes for elektromagnetiske bølger, men hvordan kan det være? 

Grundlæggende set er en bølge en forstyrelse der udbreder sig, ofte i et medie. Det kan være krusningerne på overfladen af en dam, eller svingningerne i luften, vi opfanger som lyd. Mange fysiske fænomener beskrives som løsninger til anden ordens differentialligninger, bølger er ingen undtagelse. Ligningen bag bølgefænomener kaldes logisk nok for bølge ligningen. I en dimmension er den:
$$
\frac{\partial^2 f}{\partial x^2} = \frac{1}{v^2}\frac{\partial^2 f}{\partial t^2}
$$
Det viser sig at bølgeligningen kan løses af alle funktioner på formen: $f(x\pm vt)$ hvor $f$ er en vilkårlig funktion. Af særlig interesse er dog sinusbølger. De skrives ofte med cosinus:
$$
f = A\cos\left(2\pi\left(\frac{x}{\lambda} - ft\right)+\phi\right) = A\cos\left( kx-\omega t+\phi\right)
$$
Her er $\lambda$ bølgelængden, afstanden imellem to bølgetoppe $f$ er frekvensen, hvor mange toppe paserer et punkt over tid. $\phi$ er en forskydning af bølgen. $k$ og $\omega$ kaldes bølgetallet og vinkelfrekvensen, de spiller samme rolle som bølgelængden og frekvensen, men inkluderer de $2\pi$. Sammen hængen imellem dem er: $$k=\frac{2\pi}{\lambda}~~ og ~~f = 2\pi\omega$$ For at opfylde bølgeligningen må det også gælde at:
$$
v = f\lambda = \frac{\omega}{k}
$$
\section{Elektromagnetiske bølger}
I vakuum er der hverken ladninger eller elektriske strømme, så maxwells ligninger forsimples en anelse.
\begin{align*}
\v{\nabla}\cdot \v{E} &= 0\\
\v{\nabla}\cdot \v{B} &= 0\\
\v{\nabla}\times \v{E} &= -\frac{\partial \v{B}}{\partial t}\\
\v{\nabla}\times \v{B} &= \mu_0\varepsilon_0 \frac{\partial \v{E}}{\partial t}
\end{align*}
Vi starter med at tage rotationen af $E$ to gange. Efter Faradays lov er  det:
$$
\v{\nabla}\times (\v{\nabla}\times\v{E}) = -\v{\nabla}\times\frac{\partial \v{B}}{\partial t}
$$
Siden $\v{\nabla}\times$ og $\frac{\partial}{\partial t}$ differentierer med hensyn til forskellige vairable er deres rækkefølge underordnet. Det tillader os at indsætte Amperes lov på højre side:
$$
\v{\nabla}\times (\v{\nabla}\times\v{E})  = -\frac{\partial}{\partial t}(\v{\nabla}\times \v{B}) = -\mu_0\varepsilon_0\frac{\partial^2 \v{E}}{\partial t^2}
$$
Vi er nu færdig med højre side, men højre side kan gøre simplere. Dobbelt rotation viser sig at være: 
$$\v{\nabla}\times (\v{\nabla}\times\v{F}) =\v{\nabla}(\v{\nabla}\cdot \v{F})- \v{\nabla}^2\v{F} = \v{\nabla}(\v{\nabla}\cdot \v{F})-\frac{\partial^2 \v{F}}{\partial x^2}-\frac{\partial^2 \v{F}}{\partial y^2}-\frac{\partial^2 \v{F}}{\partial z^2}
$$
Da $\v{\nabla}\cdot\v{E}=0$ findes differentialligningen:
$$
\frac{\partial^2\v{E}}{\partial x^2}+\frac{\partial^2\v{E}}{\partial y^2}+\frac{\partial^2\v{E}}{\partial z^2} = \mu_0\varepsilon_0\frac{\partial^2 \v{E}}{\partial t^2}
$$
Dette er egentligt en bølgeligning, men vi kan reducere det til en en dimmesionel bølgeligning ved at antage at $E$ altid ligger langs $x$ aksen.
$$
\frac{\partial^2 \v{E}}{\partial x^2} = \mu_0\varepsilon_0\frac{\partial^2\v{E}}{\partial t^2}
$$
Der er intet særligt ved $x$-aksen, og $E$-feltet kunne lige så godt have enhver anden retning i $xy$-planen. $E$-feltet er dog altid vinkelret på udbredelsesretningen, så alt lys er transversale bølger. Retningen af $E$-feltet angiver lystets polarisering. $E$-feltet fra polariseret lys kan skrives:
$$
\v E = \begin{pmatrix}
E_x\cos(kx-\omega t)\\
E_y\cos(kx-\omega t)\\
0
\end{pmatrix}=
\begin{pmatrix}
E_x\\E_y\\0
\end{pmatrix}
\cos(kx-\omega t)
$$
Her ligger al informationen om lystets polarisering i vektoren, så hvis man arbejder med polariseret lys bruges ofte Jones vektorer. Det er en to dimmensionel vektor med $x$ og $y$ komposanterne.
\begin{table}[h]
\center
\begin{tabular}{c|c r r}
Polarisering &$\v E$-felt & Jones vektor & polariseringsvinkel\\\hline
Vandret & $E_x\cos(kx-\omega t)\hat{\v x}$ & $\dbinom{1}{0}$&$0^\circ$\\
Lodret & $E_y\cos(kx-\omega t)\hat{\v y}$ & $\dbinom{0}{1}$&$90^\circ$\\
Diagonalt & $\dfrac{E}{\sqrt{2}}\cos(kx-\omega t)(\hat{\v x}+\hat{\v y})$ & $\frac{1}{\sqrt{2}}\dbinom{1}{1}$&$45^\circ$\\
antidiagonalt & $\dfrac{E}{\sqrt{2}}\cos(kx-\omega t)(\hat{\v x}-\hat{\v y})$ & $\dfrac{1}{\sqrt{2}}\dbinom{1}{-1}$&$-45^\circ$\\
\end{tabular}
\caption{Forskellige polariseringer og deres Jones vektorer}
\end{table}

\begin{opgave}{Harmonisk oscillator med $\op a_- \op a_+$}{1}
\label{kvant:opg:amap}
Færdiggør udledningen af den harmoniske oscillator. Hvis du havde problemer med denne udledning er denne opgave stærkt anbefalet.
\opg Udregn $\op a_-\op a_+$.
\opg brug dette til at udlede ligning \eqref{kvant:eq:Hamap}.
\opg Vis at hvis $\psi_n$ er en løsning til Schrödingerligningen, så er $\op a_-\psi_n$ det også.
\opg Hvad er energien af $\op a_- \psi_n$
\end{opgave}

\begin{opgave}{Generering af nye tilstande.}{2}
%Jeg gad virkeligt ikke lave et facit til denne opgave, den skal nok fjernes for at være røvsyg.
Start med grundtilstanden for den harmoniske oscilator $\psi_0$ og generer de første exciterede tilstande:
\opg $\psi_1$
\opg $\psi_2$
\opg $\psi_3$
\opg $\psi_4$
\end{opgave}

\begin{opgave}{Det klassisk tilladte område.}{2}
Vi vil her sammenligne den kvanteharmoniske oscillator med en klassisk harmonisk oscillator med samme $V$ og $E=\frac{\hbar\omega}{2}$
\opg Hvad er den maksimale værdi af $x$?
\opg
Hvad er sandsynligheden for at finde en partikel i $\psi_0$ tilstanden i intervallet fra $-x_\text{max}$ til $x_\text{max}$?

(Hint: $\integral{xe^{-ax^2}}{x}{}{} = \frac{e^{-ax^2}}{2a}+k$.)
\end{opgave}

\begin{opgave}{Sjov med operatorer}{2}
\label{kvant:opg:sjov}
Vi vil her komme ind på en af grundene til, at hæve-/sænkeoperatorerne er smarte. Udenyt at bølgefunktionerne er ortonormale.
\opg Udtryk $\op x$ og $\op p$ med $\op a_+$ og $\op a_-$.
\opg Find $\expect x$ og $\expect p$ for $\psi_0$, $\psi_1$ og $\psi_{n}$.
\opg Gør det samme for $\expect{x^2}$ og $\expect{p^2}$.
\opg Hvad er $\sigma_x$ og $\sigma_p$ for $\psi_n$?
\opg Hvor dan passer det med Heisenbergs usikkerhedsprincip?

Husk: $\op a_\pm = \frac{1}{\sqrt{2 m \hbar\omega}}(\mp i\op p+m\omega \op x)$
\end{opgave}

\begin{opgave}{Nu med tid}{2}
Til tiden $t=0$ har vi bølgefunktionen: $$\Psi(x,t=0) = N(\psi_0+\psi_1)$$ (Det kan være en fordel at have lavet opgave \ref{kvant:opg:sjov} først.)
\opg Hvad er $N$?
\opg Hvad er $\Psi(x,t)$?
\opg Hvad er $\expect E$?
\opg Hvad er $\expect {x(t)}$?
\end{opgave}
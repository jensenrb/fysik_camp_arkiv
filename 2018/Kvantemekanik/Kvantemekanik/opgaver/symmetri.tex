\begin{opgave}{Lige og ulige funktioner}{1}
En funktion så som $f(x) = x^2$ der opfylder kravet: $f(x) = f(-x)$ kaldes en lige funktion. Tilsvarende er funktioner som $f(x) = x$ der opfylder det ligende krav: $f(x)=-f(-x)$.
Det vil sige at en lige funktion er uændret hvis man spejler den i $y$-aksen, mens en ulige funktion skifter fortegn.
Bemærk at de fleste funktioner er hverken lige eller ulige, og unikt er funktionen $f(x) = 0$ både lige og ulige.
Afgør om følgende funktioner er lige eller ulige:
\opg $\sin x$
\opg $e^{x^2}$
\opg $\cos x$
\end{opgave}

\begin{opgave}{Mere om lige og ulige funktioner}{1}
Lad $f_g(x)$ være en lige funktion og $f_u(x)$ være en ulige funktion\footnote{$g$ og $u$ står for gerate og ungerate, de tyske ord for lige og ulige.}.
\opg Vis at produktet af lige og ulige funktioner fungerer på samme måde som produktet af lige og ulige tal.
\opg Er $\frac{1}{f_{g}(x)}$ lige eller ulige?
\opg Hvad med :$\frac{1}{f_{u}(x)}$?
\opg Hvad er regnereglen med division?
\end{opgave}
\begin{opgave}{Sammensætning af lige og ulige funktioner}{2}
Alle funktioner kan skrives som en unik sum af en lige og en ulige funktion: $f(x) = f_g(x)+f_u(x)$. 
\opg Skriv $f(-x)$ ud fra $f_g(x)$ og $f_u(x)$.
\opg Skriv $f_g(x)$ og $f_u(x)$ ud fra $f(x)$ og $f(-x)$.
\opg Hvad er den lige og den ulige komposant af eksponentialfunktionen $e^x$?
\end{opgave}

\begin{opgave}{Integraler af lige og ulige funktioner.}{3}
Vi vil her finde nogle meget praktiske regneregler for integraler af lige og ulige funktioner over et symmetrisk interval.
Lad $f_g(x)$ være en lige funktion og $f_u(x)$ være en ulige funktion.
Antag derudover at $\integral{f_g(x)}{x}{0}{a}$ og $\integral{f_u(x)}{x}{0}{a}$ er kendte.
\opg Vis at $\integral{f_g(x)}{x}{-a}{a} = 2\integral{f_g(x)}{x}{0}{a}$
\opg Vis at $\integral{f_u(x)}{x}{-a}{a} = 0$
\opg Brug dette til at løse integralet: $$\integral{x\cos(x)\sin(x)+x^2-x\exp(x^2)}{x}{-1}{1}$$
\end{opgave}


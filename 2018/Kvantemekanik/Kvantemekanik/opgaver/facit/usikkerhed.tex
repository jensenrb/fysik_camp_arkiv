\begin{opgave}{Parabelformet bølgefunktion igen}{1}
Denne opgave bygger videre på opgave \ref{kvant:opg:parabel}, så det er en fordel at have lavet denne opgave først. Vi ser igen på en parabelformet bølgefunktion i en uendelig brønd:
$$
\psi=Nx(L-x)
$$
\opg Hvad er $\sigma_x^2 = \expect{x^2}-\expect{x}^2$?

Vi kender allerede $\expect{x}$ og $\abs N^2$ fra opgave \ref{kvant:opg:parabel}.
\begin{align*}
\expect{x} &= \frac{L}{2}\\
\abs N^2 &= \frac{30}{L^5}
\end{align*}
For at finde $\expect{x^2}$ bruges sandwichen:
\begin{align*}
    \expect{x^2}&= \abs N^2 \integral{x^3(L-x)}{x}{0}{L}\\
    &= \frac{30}{L^5}\integral{x^5-2x^4L+x^3L^2}{x}{0}{L}\\
    &= \frac{30}{L^5}\left[\frac{x^6}{6}-\frac{2x^5L}{5}+\frac{x^6}{6}\right]_0^L\\
    &= 30 L^2\left(\frac{15}{60}-\frac{24}{60}+\frac{10}{60}\right)\\
    &= \frac{L^2}{2}
\end{align*}
Nu er usikkerheden:
$$
\sigma_x^2 = \expect{x^2}-\expect{x}^2 = \frac{L^2}{2}-\frac{L^2}{4}=\frac{L^2}{2}
$$
og
$$
\sigma_x = \frac{L}{\sqrt{2}}
$$
\opg Hvad er $\sigma_p^2 = \expect{p^2}-\expect{p}^2$?
Her skal vi bruge at $\op p = -i\hbar\pdif{x}{}$. Så bliver sandwichen:
\begin{align*}
    \expect{p} &= -i\hbar\abs N^2\integral{x(L-x)\pdif{x}{}\left(x(L-x)\right)}{x}{0}{L}\\
    &= \frac{-30i\hbar }{L^5}\integral{x(L-x)(L-2x)}{x}{0}{L}\\
    &= \frac{-30i\hbar }{L^5}\integral{xL^2-3x^2L+2x^3}{x}{0}{L}\\
    &= \frac{-30i\hbar }{L^5}\left[\frac{x^2L^2}{2}-x^3L+\frac{x^4}{2}\right]_0^L\\
    &= \frac{-30i\hbar }{L} \left(\frac{1}{2}-1+\frac{1}{2}\right)\\
    &=0
\end{align*}
Tilsvarende for $\expect{p^2}$, hvor $op p^2 = -\hbar^2\pdif[2]{x}{}$.
\begin{align*}
    \expect{p^2} &= \abs N^2 \integral{x(L-x)\pdif[2]{x}{}\left(x(L-x)\right)}{x}{0}{L}\\
    &= \frac{-30\hbar^2 }{L^5}\integral{x(L-x)(-2)}{x}{0}{L}\\
    &= \frac{60\hbar^2}{L^5}\integral{xL-x^2}{x}{0}{L}\\
    &= \frac{60\hbar^2}{L^5}\left[\frac{x^2L}{2}-\frac{x^3}{3}\right]_0^L\\
    &= \frac{60\hbar^2}{L^5}\left(\frac{L^3}{2}-\frac{L^3}{3}\right)\\
    &= \frac{60\hbar^2}{L^2}\left(\frac{3}{6}-\frac{2}{6}\right)\\
    &= \frac{10\hbar^2}{L^2}
\end{align*}
Så usikkerheden er:
$$
\sigma_p^2 = \expect{p^2} = \frac{10\hbar^2}{L^2}
$$
og:
$$
\sigma_p = \frac{\hbar\sqrt{10}}{L}
$$
\opg Passer det med Heisenbergs usikkerhedsprincip?

Sættes $\sigma_x$ og $\sigma_p$ ind i Heisenbers usikkerhedsprincip findes:
$$
\sigma_x\sigma_p = \frac{L}{\sqrt{2}}\frac{\hbar\sqrt{10}}{L}= \hbar\sqrt{5}\geq\frac{\hbar}{2}
$$
Heisenberg er tilfreds $\ddot \smile$
\end{opgave}

\begin{opgave}{Den frie partikel}{2}
En fri partikel er en partikel der ikke påvirkes af noget potentiale, så $V(x)=0$ for alle $x$
\opg Hvad er $\op H$?

Der er ikke noget potentiale så:
$$
\op H = \frac{-\hbar^2}{2m}\pdif[2]{x}{} = \frac{\op p^2}{2m}
$$
\opg Hvad er $[\op p,\op H]$?

Bemærk at operatorer altid kommuterer med sig selv, så:
\begin{align*}
    [\op p,\op H]&= \op p\op H-\op H\op p\\
    &= \frac{\op p^3}{2m}-\frac{\op p^3}{2m}\\
    &= 0
\end{align*}
\opg Hvilke bølgefunktioner opfylder: $\op p \psi_p = -i\hbar\pdif{x}{\psi_p} = p\psi_p$?

Omskrives dette findes differentialligningen:
$$
\pdif{x}{\psi_p}=\frac{ip}{\hbar}
$$
Løsningen til denne differentialligning er blot en eksponentialfunktion. Imaginærfaktoren ændrer ikke dette, så:
$$
\psi_p = e^{\frac{ipx}{\hbar}}
$$
\opg Hvad sker der hvis man sætter $\psi_p$ ind i Schrödingerligninen?
\begin{align*}
    \op H\psi_p &= \frac{-\hbar^2}{2m}\pdif[2]{x}{\psi_p}\\
    &= \frac{-\hbar^2}{2m}\left(\frac{ip}{\hbar}\right)^2\psi_p\\
    &= \frac{-\hbar^2}{2m}\frac{-p^2}{\hbar^2}\psi_p\\
    &=\frac{p^2}{2m}\psi_p
\end{align*}
\opg Hvad er sammenhængen imellem $E$og $p$?

Energien er:
$$
E=\frac{p^2}{2m}
$$
\opg Hvad er $\sigma_p$ og $\sigma_E$?

Først findes $\expect{p}$, $\expect{p^2}$, $\expect{E}$ og $\expect{E^2}$.
\begin{align*}
    \expect{p} &= \braket{\psi_p}{\op p\psi_p}\\
    &= p\braket{\psi_p}{\psi_p}\\
    &=p
\end{align*}
Dette er en konsekvens af at det er en egentilstand. Udregningen er tilsvarende for de andre.
\begin{align*}
    \expect{p^2} &= p^2\\
    \expect{E} &= E\\
    \expect{E^2} &= E^2
\end{align*}
Det gør det muligt at finde usikkerhederne:
\begin{align*}
    \sigma_p &= \expect{p^2}-\expect p^2 = 0\\
    \sigma_E &= \expect{E^2}-\expect E^2 = 0
\end{align*}
Man kan godt kende $p$og $E$ på samme tid for den frie partikel, og det er ved $\psi_p$ tilstandene.
\end{opgave}
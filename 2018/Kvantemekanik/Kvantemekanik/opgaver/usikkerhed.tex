\section*{Usikkerhedsrelationen}
\begin{opgave}{Parabelformet bølgefunktion igen}{1}
Denne opgave bygger videre på opgave \ref{kvant:opg:parabel}, så det er en fordel at have lavet denne opgave først. Vi ser igen på en parabelformet bølgefunktion i en uendelig brønd:
$$
\psi=Nx(L-x)
$$
\opg Hvad er $\sigma_x^2 = \expect{x^2}-\expect{x}^2$?
\opg Hvad er $\sigma_p^2 = \expect{p^2}-\expect{p}^2$?
\opg Passer det med Heisenbergs usikkerhedsprincip?
\end{opgave}
%
\begin{opgave}{Den frie partikel}{2}
En fri partikel er en partikel der ikke påvirkes af noget potentiale, så $V(x)=0$ for alle $x$
\opg Hvad er $\op H$?
\opg Hvad er $[\op p,\op H]$?
\opg Hvilke bølgefunktioner opfylder: $\op p \psi_p = -i\hbar\pdif{x}{\psi_p} = p\psi_p$?

OBS: $\op p$ er en operator og $p$ er et tal.
\opg Hvad sker der hvis man sætter $\psi_p$ ind i Schrödingerligninen?
\opg Hvad er sammenhængen imellem $E$og $p$?
\opg Hvad er $\sigma_p$ og $\sigma_E$?
\end{opgave}
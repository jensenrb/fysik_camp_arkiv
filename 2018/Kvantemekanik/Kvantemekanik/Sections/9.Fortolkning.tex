\documentclass[../Kvantemekanik.tex]{subfiles}
 
\begin{document}

\section{Fortolkninger af kvantemekanikken}

Et kriterium for en god fysisk teori er, at den er i stand til at forudsige observationer. Her brillierer kvantemekanikken. Der er dog uenighed om hvad udregningerne repræsenterer.

I løbet af dette kapitel har vi set på partiklernes position som ukendt indtil de observeres. I kvantemekanikken er en observation ikke begrenset til at der er en person der ser systemet. Mange kvantesystemer er mindre end bølgelængden af synligt lys, så de er umuligt at observere direkte. Istedet er en observation en hver vekselvirkning med omgivelserne. Det kunne for eksempel være lys der bliver absorberet af et atom.

Mange af de mystiske kvanteeffekter bliver sat på spidsen i Københavnerfortolkningen. Mange fysikere, heriblandt Einstein har igennem tiden være utilfreds med det. Der er igennem årene blevet udviklet en del alternativer, hvor en partikels placering blot er ukendt i stedet for at være ubestemmelig indtil observation.
Den nok mest kendte af disse er de Brolige-Bohm pilotbølge modellen. Her rider partiklen på en bølge der på mange måder minder om bølgefunktionen.


\end{document}
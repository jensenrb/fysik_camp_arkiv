\documentclass[../Kvantemekanik.tex]{subfiles}
 
\begin{document}
Velkommen til forløbet i kvantemekanik. Formålet med dette forløb er at give en ordentlig introduktion til hvad kvantemekanik handler om, samt at give dig nogle værktøjer, så du bliver i stand til at regne på kvantemekaniske problemer. Selvsagt vil der være massere af ting der vil blive udeladt i dette forløb da tiden er knap (på mange universiteter vil man have mulighed for at tage kvant kurser i sammenlagt mere end 12 måneder, hvis man vil forstå alle aspekter af kvantemekanik), men dette skyldes primært at man i kvantemekanik har mange specialtilfælde og at ligningerne man prøver at løse hurtigt bliver så kompliceret at der ikke findes analytiske løsninger til dem. Grundprincipperne og de få ting man kan løse analytisk i kvantemekanik viser sig faktisk at kunne formuleres på relativt lidt plads, og det er de fleste af disse som vi vil præsentere for dig i dette forløb. Det betyder dog ikke at det bliver nemt, da matematikken sagtens kan drille og tankegangen man skal benytte er helt anderledes end hvad du er vant til fra klassisk mekanik. Så hvis du er klar på en ekstra udfordring, så lad os bare springe ud i det!

 

\end{document}
\section*{Harmoniske oscillatorer}
\begin{opgave}{Harmonisk oscillator med $\boldsymbol{\op a_- \op a_+}$}{1}
\label{kvant:opg:amap}
Færdiggør udledningen af den harmoniske oscillator. Hvis du havde problemer med denne udledning er denne opgave stærkt anbefalet.
\opg Udregn $\op a_-\op a_+$.
\opg Brug dette til at udlede ligning \eqref{k-kvant:eq:Hamap} i kompendiet.
\opg Vis at hvis $\psi_n$ er en løsning til Schrödingerligningen, så er $\op a_-\psi_n$ det også.
\opg Hvad er energien af $\op a_- \psi_n$
\end{opgave}
%
\begin{opgave}{Sjov med operatorer}{2}
\label{kvant:opg:sjov}
Vi vil her komme ind på en af grundene til, at hæve-/sænkeoperatorerne er smarte. Udnyt at bølgefunktionerne er ortonormale, og husk at
$$
\op a_\pm = \frac{1}{\sqrt{2 m \hbar\omega}}(\mp i\op p+m\omega \op x) \, .
$$
\opg Udtryk $\op x$ og $\op p$ ved $\op a_+$ og $\op a_-$.
\opg Find $\expect x$ og $\expect p$ for $\psi_0$, $\psi_1$ og $\psi_{n}$.
\opg Gør det samme for $\expect{x^2}$ og $\expect{p^2}$.
\opg Hvad er $\sigma_x$ og $\sigma_p$ for $\psi_n$?
\opg Hvordan passer det med Heisenbergs usikkerhedsprincip?
\end{opgave}
%
\begin{opgave}{Nu med tid}{2}
Til tiden $t=0$ har vi bølgefunktionen: $$\Psi(x,t=0) = N(\psi_0+\psi_1)$$ (Det kan være en fordel at have lavet opgave \ref{kvant:opg:sjov} først.)
\opg Hvad er $N$?
\opg Hvad er $\Psi(x,t)$?
\opg Hvad er $\expect E$?
\opg Hvad er $\expect {x(t)}$?
\end{opgave}
%
\begin{opgave}{Molekylære vibrationer}{2}
En god model for bindingen i et molekyle er Morsepotentialet:
$$
V(r) = D\left(1-e^{-(r-R)}\right)^2
$$
\opg Hvor er potentialets minimum (ligevægtsafstanden)?
\opg Hvad er minimumsværdien af potentialet?
\opg Hvad er potentialet for meget store $r$.
\opg Hvad er $\pdif[2]{r}{V}$ i ligevægtspunktet. Dette er kraftkonstanten $k$ analogt med for en fjeder.
\opg Molekylet vil kunne vibrere omkring ligevægtspunktet. Hvad er $\omega$?
\opg Morse potentialet kan tilnærmes som en harmonisk oscillator. Hvad er grundtilstandsenergien for denne?\\ \\
For et brintmolekyle er Morsepotentialet givet ved:
\begin{align*}
D&=\SI{7.24e-19}{J} \, ,\\
a &= \SI{3.93e10}{m^{-1}} \, ,\\
R &= \SI{7.40e-11}{m} \, ,\\
m_p&=\SI{1.67e-27}{kg} \, .
\end{align*}
\opg Hvad er $\omega$ og $E_0$ for brintmolekylet.
\end{opgave}
\section*{Uendeligt dyb brønd}
\begin{opgave}{Parabelformet bølgefunktion}{1}\label{kvant:opg:parabel}
Vi ser her på en partikel i en uendelig brønd i intervallet fra $0$ til $L$. Lad bølgefunktionen være:
$$
\psi = Nx(L-x) 
$$
\opg Find $N$ så bølgefunktionen er normeret.
\opg Hvad er forventningsværdien for positionen $\expect x$?
\opg Find forventningsværdien for energien $\expect E$ og sammenlign den fundne energi med grundtilstandsenergien: $\frac{\pi^2\hbar^2}{2mL^2}$.
\opg Er $\psi$ en stationær tilstand?
\end{opgave}
%
\begin{opgave}{Sammensatte bølgefunktioner}{1}
Find normeringskonstanten $N$ og energien $E$ for de følgende bølgefunktioner, der er sammensat af stationære tilstande for den uendelige brønd.
\opg $N(\psi_1+\psi_2)$.
\opg $N(\psi_1-\psi_3)$.
\opg $N(\psi_1+\psi_2-2\psi_3)$.\\
Hint: Udnyt at bølgefunktionerne er ortonormale.
\end{opgave}
%
\begin{opgave}{Den tidsafhængige bølgefunktion}{2}
I en uendelig brønd er bølgefunktionen til tiden $t=0$:
$$
\Psi(x,0) = \frac{1}{\sqrt{5}}(2\psi_1+\psi_2) 
$$
\opg Hvad er $\Psi(x,t)$? Du kan med fordel bruge 
$$
\omega = \frac{E_1}{\hbar} = \frac{\pi^2\hbar}{2mL^2} \, .
$$
\opg Hvad er $\Psi^*(x,t)$?
\opg Skriv $\Psi^*x\Psi$ så simpelt som muligt.
\opg Hvad er $\expect{x(t)}$? \\
Hint:
\begin{align*}
e^{i\theta}+e^{-i\theta} &= 2\cos \theta \, , \\ \matrixel{\psi_n}{x}{\psi_n} &= \frac{L}{2} \, , \\  \matrixel{\psi_1}{x}{\psi_2} &= \frac{-16L}{9\pi^2} \, .
\end{align*}
\end{opgave}
%
\begin{opgave}{En partikel i et kvadrat}{3}
I to dimensioner er den tidsuafhængige Schrödingerligning i kartesiske koordinater:
$$
E\psi(x,y) = \frac{-\hbar^2}{2m}\left(\pdif[2]{x}\psi +\pdif[2]{y}\psi\right) + V(x,y)
$$
Vi vil se på en kvadratisk brønd, i to dimensioner, med sidelængder på $L$. Her er potentialet nul, når $0\leq x\leq L$ og $0\leq y\leq L$.
Antag nu at man kan skrive bølgefunktionen som:
 $$\psi(x,y) = X(x)Y(y) = XY$$
\opg Indsæt $\psi = XY$ i Schrödingerligningen med $V=0$ og isoler $E$.
\opg Energien vil bestå af en bidrag fra $X$ og $Y$, så $E=E_x+E_y$. Opstil differentialligninger i stil med ligning \eqref{k-kvant:eq:infb} i kompendiet for $X$ og $Y$.
\opg Find generelle løsninger til differentialligningerne, bølgefunktionen og de tilhørende energier. Bemærk at der vil være et kvantetal for hver af differentialligningerne. 
\opg Hvad er de fem laveste energier? Skitser bølgefunktionerne med disse energier.
\end{opgave}
%
\begin{opgave}{En partikel i en boks}{3}
I tre dimensioner er den tidsuafhængige Schrödingerligning i kartesiske koordinater:
$$
E\psi(x,y,z) = \frac{-\hbar^2}{2m}\left(\pdif[2]{x}\psi +\pdif[2]{y}\psi+\pdif[2]{z}\psi\right) + V(x,y,z)
$$
Vi vil se på en kubisk boks med en sidelængde på $L$, hvor potentialet er nul, når $x$, $y$ og $z$ alle er imellem 0 og $L$.
Antag at man kan skrive bølgefunktionen som:
$$
\psi(x,y,z) = X(x)Y(y)Z(z) = XYZ
$$
\opg Indsæt $\psi = XYZ$ i Schrödingerligningen med $V=0$ og isoler $E$.
\opg Energien vil bestå af et bidrag fra $X$, $Y$ og $Z$, så $E=E_x+E_y+E_z$. Opstil differentialligninger i stil med ligning \eqref{k-kvant:eq:infb} i kompendiet for $X$, $Y$ og $Z$.
\opg Find generelle løsninger til differentialligningerne, bølgefunktionen og de tilhørende energier. Bemærk at der vil være et kvantetal for hver af differentialligningerne. 
\opg Find de laveste 5 mulige energier udtrykt i $E_1 = \frac{\pi^2\hbar^2}{2mL^2}$ energien for en  endimensionel uendelig brønd med samme brede som boksens sidelængde.
\end{opgave}
\chapter{Kvantemekanik}

%Her bør vi have en historisk motivation for at indføre kvantemekanik.

Kvantemekanikken beskriver hvordan verden virker på lille skala.
Det betyder blandt andet at vi ikke lægger mærke til kvante fenomener i vores hverdag, og og derfor kan mange kvantefenomener virke uintuitive.
Når man undersøger naturen på lille skala er den dog uundværlig.
Et af de første steder hvor den klassiske fysik var utilstrækkelig er lys.
På Newtons tid var der to modstridende beskrivelser af lyset.
I den ene beskrivelse var lyset strømme af partikler kaldet korpuskler.
I den anden var lyset bølger i et medie kaldet æteren.
Uden eksperimentelt belæg var det ikke muligt at afgøre hvilken model der bedst beskrev virkeligheden.
Det var først i begyndelsen af 1800-tallet at Youngs dobbeltspalte eksperiment viste at lys opførte sig som bølger.
Thomas Young sendte lys igennem to spalter og observerede et interferens mønster på en skærm bagved.
Interferens er et typisk bølgefenomen og ville ikke findes hvis lyset var partikler.
Dette samt Maxwells beskrivelse elektromagnetiske bølger betød at bølgemodellen blev gennerelt accepteret.
Bølge beskrivelsen havde dog problemer.
Helt grel var beskrivelsen af varmestråling.
Her var det muligt at opstille en model, Rayleigh-Jeans lov, der gav en glimrende beskrivelse ved lange bølgelængder.

$$
B_\lambda (T)= \frac{2c\sub{k}{B}T}{\lambda^4}
$$
Problemet er for korte bølgelængder hvor $\frac{1}{\lambda^4}$ ledert eksploderer, hvilket forudsage at alle legemer ville udsende uendelige mængder kortbølget lys. Denne såkaldte ultraviolette katastrofe forekommer tydelig vist ikke, da det ikke ville tillade liv som vi kender det i universet.

En tilfredstillende model blev fremstillet af Max Planck, men i hans udregninger antog han at lys kun kunne udsendes i pakker med en energi på:
\begin{equation}
E_\gamma = \frac{h}{\lambda}
\end{equation}
Planck så blot dette som et smart regnetrik, men idag ved vi at det var et af de første blik ind i kvantemekanikkens verden. Konstanten $h$ kaldet idag Plancks konstant og optræder i stort set alle ligninger der involverer kvantemekanik. Ofte buger man istedet Plancks reducerede konstant $\frac{h}{2\pi}=\hbar$ ($h$-streg).
Grunden til at kvantemekanikken er ubetydelig i vores hverdag er at Plancks konstant er så uhyre lille. 

\begin{align*}
h &= \SI{6.626e-34}{Js}\\
\hbar &= \SI{1.055e-34}{Js}
\end{align*}

At Planck ikke blot havde udført et smart regnetrik fandt man ud af ved at undersøge den fotoelektriske effekt. Hvis man sender ultraviolet lys ind på en metalplade vil lyset slå elektroner fri af metallet, hvilket kan måles. Øger man intensiteten af lyset slår man flere elektroner løs. Hvis man måler de løsrevne elektroners kinetiske energi finder man at der er en øvre grense for energien. Denne grense afhænger af lysets bølgelængde.
$$
\sub{T}{max} = \frac{h}{\lambda}-\sub{E}{binding}
$$
Nettop som man ville forvente hvis lyset var kvantiseret. Bølger kommer ikke i diskrete pakker, dette er en klar partikel egenskab.
lige pludeselig var partikkel modellen ikke helt så død som man havde troet. I 1924 i sin PhD. afhandling fremsatte de Broglie en model hvor ikke kun lys var både var både bølger og partikler, men også alt andet. For ting med masse er bølgelængden bestemt af impulsen(bevægelsesmængden) og dermed hastigheden af partiklen.
\footnote{I de Broglies oprindelige model blev partikler nes bevægelse bestemt af en pilotbølge. Denne fortolkning af kvantemekanikken blev ret hurtigt forkastet. Der er dog en modificeret udgave, de Broglie-Bohm fortolkningen, fra 50-erne.}
\begin{equation}
\lambda = \frac{h}{p} = \frac{h}{mv}
\end{equation}

Året efter opstillede Schrödinger en ligning, der senere er blevet opkaldt efter ham,der beskriver hvordan systemerne udvikler sig over tid. I en dimension er Schrödinger ligninen:

\begin{equation}
i\hbar\pdif{t}{\Psi}=-\frac{\hbar^2}{2m}\pdif[2]{t}\Psi+V(x,t)\Psi
\end{equation}

Lige som i klassisk mekanik er $V$ et potentiale.
Scrödinger ligningen spiller samme rolle som bevægelsesligninger gør i klassisk mekanik. Schrödingerligningen er netop på en form så løsningerne vil være bølger. I klassisk mekannik var målet i sidste ende at finde en sted funktion $x(t)$ der beskriver positionen til alle tider. I kvantemekanikken finder man istedet den såkaldte bølgefunktion: $\Psi(x,t)$. For bølgefunktionen  bruges normalt et stort græsk psi.

Bølgefunktionen beskriver ikke hvor partiklen er, men sandsynligheden for at finde partiklen i hvert punkt. Da $i$ indgår i Schrödinger ligningen vil bølgefunktionen være kompleks. Sandsynligheden kan derfor ikke være bølgefunktionen, da negative sandsynligheder ikke giver mening og imaginære sandsynligheder endnyu værre.
Istedet er det normkvadratet af bølgefunktionen der giver sandsynlighedstætheden for at finde partiklen i hvert punkt.
\begin{equation}
\rho(\text{partiklen er i }x\text{ til tiden }t)= \Psi(x,t)\Psi^*(x,t) = \abs{\Psi}(x,t)
\end{equation}

En sandsynlighedstæthed er indrette således at den samlede sandsynlighed for alle udfald er 100\%. Derfor er et krav vi kan stille til bølgefunktionen at dens normkvadrat integreret op over hele rummet er 1. En sådan bølgefunktion kaldes en nomaliseret bølgefunktion. 
\begin{equation}
\integral{\abs{\Psi}(x,t)}{x}{-\infty}{\infty} = 1
\end{equation}


Differentialligninger i flere variable kan være meget udfordrende at løse, og der er da heller ingen gennerelle løsninger til Schråodingerligningen. Schrödingerligningen er dog ikke helt slem. Den er det man kalder en lineær differentialligning. Det vil sige at summen af to løsninger også er en løsning og at en løsning gange et tal også er en løsning.

En særlig pæn type løsning er dem der kan skrives som produktet af en tidsuafhængig del og en rumuafhøngig del.
\begin{equation}
\Psi(x,t) = \psi(x)\phi(t)
\label{lig:xtsep}
\end{equation}
Det viser sig at hvis potentialet er tidsuafhængigt kan alle løsninger til schrödingerligningen skrives som en skaleret sum af denne type løsninger. Af denne grund vil vi begrense os til tidsuafhængige potentialer.
Indsættes \eqref{lig:xtsep}

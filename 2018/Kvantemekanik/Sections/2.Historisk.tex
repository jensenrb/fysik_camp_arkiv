\documentclass[../Kvantemekanik.tex]{subfiles}
 
\begin{document}
\section{Historisk perspektiv}
Historien bag opfindelsen af kvantemekanik er i sig selv utrolig fascinerende og underholdende at læse og høre om. Lige fra hvordan Heisenberg fik idéen til usikkerhedsrelationen i et badekar på loftet af Niels Bohr instituttet, over hvordan Bohr og Einstein glemte at stige af bussen under en ophedet diskussion, om Guds tilbøjelighed til hasardspil, til hvordan Schrödinger kom på sin berømte ligning, mens han var i et lysthus i de Schweiziske Alper sammen med en elskerinde \footnote{Oprindeligt brugte han ligningen som et modargument imod kvantemekanik, da han mente at den gav nogle absurde konsekvenser. Ironisk nok er hans ligning i dag grundlaget  for det hele}. Vi vil dog her holde os til de mest essentielle begivenheder, og lade anekdoterne vente til de sene aftener.


Kvantemekanikken beskriver hvordan verden virker på lille skala.
Det betyder blandt andet, at vi ikke lægger mærke til kvantefænomener i vores hverdag, og derfor kan mange kvantefænomener virke kontraintuitive.
Når man undersøger naturen på lille skala, er den dog uundværlig.
Et af de første steder, hvor den klassiske fysik var utilstrækkelig, er lys.
På Newtons tid var der to modstridende beskrivelser af lyset.
I den ene beskrivelse var lyset strømme af partikler kaldet korpuskler.
I den anden var lyset bølger, i et medie kaldet æteren.
Uden eksperimentelt belæg var det ikke muligt at afgøre hvilken model, der bedst beskrev virkeligheden.
Det var først i begyndelsen af 1800-tallet, at Youngs dobbeltspalteeksperiment viste, at lys opførte sig som bølger.
Thomas Young sendte lys igennem to spalter, og observerede et interferens mønster på en skærm bagved.
Interferens er et typisk bølgefenomen, og det ville ikke findes, hvis lyset var partikler.
Dette, samt Maxwells beskrivelse elektromagnetiske bølger, betød at bølgemodellen blev generelt accepteret.
Bølge beskrivelsen havde dog problemer.
Helt utilstrækkelig var beskrivelsen af varmestråling.
Her var det muligt at opstille en model, Rayleigh-Jeans lov, der gav en glimrende beskrivelse ved lange bølgelængder.
\begin{equation}
B_\lambda (T)= \frac{2c\sub{k}{B}T}{\lambda^4}
\end{equation}
Problemet er for korte bølgelængder, hvor $\frac{1}{\lambda^4}$ ledet eksploderer, hvilket forudsagde, at alle legemer ville udsende uendelige mængder kortbølget lys. Denne såkaldte ultraviolette katastrofe forekommer tydeligvist ikke, da det ikke ville tillade liv, som vi kender det, i universet.

En tilfredstillende model blev fremstillet af Max Planck, men i hans udregninger antog han, at lys kun kunne udsendes i pakker med en energi på:
\begin{equation}\label{kvant:fotonEnergi}
E_\gamma = \frac{hc}{\lambda}
\end{equation}
Planck så blot dette som et smart regnetrik, men i dag ved vi, at det var et af de første blik ind i kvantemekanikkens verden. Konstanten $h$ kaldes i dag Plancks konstant og optræder i stort set alle ligninger, der involverer kvantemekanik. Ofte bruger man i stedet Plancks reducerede konstant $\frac{h}{2\pi}=\hbar$ ($h$-streg). $c$ er lysets hastighed i vakuum.
Havde Plancks konstant været større, ville kvantemekanikken spille en større rolle i vores hverdag.

\begin{align*}
h &= \SI{6.626e-34}{\joule\second}\\
\hbar &= \SI{1.055e-34}{\joule\second}
\end{align*}

\begin{figure}[h!]
    \centering
    \includegraphics[width = 0.8\textwidth]{Kvantemekanik/billeder/dualitet.jpg}
\end{figure}


At Planck ikke blot havde udført et smart regnetrick, fandt man ud af ved at undersøge den fotoelektriske effekt. Hvis man sender ultraviolet lys ind på en metalplade, vil lyset slå elektroner fri af metallet, hvilket kan måles. Øger man intensiteten, af lyset slår man flere elektroner løs. Hvis man måler de løsrevne elektroners kinetiske energi finder man at der er en øvre grænse for energien. Denne grænse afhænger af lysets bølgelængde.
\begin{equation}
\sub{T}{max} = \frac{h}{\lambda}-\sub{E}{binding}
\end{equation}
Netop som man ville forvente, hvis lyset var kvantiseret. Bølger kommer ikke i diskrete pakker, dette er en klar partikel egenskab.
Pludseligt var partikel modellen ikke helt så død, som man havde troet. I 1924 i sin PhD. afhandling fremsatte de Broglie en model, hvor ikke kun lys var både bølger og partikler, men også alt andet. For ting med masse er bølgelængden bestemt af impulsen (bevægelsesmængden), og dermed hastigheden af partiklen.
\footnote{I de Broglies oprindelige model blev partiklernes bevægelse bestemt af en pilotbølge. Denne fortolkning af kvantemekanikken blev ret hurtigt forkastet. Der er dog en modificeret udgave, de Broglie-Bohm fortolkningen, fra 50'erne.}
\begin{equation}
\lambda = \frac{h}{p} = \frac{h}{mv}
\end{equation}

Året efter opstillede Schrödinger en ligning, der senere er blevet opkaldt efter ham, der beskriver hvordan systemerne udvikler sig over tid. 

\end{document}

\section{Kvantemekanikkens forunderlige egenskaber}
Løsningen til den uendelige potentialebrønd ser måske ikke ud af meget, men bliv ikke snydt, for dette resultat indeholder faktisk en hel del information om mange af de underlige egenskaber der er ved kvantemekanik, så lad os bruge en del energi på at gennemgå dem.
\subsection{Ortonormalitet}
For det første er det værd at bemærke at vi i ligningerne for energien og bølgefuntionen har benævnt dem begge med et subscript n. Dette er gjort, da alle de andre konstanter der indgår i udtrykkene er bestemt for os på forhånd, hvorimod n er et vilkårligt tal vi var nød til at indføre da matematikken krævede det. Dette n er skyld i at vores partikel ikke kan have alle tænkelige energier, men er begrænset til at have en ud af mange diskrete energier. Samtidig betyder det også at partiklen er begrænset til at have én ud af en række bestemte bølgefunktioner.

Betyder det at en partikel ikke kan bevæge sig frit i en sådan boks, og at den samtidig er nød til at have helt bestemte energier? Nej ikke helt, men for at forstå det, skal vi først lære lidt om hvordan disse bølgefunktioner fungerer.

Den første ting vi skal indse er at disse funktioner er indbyrdes ortonormale, ortonormale betyder at en funktion $\psi_n$ ikke kan skrives som en kombination af andre funktioner $\sum_m \psi_m$ for $m\neq n$. Dette kan ses ved at udføre integralet.
\begin{equation*}
    \integral{\psi_m^*\psi_n}{x}{0}{L} = \delta_{m,n}
\end{equation*}
hvor tegnet $\delta_{m,n}$ er det såkaldte kroenicka-delta tegn, der er 1 når $m=n$ og 0 når $m\neq n$. Dette integral kan tolkes som at funktionen $\psi_n$ ikke indeholder noget af funktionen $\psi_m$.

\subsection{Repræsnetation}
Ortonormalitet betyder at funktionerne $\psi_1, \psi_2...$ udspænder et komplet rum, hvor de hver især er uafhængige af hinanden, men ved at lave lineare kombinationer af disse funktioner, kan vi repræsentere alle andre funktioner vi kan tænke på inde i intervallet $x=0,L$, og alle disse funktioner vil have en unik kombination af $\psi_n$ funktioner at blive repræsenteret af. Man kan lidt tænke på dette som et koordinatsystem. Et punkt i et 3 dimensionelt kartesisk koordinatsystem $(a,b,c)$, kan beskrives ved tre enhedsvektorer $\hat{x}$, $\hat{y}$, $\hat{z}$ på følgende måde:
\begin{equation*}
    P(a,b,c)=a\hat{x}+b\hat{y}+c\hat{z}
\end{equation*}
Denne repræsentation er unik, da man ikke kan beskrive punktet $P(a,b,c)$ på andre måder i det kartesiske rum end den måde vi lige har vist. Derfor kan man se funktionerne $\psi_n$ som enhedsvektorer i et komplet rum, der gør det muligt at beskrive alle funktioner som en unik linear kombination af disse enhedsfunktioner, også kaldet egenfunktioner. Dette er en parallel til Taylorudvikling, her bruger vi bare ikke polynomier, men sinus funktioner som vores funktions base.

Dette kan måske virke som en underlig egenskab ved den uendelige potentialebrønd, men faktisk viser det sig at dette er en generel egenskab ved alle potentialer. Det er nemlig sådan at ved alle potentialer vi kan tænke på til Schrödinger ligningen vil løsningerne være et sæt af egenfunktioner der udspænder et komplet rum, og disse egenfunktioner vil hver især have en egenenergi $E_n$ knyttet til sig. Det vil samtidig være muligt at udtrykke alle tænkelige funktioner som en linear kombination af disse egenfuntioner. 

Men hvordan repræsenterer man så en vilkårlig funktion ud fra en egenfunktioner? Svaret på dette er faktisk ganske simpelt (omend matematikken godt kan være lidt drilsk), man spørger simpelthend hvor meget af funktionen der er indeholdt i hver af egenfunktionerne. Dette gøres på samme måde som vi før gjorde, da vi så hvor meget af egenfunktionerne der er indeholdt i de andre egenfunktioner (i det tilfælde var resusltatet 0). Så hvis vi har en funktion $f(x)$ der er normaliseret således at:
\begin{equation*}
    \integral{\abs{f(x)}^2}{x}{0}{L} = 1
\end{equation*}
så vil vi være i stand til at finde hvor meget af hver egenfunktion $\psi_n$ der er indeholdt i $f(x)$ ved at sige:
\begin{equation}
c_n=\int_{0}^{L}f^*(x)\psi_n dx  
c_n= \integral{f^*(x)\psi_n}{x}{0}{L}  
\end{equation}
Hvor
\begin{equation*}
    \sum_{n=1}^{\infty}|c_n|^2=1
\end{equation*}
Hvilket så betyder at vi kan udtrykke $f(x)$ som en linear kombination af egenfunktioner $\psi_n$ på følgende måde:
\begin{equation}
    f(x)=\sum_{n=1}^{\infty}c_{n}\psi_{n}
\end{equation}
\subsection{Superposition}
Dette er i sig selv et fantastisk resultat, men den opmærksomme læser vil allerede nu have stillet sig selv et meget dybt spørgsmål. For hvis partiklen befinder sig i tilstanden $f(x)$, hvad er så dens energi? For at besvare dette spørgsmål må vi henvende os til kvantemekanikkens tredje postulat, der handler om målingen af forventningsværdier, der siger at vi skal præsse operatoren for den værdi vi gerne vil måle ind imellem de to funktioner når vi tager integralet af funktionerne. Fra sidste afsnit ved vi yderligere at energioperatoren er den såkaldte hamiltonoperator, og vi har jo lige fundet alle egenenergierne til den uendelige potentialebrønd, så vi har alt hvad vi skal bruge for at finde svaret. Vi sætter derfor bare ind og får:
\begin{align*}
    \expect E&=\integral{f^*(x)\op H f(x)}{x}{0}{L}\\
    &=\integral{(c_1\psi_1+c_2\psi_2+...)^{*}\hat{H}(c_1\psi_1+c_2\psi_2+...)}{x}{0}{L}\\
    &=\integral{(c_1\psi_1+c_2\psi_2+...)^{*}(E_1c_1\psi_1+E_2c_2\psi_2+...)}{x}{0}{L}\\
    &=E_1\abs{c_1}^2\integral{\abs{\psi_1}^2}{x}{0}{L}+E_2\abs{c_2}^2\integral{\abs{\psi_2}^2}{x}{0}{L}+...\\
    &=E_1\abs{c_1}^2+E_2\abs{c_2}^2+E_3\abs{c_3}^2+E_4\abs{c_4}^2+...
\end{align*}
Hvor vi her har brugt at 
$$ \integral{\psi_{m}^{*}\psi_n}{x}{0}{L}=0$$
Ud fra dette kun man derfor tro at partiklen så har energien $\expect E$, hvilket så må medføre at partiklen dermed kan have alle energier, bare vi sørger for at forberede den i den rigtige tilstand $f(x)$, men dette er dog ikke tilfældet.
Det er nemlig sådan at en partikel kun kan befinde sig i sin egenfunktion til potentialet $\psi_n$. Vi skal jo huske på at $\expect E$ står for den gennemsnitlige energi vi burde finde hvis vi laver uendelig mange målinger på et system med funktionen $f(x)$. Men ved en enkelt måling er det kun muligt at måle energierne $E_n$.

Men hvordan ved vi hvilken $E_n$ vi så måler når vi laver en måling af systemet? Svaret er, at det ved vi ikke, men vi kender sandsynligheden for at få energien $E_n$, og denne sandsynlighed er netop koefficienten $\abs{c_n}^2$. 

Nu skal vi holde tungen lige i munden, for efter man har foretaget en måling a energien og fået af vide at partiklel har energien $E_n$, så betyder det at hvis vi foretager endnu en måling af energien lige bagefter, så må vi nødvendigvis få samme svar tilbage, da partiklen jo lige har sagt at den har energien $E_n$, og at den befinder sig i tilstanden $\psi_n$.
Det vi er her er stødt på er det man kalder superpositionsprincippet, hvilket siger, at hvis en partikel befinder sig i en bestemt tilstand
\begin{equation*}
    f(x)=\sum_{n=1}^{\infty}c_n\psi_n
\end{equation*}
og vi så måler på den, så tvinger vi partiklen til at beslutte sig for hvilken tilstand $\psi_n$, den gerne vil være i hvorefter den kollapser til denne tilstand, således at $f(x)$ ændres til:
\begin{equation*}
    f(x)=\psi_n
\end{equation*}
med andre ord, når vi måler på et system, så ændrer vi på systemet. Dette betyder ikke at partiklen så vil blive i tilstanden $\psi_n$ indtil vi igen måler på den, for den vil blive påvirket af alle mulige andre ting man ikke kan skærme den for, som f.eks. fotoner, interaktioner med andre partikler og kvantefluktioner, så den vil ret hurtigt igen blive ændret, til at være i en tilstand der er en linear kombination af egenfunktioner $\psi_n$, men det vigtige her er at denne tilstand ikke er den samme tilstand som vi startede med. 
Men hvorfor er det så vigtigt at vide hvad der sker når vi laver en linear kombination af egenfunktioner? Grunden til dette er at vi kun kan have bevægelse hvis vores partikel har en tilstand der er en linear kombination af egentilstande, hvilket vi nu skal til at vise.
\subsection{Tidsudvikling}
Vi skal først huske på at den generelle løsning til Schrödingerligningen er
$$\Psi_n(x,t)=\psi_n(x)\phi_n(t)=\psi_n(x)e^{\frac{-iE_nt}{\hbar}}$$
Så hvis vi vil få vores bølgefunktion til at udvikle sig over tid, så skal vi gange det tidslige led $\Phi_n(t)$ på hver af vores egentilstande. Derfor kan vi udtrykke en generel tidsafhængig tilstand således:
\begin{equation}
    f(x,t)=\sum_{n=1}^{\infty}c_n\Psi_n(x,t)=\sum_{n=1}^{\infty}c_n\psi_n(x)e^{\frac{-iE_nt}{\hbar}}
\end{equation}
Nu forestiller vi os så at vi har en tilstand der kun består af en egentilstand n=1, $f(x,t)=\psi_1e^{\frac{-iE_1t}{\hbar}}$, så vores sandsynlighedsfordeling $\abs{f(x,t)}^2$ kan derfor udregnes til at være:
\begin{equation*}
    \abs{f(x,t)}^2=\left(\psi_1e^{\frac{-iE_1t}{\hbar}}\right)^{*}\left(\psi_1e^{\frac{-iE_1t}{\hbar}}\right)=\psi_1^{*}\psi_1 e^{\frac{+iE_1t}{\hbar}}e^{\frac{-iE_1t}{\hbar}}=\abs{\psi_1}^{2}
\end{equation*}
som tydeligvis ikke er tidsafhængigt, da egentilstandende $\psi_n$ kun er funktioner af sted.

Hvis vi i stedet prøver at kigge på et tilstand $f(x,t)$ som består af to egentfunktioner, således at: $f(x,t)=c_1\Psi_1(x,t)+c_2\Psi_2(x,t)$, får vi følgende sandsynlighedsfordeling:
\begin{align*}
    \abs{f(x,t)}^2&=(c_1\Psi_1(x,t)+c_2\Psi_2(x,t))^{*}(c_1\Psi_1(x,t)+c_2\Psi_2(x,t))\\
    &=\abs{c_1}^{2}\abs{\psi_1}^{2}+\abs{c_2}^{2}\abs{\psi_2}^{2}+c_1^{*}c_2\Psi_1^{*}\Psi_2+c_1c_2^{*}\Psi_1\Psi_2^{*}\\
    &=\abs{c_1}^{2}\abs{\psi_1}^{2}+\abs{c_2}^{2}\abs{\psi_2}^{2}+c_1^{*}c_2\psi_1^{*}\psi_2e^{\frac{-i(E_2-E_1)t}{\hbar}}+c_1c_2^{*}\psi_1\psi_2^{*}e^{\frac{i(E_2-E_1)t}{\hbar}}\\
    &\text{Hvis vi antager at $c_1$, $c_2$ og $\psi_1$, $\psi_2$ er reelle får vi:}\\
    &=\abs{c_1}^{2}\abs{\psi_1}^{2}+\abs{c_2}^{2}\abs{\psi_2}^{2}+c_1c_2\psi_1\psi_2\cos\left(\frac{(E_2-E_1)t}{\hbar}\right)
\end{align*}
Hvilket viser at sandsynlighedsfordelingen for denne tilstand oscillerer som en cosinusfunktion med frekvensen $\frac{E_2-E_1}{\hbar}$. Nu er det sådan at siden sandsynlighedordelingen ændrer sig med tiden, så bør vi også forvente at forventningsværdien af positionen $\expect x$ ændrer sig med tiden. Dette kan vi vise først ved at se på:
\begin{equation*}
    \integral{\abs{f(x,t)}^2}{x}{0}{L}=\integral{\abs{c_1}^{2}\abs{\psi_1}^{2}+\abs{c_2}^{2}\abs{\psi_2}^{2}+c_1c_2\psi_1\psi_2\cos\left(\frac{(E_2-E_1)t}{\hbar}\right)}{x}{0}{L}=\abs{c_1}^2+\abs{c_2}^2=1
\end{equation*}
hvor vi har brugt at $\integral{\psi_1\psi_2}{x}{0}{L}=0$ grundet ortonormalitet. Hvis vi vil måle forventningsværdien af en størrelse skal vi indføre den mellem de to $\psi$ funktioner i integralet, hvilket for positionen i vores tilfælde giver resultatet:
\begin{align*}
    \expect x=\integral{f^*(x,t)xf(x,t))}{x}{0}{L}=\integral{\abs{c_1}^{2}\abs{\psi_1}^{2}+\abs{c_2}^{2}\abs{\psi_2}^{2}+c_1c_2\psi_1\psi_2\cos\left(\frac{(E_2-E_1)t}{\hbar}\right)}{x}{0}{L}=\abs{c_1}^2+\abs{c_2}^2=1
\end{align*}
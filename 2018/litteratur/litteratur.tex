\chapter{Litteratur}
Hvis vi har vakt jeres interesse er her en liste over bøger, der kan give en dybere indsigt i de forskellige emner, og det er disse, selv vi har konsulteret i løbet af skriveprocessen til dette kompendium.\footnote{Mange af bøgerne findes i mange forskellige versioner og endda fra forskellige forlag. Denne liste er bare de udgaver, kompendiets forfattere har benyttet.}
\subsection*{\ref{cha:Mekanik}~~Analytisk Mekanik}
\begin{itemize}
\item Taylor, John R., \textit{Classical Mechanics}, University Science Books, 2005
\item Young, Hugh \& Freedman, Roger, \textit{University Physics}, Pearson Education, 2016
\end{itemize}
\subsection*{\ref{cha:Kvant}~~Kvantemekanik}
\begin{itemize}
\item Griffiths, David J., \textit{Introduction to Quantum Mechanics}, Cambridge University, 2016
\end{itemize}
\subsection*{\ref{cha:Astro}~~Exoplaneter}
\begin{itemize}
\item Cockell, Charles, \textit{Astrobiology and the Search for Extraterrestrial Life}, MOOC af The University of Edinburgh
\item Jørgensen, Uffe Gråe, \textit{Exoplanets and Astrobiology}, 2017
\item Ryden, Barbara \& Peterson, Bradley M., \textit{Foundations of Astrophysics}, Pearson Education, 2010
\end{itemize}
\subsection*{\ref{cha:AMO}~~Atom- og Molekylefysik}
\begin{itemize}
\item Griffiths, David J., \textit{Introduction to Quantum Mechanics}, Cambridge University, 2016
\item Rayner-Canham , Geoff \& Overton, Tina, \textit{Descriptive Inorganic Chemestry}, W.H. Freeman and Company, 2014
\end{itemize}
\subsection*{\ref{cha:Planet}~~Planetbevægelse}
\begin{itemize}
\item Taylor, John R., \textit{Classical Mechanics}, University Science Books, 2005
\end{itemize}
\subsection*{\ref{cha:Optik}~~Geometrisk Optik}
\begin{itemize}
\item Young, Hugh \& Freedman, Roger, \textit{University Physics}, Pearson Education, 2016
\end{itemize}
%\subsection*{\ref{cha:matematik}~~Matematik}
%\begin{itemize}
%\item Steward, James, \textit{Calculus - Concepts \& Contexts}, Thomson Brooks/Cole, 2006
%\end{itemize}
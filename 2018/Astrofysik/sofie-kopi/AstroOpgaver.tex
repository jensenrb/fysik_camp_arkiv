\section{Opgaver}
\begin{opgave}{God opgave}{1}

Find noget
\opg Udled ligning \ref{Pstar}.
\opg Og nu denne ting. Wow.
\end{opgave}


\begin{opgave}{Gravitationelle linser}{1}
Ligning, hvor man beregner hvornår der fokuseres mest
\opg Denne ting
\opg Og nu denne ting. Wow.

(Hint: Udnyt at noget andet gælder)
\end{opgave}

\begin{opgave}{Atmosfærekrav}{3}
For at en planet kan have en stabil atmosfære er den nød til at kunne holde simple molekyler fanget i dens tyngdefelt. Massen af $O_2$ er $m_{O_2} = \SI{5,31e-26}{\kilo\gram}$.
\opg Benyt Newtons gravitationslov og Newtons 2. lov til at bestemme tyngdeaccelerationen på en planets overflade.
\opg Er det legemes kinetiske energi præcis ligeså stor som dens gravitationelle potentielle energi, vil den kunne undslippe det tyngdefelt det befinder sig i. Vis at denne hastighed, kaldet undvigelseshastigheden fra en planet, er givet som
\begin{align*}
	v_\mathrm{ecs}^2 = \frac{2M_pG}{R_p}
\end{align*}
\opg Det oplyses nu fra kinetisk gasteori at
\begin{align*}
	\frac{1}{2}m\bar{v}^2 = \frac{3}{2}k_BT
\end{align*}
hvor $\bar{v}$ er partiklerne i gassens gennemsnitlige fart, $m$ er partiklernes masse, $k_B$ er Boltzmanns konstant og $T$ er temperaturen i Kelvin. \\
Vis at
\begin{align*}
	\frac{M_p}{R_p} = \frac{3}{2}\frac{k_BT}{mG}
\end{align*}
hvis $v_\mathrm{ecs} = \bar{v}$.
\opg Er $v_\mathrm{ecs} = \bar{v}$ vil op mod halvdelen af molekyler med massen $m$ forsvinde væk fra planeten. Konkluder at for at en exoplanet kan fastholde $O_2$ i sin atmosfære, skal der om planeten gælde at
\begin{align} \label{eq:atm}
	\frac{M_p}{R_p} > \frac{3}{2}\frac{k_BT}{m_{O_2}G}
\end{align}
\opg Den gennemsnitlige temperatur på Jorden antages at være \SI{16}{\degree}. Opfylder Jorden ligning \ref{eq:atm}?
\end{opgave}

\begin{opgave}{Gravitationslinser}{1}
I afsnit \ref{sec:GravLinse} om gravitationslinsemetode blev det nævnt at begivenheden skal observeres af flere forskellige målinger.
\opg Hvorfor er 1 måling ikke nok?
\opg Hvorfor er det usandsynligt at observere samme planet med gravitationslinsemetoden mere en én gang?
\opg Hvordan medfører ovenstående at planeten skal kunne ses i flere forskellige uafhængige målinger, for at kunne kaldes en planet?
\end{opgave}

\begin{opgave}{Bose-Einstein kondensat og laserkøling}{3}
For at lave Bose-Einstein kondensater i laboratoriet skal man have nogle lette partikler, der køles ekstremt meget ned.\footnote{For den interesserede læser er et Bose-Einstein kondensat at atomernes bølgefunktioner udvider sig, som temperaturen falder, indtil de sidst overlapper så meget at du får samme bølgefunktion. De kan derfor betragtes som én stor partikel.} Mere præcist køles atomer (ofte alkalimetaller) ned til ~\SI{.1}{\micro\kelvin}, hvilket gøres ved først laserkøling og senere fordampningskøling. Laserkøling fungerer ved at udnytte at fotoner har impuls. Rammes et atom af en foton, der bevæger sig i modsat retning af atomet, exciteres den til et højere energiniveau, men dens fart falder også en smule, for at bevare impulsen. Når atomet henfalder udsendes fotonen igen, og den accelereres en smule i en tilfældig retning. Sker dette mange gange går disse små accelerationer ud med hinanden over tid, hvorved atomet bremses og atomskyen køles. Det virker dog kun, hvis an kan sikre sig at atomer kun interagerer med fotoner, der bevæger sig i modsat retning af dem selv.
\opg Hvad er sammenhængen mellem en fotons energi og bølgelængde?
\opg Hvad sker der hvis et atom rammes af en foton med en energi, der ikke svarer til en (tilladt) atomar overgang? \textit{Hint: Tænk på atomet kvantemekanisk.} \\

Ved laserkølings sendes laserstråler ind fra seks retninger (positiv og negativ retning af hver af de kartesiske akser). Nu simplificeres systemet ved kun at kigge på systemet i én dimension. Betragt nu to ens atomer, der bevæger sig i hver sin retning. Fotonerne kommer fra en laser, der står stille i laboratoriet.
\opg Skiftes referencesystem til hvert atoms referenceramme, ser laseren ud til at bevæge sig. Hvorfor bevæger laseren sig ikke lige hurtigt i begge atomers referencesystem?
\opg Eftersom laseren bevæger sig Dopplerforskydes lyset. Hvad betyder forskellen i bevægelse for Doppllerforskydningen?
\opg Kan begge atomer exciteres af fotoner med samme bølgelængde?
\opg Hvordan kan dette bruges til at bremse atomerne selektiv?
\opg Er opgavetitlen clickbait?
\end{opgave}

\newpage\newpage
\setcounter{opgave}{0}

\section{Facit}
\begin{opgave}{Atmosfærekrav}{3}
For at en planet kan have en stabil atmosfære er den nød til at kunne holde simple molekyler fanget i dens tyngdefelt. Massen af $O_2$ er $m_{O_2} = \SI{5,31e-26}{\kilo\gram}$.
\opg Newtons gravitationslov siger at tyngdekraften mellem legemer i afstand $r$ er
\begin{align*}
	F_G = G\frac{mM}{r^2}
\end{align*}
Bruges Newtons anden lov er
\begin{align*}
	mg &= G\frac{mM}{r^2} \\
	g &= G\frac{mM}{r^2}
\end{align*}
hvor $g$ er tyngdeaccelerationen.
\opg Fra mekanik er de omtalte energier
\begin{align*}
	K &= \frac{1}{2}mv^2 \\
	V &= mgh
\end{align*}
hvor $h$ er afstanden fra nulpunktet. Defineres planetens centrum som nulpunktet er $h = R_p$, og sættes de to energi lig hinanden fås
\begin{align*}
	\frac{1}{2}mv^2 = mgR_p
\end{align*}
Isoleres $v^2$ før tyngdeaccelerationen fra før indsættes, fås
\begin{align*}
	v_\mathrm{ecs}^2 = \frac{2M_pG}{R_p}
\end{align*}
\opg I ligningen fra kinetisk gasteori isoleres $\bar{v}^2/2$
\begin{align*}
	\frac{1}{2}\bar{v}^2 = \frac{3}{2}\frac{k_BT}{m}
\end{align*}
Kombineres dette med undvigelseshastigheden fås
\begin{align*}
	\frac{3}{2}\frac{k_BT}{m} &= \frac{M_pG}{R_p} \\
	\implies \frac{M_p}{R_p} &= \frac{3}{2}\frac{k_BT}{mG}
\end{align*}
\opg Udtrykket på venstre side af lighedstegnet er egenskaber ved selve planten, mens udtrykket på højre side er egenskaber ved plantens atmosfære, der udtrykker partiklerne i atmosfærens hastighed. Hvis planeten skal kunne fastholde atmosfæren må den hastighed ikke overskride undvigelseshastigheden, hvorfor uligheden må være
\begin{align}
	\frac{M_p}{R_p} > \frac{3}{2}\frac{k_BT}{m_{O_2}G}
\end{align}
\opg For Jorden er forholdet mellem masse og radius
\begin{align*}
	\frac{M_\oplus}{R_\oplus} &= \frac{\SI{5,972e24}{\kilo\gram}}{\SI{6371}{\metre}} \\
	&= \SI{9.3737e+17}{\kilo\gram\per\metre}
\end{align*}
Sættes tallene ind i højresiden fås
\begin{align*}
	\frac{3}{2}\frac{k_BT}{m_{O_2}G} &= \frac{3\cdot\SI{1.38065e-23}{\joule\per\kelvin}\cdot(273,15 + 16)\SI{}{\kelvin}}{2\cdot\SI{5,31e-26}{\kilo\gram}\cdot\SI{6.67408e-11}{\metre\cubed\per\kilo\gram\per\second\squared}} \\
	&= \SI{1.6886e+15}{\kilo\gram\per\metre}
\end{align*}
Ergo er ligningen opfyldt for Jorden.
\end{opgave}

\begin{opgave}{Gravitationslinser}{1}
I afsnit \ref{sec:GravLinse} om gravitationslinsemetode blev det nævnt at begivenheden skal observeres af flere forskellige målinger.
\opg Kigges der ud i verdensrummet er der mange ting der kan ses. En lyskurve, som beskrevet i afsnittet, kan også skyldes andre fænomener, skyldes en fejl i målingerne, eller bare et statistisk udsving. For at man kan konkludere noget naturvidenskabeligt, så er man derfor nød til at observere det samme fænomen på en sådan måde, sådan så der kun er en realistisk konklusion.
\opg Gravitationslinsemetoden kræver en meget specifik situation, netop at en stjerne med tilhørende planten bevæger sig ind foran en anden stjerne alt samme i et plan, hvori Jorden også er. For at "lensingen" kan ske er de to stjerne separeret af en anseelig afstand, hvorfor de ikke nødvendigvis påvirker hinanden gravitationelt signifikant. Det er derfor meget muligt at begivenheden kun forekommer én gang, og hvis den ene stjerne er i bane om den anden, så er perioden så kæmpe stor at det er urealistisk at vente lang nok tid til at se det igen.
\opg Hvis man kun har én mulighed for at observere begivenheden, men stadig vil kunne afvise at det er er andet end en planet, så er man for det første nød til at have så mange målinger med forskelligt udstyr, at man kan afvise at der er tale om en fejl i målingerne eller en statistisk afvigelse fra normen. Derudover er man også nød til på en eller anden måde at kunne verificere at det ikke kan skyldes andre fænomener. Den bedste måde at gøre dette på er at observere systemet i lang tid før og efter, med flere forskellige instrumenter, så man ved hvordan området på stjernehimlen normalt ser ud.
\end{opgave}

\begin{opgave}{Bose-Einstein kondensat og laserkøling}{2}
\opg $E = \frac{hc}{\lambda}$ hvor $h$ og $c$ er konstanter, hvorfor energien bliver større, hvis bølgelængden bliver kortere
\opg Ingenting (ideelt set), da atomets energiniveauer er kvantiseret, hvorfor der ikke kan ske noget, hvis ikke fotonens energi er lige præcis nok til at to at flytte elektronen fra ét energiniveau til et andet.
\opg For at holde styr på atomerne kaldes den der bevæger sig mod laseren (1), og den der bevæger sig væk fra (2). Kaldes retningen af atom (1) for den positive retning, gælder følgende: \\
(1) Laseren bevæger sig mod atomet, og altså i negativ  retning. \\
(2) Laseren bevæger sig væk fra atomet, og altså i positiv retning.
\opg Der kigges nu på samme foton fra de to referencesystemer. \\
(1) Fotonen er udsendt fra en kilde, der bevæger sig mod atomet, hvorfor fotonen er blåforskydt. \\
(2) Fotonen er udsendt fra en kilde, der bevæger sig væk fra atomet, hvorfor fotonen er rødforskydt.
\opg De to atomer ser ikke fotonen, som havende samme bølgelængde, og dermed ikke samme energi. Da atomerne er ens, har de samme energiniveauer, men da atomerne oplever fotonen forskelligt, så passer fotonen ikke med en overgang i begge atomer.
\opg Atomet skal gerne holdes samlet i midten af den fælde, hvori de holdes. De atomer, der bevæger sig væk fra fælden mod laseren ser laseren som blåforskudt. Vælges laserens til at have en bølgelængde, der er lidt længere end den, der svarer til overgangen hvis atomet står stille. Det betyder at kun atomer på vej ud af fælden oplever lyset, som værende Dopplerforskudt på den måde, der gør det muligt for atomet at interagere med lyset. Derved kan man sikre sig at man kun påvirker de rigtige atomer, og derved køler gassen.
\opg Dette kan opgavens forfatter ikke besvare. Ifølge JR er svaret entydigt "ja"!
\end{opgave}
\documentclass[a4paper,hidelinks,11pt]{memoir}
\usepackage[utf8]{inputenc} % Do not change or remove!
\usepackage[T1]{fontenc} % Do not change or remove
\usepackage[danish]{babel} % Sproget, vi skriver på
\renewcommand\danishhyphenmins{22} % Kun hvis vi skriver på dansk

%%%%%%%%%%%%%%%%%%%%%%%%%%%%%%%%%%%%%%%%%%%%%%%%%%%%%
% Niels Jakob Søe Loft                              %
% nsl@phys.au.dk                                    %
%%%%%%%%%%%%%%%%%%%%%%%%%%%%%%%%%%%%%%%%%%%%%%%%%%%%%

% Denne skabelon er baseret på Rasmus Villemoes' veldokumenterede
% phd-afhandling i matematik, som jeg har ændret på, så den passer til
% et bachelorprojekt i fysik. Som hovedregel er ting kommenteret på
% engelsk fra Rasmus' skabelon, mens jeg har skrevet på dansk. De
% væsentligste ændringer er, at skabelonen er gjort mere egnet til et
% mindre projekt som et bachelorprojekt er i forhold til en
% phd-afhandling, hvorfor nogle ting er skåret væk, og jeg har
% inkluderet en liste fysik-relaterede makroer. Desuden er
% bibliografien konverteret fra BibTeX til BibLateX pr. marts 2014.

% Pr. 29. marts 2014 har jeg ændret skabelonen, så den kan bruges til
% kompendiet til UNFs Fysik Camp 2014.

%%%%%%%%%%%%%%
%% Generelt %%
%%%%%%%%%%%%%%
% ***************** UNF Science camp  kompendie ***************** %
% Dette dokument indeholder enviroments, comannds, makroer og
% layot specifikt til UNF science camp kompendier

% Pakker der anvendes. Kendte 'issues:
%	- xcolor skal loades før pdfpages, da den ellers loades uden dvipsnames
\usepackage[dvipsnames]{xcolor}		% Farver
\usepackage{xparse}							% Mere flexibel definition af makroer
\usepackage{marginnote}					% Noter i margen
\usepackage{forloop}						% Mulighed for forløkker



% ***************** Opgave enviroment ***************** %
% Sætter en opgave op og angiver sværhedsgraden. Opgavenummereringen nulstilles
% efter hvert ny kapitel.
% Anvenedelse: 
%		\begin{opgave}[farve]{Titel}{Sværhedgrad}
%			Introduktion
%			\opg
%			Delopgave 1
%			\opg
%			Delopgave 2
%			...
%		\end{opgave}
%
% Definer selve enviromentet. i´
\newcounter{opgave}[chapter]
\newcounter{delOpgave}[opgave]
\newenvironment{opgave}[3][NavyBlue]
	{\newcommand{\opg}{{{\refstepcounter{delOpgave}\smallskip\newline\textbf\thedelOpgave})\,}}
	\noindent\ignorespaces\refstepcounter{opgave}\newline\textbf{Opgave \theopgave:\,#3 #2}\newline}
	{\newline\bigskip}
% Definer 
%\newcommand{\lvl}[2][NavyBlue]{
%	\setcounter{nBullets}{#2}
%	\addtocounter{nBullets}{1}
%	\checkoddpage
%	\ifoddpages
%		\normalmarginpar
%		\marginnote{\textcolor{#1}{\lvltoken{\value{nBullets}}}}
%	\else
%		\reversemarginpar
%		\marginnote{\textcolor{#1}{\lvltoken{\value{nBullets}}}}
%	\fi
%}
\NewDocumentCommand{\lvl}{ O{NavyBlue} O{$ \bullet $} m}{
	\setcounter{nBullets}{#3}
	\addtocounter{nBullets}{1}
	\checkoddpage
	\ifoddpage
	\normalmarginpar
	{\textcolor{#1}{\lvltoken[#2]{\value{nBullets}}}}
	\else
	\reversemarginpar
	{\textcolor{#1}{\lvltoken[#2]{\value{nBullets}}}}
	\fi
}
\newcounter{lvl}
\newcounter{nBullets}
\newcommand{\lvltoken}[2][$ \bullet $]{
	\forloop{lvl}{1}{\value{lvl} < #2}{#1}} % load UNF-layout
\usepackage{graphicx} % Billeder
\usepackage{epstopdf} % Så vi kan indsætte eps-filer
\usepackage{lipsum} % Dummytekst
\usepackage{pdfpages} % Indsættelse af pdf-sider
\usepackage{url} % Håndtering af URL'er
\usepackage{xspace} % Smarte mellemrum i egne makroer
\usepackage[final]{fixme} % Indsæt kommentarer i margin
%\usepackage{xstring} % Til sværhedsgrad-makro (se old/macros)
\usepackage[misc]{ifsym} % Til sværhedsgrad, skriv \Cube{n} hvor n=1,2,3
\usepackage{newtxtext}
\usepackage{newtxmath}
\usepackage{subcaption} %sub-figurer
\usepackage{framed} % tekst-bokse
\usepackage{wrapfig}
\usepackage{enumitem}
\usepackage{microtype} % Mellemrumsjustering
\usepackage{xcolor} % flere farver
\usepackage{tikz} % tegninger i latex
\usepackage{empheq}
\usetikzlibrary{decorations.pathmorphing,patterns} % til tikz
\usetikzlibrary{calc}

%% Bibliografi og referencer

%\usepackage{natbib} % Til biblografi, hvis man IKKE bruger BibLaTeX

%\usepackage[style=alphabetic,  % alternativt: style=numeric
%            backend=biber]{biblatex} % BibLaTeX, kræver installering
                                % af biber-pakken
%\addbibresource{kompendie.bib} % BibLaTeX tager referencer fra bach.bib

%\usepackage{cleveref} % Smarte referencer: skriv \cref{...}
\usepackage[colorlinks=true, hidelinks]{hyperref} % Farvede links

%%%%%%%%%%%%%%%%%%%%%%
%% Tekst og formler %%
%%%%%%%%%%%%%%%%%%%%%%

%\usepackage[osf]{mathpazo} % Skrift

\usepackage{wasysym} % Font til smileys \smiley og \frownie

%\usepackage[sf]{libertine} % Til slanted skrift NJ's emacs er pigesur
\usepackage{libertine}

\linespread{1.06} % Større linjeafstand pga. font
\usepackage{fourier-orns} % Sjove symboler NJ's emacs er pigesur igen
\usepackage{textcomp} % Tilføjer flere tegn
\renewcommand\ttdefault{txtt} % Pænere teletype-skrift

\usepackage{mathtools} % Matematiktricks
\usepackage{cancel} % Ting der går ud med hinanden
\usepackage{siunitx} %SI-enheder
\sisetup{separate-uncertainty=true % gør at siunitx skriver +/- i
  % stedet for at bruge parentes til
  % at angive usikkerheder.
  ,output-decimal-marker={,}, % gør at der bruges komma til komma og
  % ikke punktum som i USA.
  ,load=abbr, % så vi kan bruge \keV
  ,exponent-product = \cdot, output-product = \cdot, % skift gangetegn fra \times til \cdot
}
%% VI LAVER NOGLE FYSIK- OG MATEMATIK-MAKROER:


%% Generelt
%\newcommand{\g}{\cdot} % Prikprodukt, gangetegn
\newcommand{\subv}[2]{\gv{#1}_{\text{#2}}} % Pæn subscript til vektorer
\newcommand{\sub}[2]{#1_{\text{#2}}} % Pæn subscript til
\newcommand{\e}{\mathcal{E}} % Skrevet E
\newcommand{\abs}[1]{\left| #1 \right|} % Numerisk værdi
\newcommand{\N}{\ensuremath{\mathbb{N}}} % Naturlige tal
\newcommand{\Z}{\ensuremath{\mathbb{Z}}} % Hele tal
\newcommand{\Q}{\ensuremath{\mathbb{Q}}} % Rationelle tal
\newcommand{\R}{\ensuremath{\mathbb{R}}} % Reelle tal
\newcommand{\C}{\ensuremath{\mathbb{C}}} % Komplekse tal
\newcommand{\F}{\ensuremath{\mathbb{F}}} % Legeme tal
\newcommand{\A}{\ensuremath{\mathbb{A}}} % Algebraiske tal
\newcommand{\re}{\text{Re}}
\newcommand{\im}{\text{Im}}

\renewcommand{\phi}{\varphi}
\renewcommand{\epsilon}{\varepsilon}

%% Angiv sværhedsgrad til opgaver (benytter \usepackage{xstring})
%\newcommand{\lvl}[1]{%
%\IfStrEqCase{#1}{{1}{\ensuremath{\star}}
%    {2}{\ensuremath{\star\star}}
%    {3}{\ensuremath{\star\star\star}}}
%    [nada]
%}

%% Infinitesimalregning

\let\underdot=\d % omdøb indbygget kommando \d{} til \underdot{}
%\renewcommand{\d}[2]{\partial_{#2} \, #1} % afledt
%\newcommand{\dd}[2]{\partial_{#2}^2 \, #1} % dbl.afledt

%differentierings d
\renewcommand{\d}{\mathrm{d}}

%haard differentiering
\newcommand{\dif}[3][]{\frac{\d^{#1}{#3}}{{\d {#2}}^{#1}}}

%partiel differentiering
\newcommand{\pdif}[3][]{\frac{\partial^{#1}{#3}}{\partial {#2}^{#1}}}

\newcommand{\dt}[1]{\dot{#1}} % afledt mht. t (dot-notation)
\newcommand{\ddt}[1]{\ddot{#1}} % dbl.afledt mht. t (dbl.dot)

\newcommand{\integral}[4]{\int\limits_{#3}^{#4} \! #1 \, \textrm{d}#2} % integrere
\newcommand{\ubint}[2]{\integral{#1}{#2}{}{}}
\renewcommand{\iint}{\int\!\!\!\!\int}
\newcommand{\ubiint}[3]{\ubint{\!\!\!\ubint{#1}{#2}}{#3}}
% til Euler-Lagrange ligningen
\newcommand{\el}[1]{\dif{t}{}\left(\pdif{#1}{L}\right)}


% Vektorer

\newcommand{\xyz}[3]{\begin{bmatrix} #1 \\ #2 \\ #3 \end{bmatrix}} %3D-vektor
\newcommand{\xy}[2]{\begin{bmatrix} #1 \\ #2 \end{bmatrix}} %2D-vektor
\let\vaccent=\v % Omdøb \v{} til \vaccent{}

\newcommand{\gv}[1]{{\vec{\boldsymbol{#1}}}} % Vektor med græske bogstaver
\renewcommand{\v}[1]{\gv{#1}} % Vektor med fed
\newcommand{\hatvec}[1]{\hat{\mathbf{#1}}} % Hatvektor
\newcommand{\ihat}{\boldsymbol{\hat{\textbf{\i}}}} % Enhedsvektor i
\newcommand{\jhat}{\boldsymbol{\hat{\textbf{\j}}}} % .. j
\newcommand{\khat}{\mathbf{\hat{k}}}  % .. k
\newcommand{\xhat}{\mathbf{\hat{x}}} % Enhedsvektor x
\newcommand{\yhat}{\mathbf{\hat{y}}} % .. y
\newcommand{\zhat}{\mathbf{\hat{z}}} % .. z
\newcommand{\rhat}{\mathbf{\hat{r}}} 
\newcommand{\thhat}{\mathbf{\hat{\boldsymbol{\theta}}}} 
\newcommand{\phhat}{\mathbf{\hat{\boldsymbol{\phi}}}} 
\newcommand{\grad}[1]{\gv{\nabla} #1} % Gradient
\let\divsymb=\div % Omdøb \div til \divsymb
\renewcommand{\div}[1]{\gv{\nabla} \cdot \v{#1}} % Divergens
\newcommand{\curl}[1]{\gv{\nabla} \times \v{#1}} % Curl
% Vil man tage div eller curl af græske bogstaver,
% skal man lade argumentetet være fx \gv{\mu} for µ-vektor

% Kvantemekanik

\newcommand{\op}[1]{\hat #1} % operator

\newcommand{\expect}[1]{\left< #1 \right>} % Forventningsværdi
\newcommand{\trace}{\ensuremath{\text{Tr}}\xspace}
\newcommand{\Hilbert}{\ensuremath{\mathcal{H}}}
\newcommand{\lag}{\ensuremath{{L}}}
\newcommand{\tr}[1]{\text{Tr}\left(#1\right)} % Trace
\newcommand{\ptr}[2]{\text{Tr}_{#1}\left(#2\right)} % Partial trace
\newcommand{\ket}[1]{\left| #1 \right>} % Dirac-notation: ket
\newcommand{\bra}[1]{\left< #1 \right|} % bra
\newcommand{\braket}[2]{\left< #1 \vphantom{#2} \, \right|
  \left. \! #2 \vphantom{#1} \right>} % bracket
\newcommand{\matrixel}[3]{\left< #1 \vphantom{#2#3} \right|
  #2 \left| #3 \vphantom{#1#2} \right>} % Bracket med ekstra streg
 % En masse matematik- og fysikmakroer

%%%%%%%%%%%%
%% Layout %%
%%%%%%%%%%%%

\newcommand{\anonbreak}{\fancybreak{$* * *$}} % Break med stjerner
\let\bar\overline % Gør at en bar over et symbol kan skalere efter symbolet

%% Sidehoved- og fod

\makepagestyle{tket}
\makeevenfoot{tket}{\thepage}{}{}
\makeoddfoot{tket}{}{}{\thepage}
\makeevenfoot{plain}{\thepage}{}{}
\makeoddfoot{plain}{}{}{\thepage}
\pagestyle{tket}

%% Margin

% Man kan sætte margins ved enten at specificere marginstørrelsen
% eller ved at specificere tekstblokken. Man skal vælge én og kun én
% af mulighederne.

% Specificer marginstørrelsen
%\setulmarginsandblock{2.7cm}{*}{1}
%\setlrmarginsandblock{1.6cm}{1.6cm}{*} 
%\setlength{\oddsidemargin}{-1cm} % Giver mere plads på siden
%\setlength{\topmargin}{-1.2cm} % Gør topmargin behagelig at se på
%\setlength{\columnsep}{1.5\columnsep}  % Afstand mellem søjlerne


\setlrmarginsandblock{2.5cm}{2.5cm}{*}

\usepackage[font={small,it}]{caption}	% Italic captions

% Tekstblok: Følgende er fra Rasmus Villemoes' thesis-layout.tex
%\setlxvchars[\normalfont] % standardbredden af tekstblok er ca. 65 tegn
%\settypeblocksize{*}{1.2\lxvchars}{1.61803} % højde, bredde, forhold
%\setulmargins{*}{*}{1.3} % lav bundmargin lidt større end topmargin
\checkandfixthelayout % memoir tjekker, at alt er ok og konsistent

%%%%%%%%%%%%%%%%%%
%% Definitioner %%
%%%%%%%%%%%%%%%%%%

% Definer titlen på projektet
 \newcommand{\thesistitle}{Kompendie til UNF Fysik Camp 2018}

%%%%%%%%%%%%%%%%%%%%%%
%% Slut på preamble %%
%%%%%%%%%%%%%%%%%%%%%%


%%%%%%%%%%%%%%%%%%%%%% 
%%  BEGIN DOCUMENT  %%
%%%%%%%%%%%%%%%%%%%%%%


\begin{document}

\mainmatter

%her sættes emner ind.
\chapter{Kvantemekanik}
\section*{Uendeligt dyb brønd}
\begin{opgave}{Parabelformet bølgefunktion}{1}\label{kvant:opg:parabel}
Vi ser her på en partikel i en uendelig brønd i intervallet fra $0$ til $L$. Lad bølgefunktionen være:
$$
\psi = Nx(L-x) 
$$
\opg Find $N$ så bølgefunktionen er normeret.
\opg Hvad er forventningsværdien for positionen $\expect x$?
\opg Find forventningsværdien for energien $\expect E$ og sammenlign den fundne energi med grundtilstandsenergien: $\frac{\pi^2\hbar^2}{2mL^2}$.
\opg Er $\psi$ en stationær tilstand?
\end{opgave}
%
\begin{opgave}{Sammensatte bølgefunktioner}{1}
Find normeringskonstanten $N$ og energien $E$ for de følgende bølgefunktioner, der er sammensat af stationære tilstande for den uendelige brønd.
\opg $N(\psi_1+\psi_2)$.
\opg $N(\psi_1-\psi_3)$.
\opg $N(\psi_1+\psi_2-2\psi_3)$.\\
Hint: Udnyt at bølgefunktionerne er ortonormale.
\end{opgave}
%
\begin{opgave}{Den tidsafhængige bølgefunktion}{2}
I en uendelig brønd er bølgefunktionen til tiden $t=0$:
$$
\Psi(x,0) = \frac{1}{\sqrt{5}}(2\psi_1+\psi_2) 
$$
\opg Hvad er $\Psi(x,t)$? Du kan med fordel bruge 
$$
\omega = \frac{E_1}{\hbar} = \frac{\pi^2\hbar}{2mL^2} \, .
$$
\opg Hvad er $\Psi^*(x,t)$?
\opg Skriv $\Psi^*x\Psi$ så simpelt som muligt.
\opg Hvad er $\expect{x(t)}$? \\
Hint:
\begin{align*}
e^{i\theta}+e^{-i\theta} &= 2\cos \theta \, , \\ \matrixel{\psi_n}{x}{\psi_n} &= \frac{L}{2} \, , \\  \matrixel{\psi_1}{x}{\psi_2} &= \frac{-16L}{9\pi^2} \, .
\end{align*}
\end{opgave}
%
\begin{opgave}{En partikel i et kvadrat}{3}
I to dimensioner er den tidsuafhængige Schrödingerligning i kartesiske koordinater:
$$
E\psi(x,y) = \frac{-\hbar^2}{2m}\left(\pdif[2]{x}\psi +\pdif[2]{y}\psi\right) + V(x,y)
$$
Vi vil se på en kvadratisk brønd, i to dimensioner, med sidelængder på $L$. Her er potentialet nul, når $0\leq x\leq L$ og $0\leq y\leq L$.
Antag nu at man kan skrive bølgefunktionen som:
 $$\psi(x,y) = X(x)Y(y) = XY$$
\opg Indsæt $\psi = XY$ i Schrödingerligningen med $V=0$ og isoler $E$.
\opg Energien vil bestå af en bidrag fra $X$ og $Y$, så $E=E_x+E_y$. Opstil differentialligninger i stil med ligning \eqref{k-kvant:eq:infb} i kompendiet for $X$ og $Y$.
\opg Find generelle løsninger til differentialligningerne, bølgefunktionen og de tilhørende energier. Bemærk at der vil være et kvantetal for hver af differentialligningerne. 
\opg Hvad er de fem laveste energier? Skitser bølgefunktionerne med disse energier.
\end{opgave}
%
\begin{opgave}{En partikel i en boks}{3}
I tre dimensioner er den tidsuafhængige Schrödingerligning i kartesiske koordinater:
$$
E\psi(x,y,z) = \frac{-\hbar^2}{2m}\left(\pdif[2]{x}\psi +\pdif[2]{y}\psi+\pdif[2]{z}\psi\right) + V(x,y,z)
$$
Vi vil se på en kubisk boks med en sidelængde på $L$, hvor potentialet er nul, når $x$, $y$ og $z$ alle er imellem 0 og $L$.
Antag at man kan skrive bølgefunktionen som:
$$
\psi(x,y,z) = X(x)Y(y)Z(z) = XYZ
$$
\opg Indsæt $\psi = XYZ$ i Schrödingerligningen med $V=0$ og isoler $E$.
\opg Energien vil bestå af et bidrag fra $X$, $Y$ og $Z$, så $E=E_x+E_y+E_z$. Opstil differentialligninger i stil med ligning \eqref{k-kvant:eq:infb} i kompendiet for $X$, $Y$ og $Z$.
\opg Find generelle løsninger til differentialligningerne, bølgefunktionen og de tilhørende energier. Bemærk at der vil være et kvantetal for hver af differentialligningerne. 
\opg Find de laveste 5 mulige energier udtrykt i $E_1 = \frac{\pi^2\hbar^2}{2mL^2}$ energien for en  endimensionel uendelig brønd med samme brede som boksens sidelængde.
\end{opgave}
\begin{opgave}{Parabelformet bølgefunktion igen}{1}
Denne opgave bygger videre på opgave \ref{kvant:opg:parabel}, så det er en fordel at have lavet denne opgave først. Vi ser igen på en parabelformet bølgefunktion i en uendelig brønd:
$$
\psi=Nx(L-x)
$$
\opg Hvad er $\sigma_x^2 = \expect{x^2}-\expect{x}^2$?

Vi kender allerede $\expect{x}$ og $\abs N^2$ fra opgave \ref{kvant:opg:parabel}.
\begin{align*}
\expect{x} &= \frac{L}{2}\\
\abs N^2 &= \frac{30}{L^5}
\end{align*}
For at finde $\expect{x^2}$ bruges sandwichen:
\begin{align*}
    \expect{x^2}&= \abs N^2 \integral{x^3(L-x)}{x}{0}{L}\\
    &= \frac{30}{L^5}\integral{x^5-2x^4L+x^3L^2}{x}{0}{L}\\
    &= \frac{30}{L^5}\left[\frac{x^6}{6}-\frac{2x^5L}{5}+\frac{x^6}{6}\right]_0^L\\
    &= 30 L^2\left(\frac{15}{60}-\frac{24}{60}+\frac{10}{60}\right)\\
    &= \frac{L^2}{2}
\end{align*}
Nu er usikkerheden:
$$
\sigma_x^2 = \expect{x^2}-\expect{x}^2 = \frac{L^2}{2}-\frac{L^2}{4}=\frac{L^2}{2}
$$
og
$$
\sigma_x = \frac{L}{\sqrt{2}}
$$
\opg Hvad er $\sigma_p^2 = \expect{p^2}-\expect{p}^2$?
Her skal vi bruge at $\op p = -i\hbar\pdif{x}{}$. Så bliver sandwichen:
\begin{align*}
    \expect{p} &= -i\hbar\abs N^2\integral{x(L-x)\pdif{x}{}\left(x(L-x)\right)}{x}{0}{L}\\
    &= \frac{-30i\hbar }{L^5}\integral{x(L-x)(L-2x)}{x}{0}{L}\\
    &= \frac{-30i\hbar }{L^5}\integral{xL^2-3x^2L+2x^3}{x}{0}{L}\\
    &= \frac{-30i\hbar }{L^5}\left[\frac{x^2L^2}{2}-x^3L+\frac{x^4}{2}\right]_0^L\\
    &= \frac{-30i\hbar }{L} \left(\frac{1}{2}-1+\frac{1}{2}\right)\\
    &=0
\end{align*}
Tilsvarende for $\expect{p^2}$, hvor $op p^2 = -\hbar^2\pdif[2]{x}{}$.
\begin{align*}
    \expect{p^2} &= \abs N^2 \integral{x(L-x)\pdif[2]{x}{}\left(x(L-x)\right)}{x}{0}{L}\\
    &= \frac{-30\hbar^2 }{L^5}\integral{x(L-x)(-2)}{x}{0}{L}\\
    &= \frac{60\hbar^2}{L^5}\integral{xL-x^2}{x}{0}{L}\\
    &= \frac{60\hbar^2}{L^5}\left[\frac{x^2L}{2}-\frac{x^3}{3}\right]_0^L\\
    &= \frac{60\hbar^2}{L^5}\left(\frac{L^3}{2}-\frac{L^3}{3}\right)\\
    &= \frac{60\hbar^2}{L^2}\left(\frac{3}{6}-\frac{2}{6}\right)\\
    &= \frac{10\hbar^2}{L^2}
\end{align*}
Så usikkerheden er:
$$
\sigma_p^2 = \expect{p^2} = \frac{10\hbar^2}{L^2}
$$
og:
$$
\sigma_p = \frac{\hbar\sqrt{10}}{L}
$$
\opg Passer det med Heisenbergs usikkerhedsprincip?

Sættes $\sigma_x$ og $\sigma_p$ ind i Heisenbers usikkerhedsprincip findes:
$$
\sigma_x\sigma_p = \frac{L}{\sqrt{2}}\frac{\hbar\sqrt{10}}{L}= \hbar\sqrt{5}\geq\frac{\hbar}{2}
$$
Heisenberg er tilfreds $\ddot \smile$
\end{opgave}

\begin{opgave}{Den frie partikel}{2}
En fri partikel er en partikel der ikke påvirkes af noget potentiale, så $V(x)=0$ for alle $x$
\opg Hvad er $\op H$?

Der er ikke noget potentiale så:
$$
\op H = \frac{-\hbar^2}{2m}\pdif[2]{x}{} = \frac{\op p^2}{2m}
$$
\opg Hvad er $[\op p,\op H]$?

Bemærk at operatorer altid kommuterer med sig selv, så:
\begin{align*}
    [\op p,\op H]&= \op p\op H-\op H\op p\\
    &= \frac{\op p^3}{2m}-\frac{\op p^3}{2m}\\
    &= 0
\end{align*}
\opg Hvilke bølgefunktioner opfylder: $\op p \psi_p = -i\hbar\pdif{x}{\psi_p} = p\psi_p$?

Omskrives dette findes differentialligningen:
$$
\pdif{x}{\psi_p}=\frac{ip}{\hbar}
$$
Løsningen til denne differentialligning er blot en eksponentialfunktion. Imaginærfaktoren ændrer ikke dette, så:
$$
\psi_p = e^{\frac{ipx}{\hbar}}
$$
\opg Hvad sker der hvis man sætter $\psi_p$ ind i Schrödingerligninen?
\begin{align*}
    \op H\psi_p &= \frac{-\hbar^2}{2m}\pdif[2]{x}{\psi_p}\\
    &= \frac{-\hbar^2}{2m}\left(\frac{ip}{\hbar}\right)^2\psi_p\\
    &= \frac{-\hbar^2}{2m}\frac{-p^2}{\hbar^2}\psi_p\\
    &=\frac{p^2}{2m}\psi_p
\end{align*}
\opg Hvad er sammenhængen imellem $E$og $p$?

Energien er:
$$
E=\frac{p^2}{2m}
$$
\opg Hvad er $\sigma_p$ og $\sigma_E$?

Først findes $\expect{p}$, $\expect{p^2}$, $\expect{E}$ og $\expect{E^2}$.
\begin{align*}
    \expect{p} &= \braket{\psi_p}{\op p\psi_p}\\
    &= p\braket{\psi_p}{\psi_p}\\
    &=p
\end{align*}
Dette er en konsekvens af at det er en egentilstand. Udregningen er tilsvarende for de andre.
\begin{align*}
    \expect{p^2} &= p^2\\
    \expect{E} &= E\\
    \expect{E^2} &= E^2
\end{align*}
Det gør det muligt at finde usikkerhederne:
\begin{align*}
    \sigma_p &= \expect{p^2}-\expect p^2 = 0\\
    \sigma_E &= \expect{E^2}-\expect E^2 = 0
\end{align*}
Man kan godt kende $p$og $E$ på samme tid for den frie partikel, og det er ved $\psi_p$ tilstandene.
\end{opgave}
\section*{Harmoniske oscillatorer}
\begin{opgave}{Harmonisk oscillator med $\boldsymbol{\op a_- \op a_+}$}{1}
\label{kvant:opg:amap}
Færdiggør udledningen af den harmoniske oscillator. Hvis du havde problemer med denne udledning er denne opgave stærkt anbefalet.
\opg Udregn $\op a_-\op a_+$.
\opg Brug dette til at udlede ligning \eqref{k-kvant:eq:Hamap} i kompendiet.
\opg Vis at hvis $\psi_n$ er en løsning til Schrödingerligningen, så er $\op a_-\psi_n$ det også.
\opg Hvad er energien af $\op a_- \psi_n$
\end{opgave}
%
\begin{opgave}{Sjov med operatorer}{2}
\label{kvant:opg:sjov}
Vi vil her komme ind på en af grundene til, at hæve-/sænkeoperatorerne er smarte. Udnyt at bølgefunktionerne er ortonormale, og husk at
$$
\op a_\pm = \frac{1}{\sqrt{2 m \hbar\omega}}(\mp i\op p+m\omega \op x) \, .
$$
\opg Udtryk $\op x$ og $\op p$ ved $\op a_+$ og $\op a_-$.
\opg Find $\expect x$ og $\expect p$ for $\psi_0$, $\psi_1$ og $\psi_{n}$.
\opg Gør det samme for $\expect{x^2}$ og $\expect{p^2}$.
\opg Hvad er $\sigma_x$ og $\sigma_p$ for $\psi_n$?
\opg Hvordan passer det med Heisenbergs usikkerhedsprincip?
\end{opgave}
%
\begin{opgave}{Nu med tid}{2}
Til tiden $t=0$ har vi bølgefunktionen: $$\Psi(x,t=0) = N(\psi_0+\psi_1)$$ (Det kan være en fordel at have lavet opgave \ref{kvant:opg:sjov} først.)
\opg Hvad er $N$?
\opg Hvad er $\Psi(x,t)$?
\opg Hvad er $\expect E$?
\opg Hvad er $\expect {x(t)}$?
\end{opgave}
%
\begin{opgave}{Molekylære vibrationer}{2}
En god model for bindingen i et molekyle er Morsepotentialet:
$$
V(r) = D\left(1-e^{-(r-R)}\right)^2
$$
\opg Hvor er potentialets minimum (ligevægtsafstanden)?
\opg Hvad er minimumsværdien af potentialet?
\opg Hvad er potentialet for meget store $r$.
\opg Hvad er $\pdif[2]{r}{V}$ i ligevægtspunktet. Dette er kraftkonstanten $k$ analogt med for en fjeder.
\opg Molekylet vil kunne vibrere omkring ligevægtspunktet. Hvad er $\omega$?
\opg Morse potentialet kan tilnærmes som en harmonisk oscillator. Hvad er grundtilstandsenergien for denne?\\ \\
For et brintmolekyle er Morsepotentialet givet ved:
\begin{align*}
D&=\SI{7.24e-19}{J} \, ,\\
a &= \SI{3.93e10}{m^{-1}} \, ,\\
R &= \SI{7.40e-11}{m} \, ,\\
m_p&=\SI{1.67e-27}{kg} \, .
\end{align*}
\opg Hvad er $\omega$ og $E_0$ for brintmolekylet.
\end{opgave}
\section*{Energibevarelse}
\begin{opgave}{Energibevarelse i kvantetilstande}{3}
Vi skal i denne opgave se på hvordan energibevarelse kommer til udtryk i kvantemekaniske tilstande. Til enhver Hamilton $\hat{H}$ operator kan vi finde et sæt af løsninger som opfylder følgende ligning: $\hat{H}\psi_n=E_n\psi_n$.
\opg Udtryk en generel funktion $f(x)$ som en kombination af løsninger til Schrödinger ligningen.

Alle funktion kan skrives som en som af egentilstande, da disse løsninger danner en komplet basis:
$$
f(x) = \sum_nc_n\psi_n
$$
Man kunne finde koefficienterne $c_n$ med ligning (2.30) men deres præcise form er underordnet 
\opg Hvordan vil denne funktion udvikle sig over tid? (Hint: udtryk $f(x,t)$ som en kombination af løsninger til Schrödinger ligningen)

Tidsudviklingen af $f(x)$ kan findes ved at multiplicere hvert led i summen med den tilsvarende faktor $\exp\left(\frac{-iE_nt}{\hbar}\right)$. Det giver:
$$
f(x,t) = \sum_n c_n\psi_ne^{-iE_nt/\hbar}
$$
\opg Udregn forventningsværdien af energien $\expect E$. Hvordan vil denne udvikle sig over tid?

Forventingsværdien er givet:
\begin{align*}
\expect E &= \matrixel{f(x,t)}{\op H}{f(x,t)}
\end{align*}
Hamiltonoperatoren kan flyttes ind i summen til højre, og anvendes på de individuelle tilstande.
\begin{align*}
&= \matrixel{\sum_n c_n\psi_ne^{-iE_nt/\hbar}}{\op H}{\sum_m c_m\psi_me^{-iE_mt/\hbar}}\\
&= \braket{\sum_n c_n\psi_ne^{-iE_nt/\hbar}}{\sum_m c_m\op H\psi_me^{-iE_mt/\hbar}}\\
&= \braket{\sum_n c_n\psi_ne^{-iE_nt/\hbar}}{\sum_m c_mE_m\psi_me^{-iE_mt/\hbar}}
\end{align*}
Istedet for at have sumtegnene inde i braketten, kan de flyttes udenfor, så det bliver en dobbelt sum.
\begin{align*}
&=\sum_n\sum_m\braket{c_n\psi_ne^{-iE_nt/\hbar}}{c_mE_m\psi_me^{-iE_mt/\hbar}}\\
&=\sum_n\sum_mc_n^*c_mE_m\braket{\psi_ne^{-iE_nt/\hbar}}{\psi_me^{-iE_mt/\hbar}}\\
&=\sum_n\sum_mc_n^*c_mE_me^{i(E_n-E_m)t/\hbar}\braket{\psi_n}{\psi_m}
\end{align*}
Når man tager en dobbelt sum vil man lægge alle kombinationer af $n$ og $m$ sammen. Siden løsningerne til schrödingerligningen er ortonormale vil braketten enten være et når $n=m$ og nul ellers. Det betyder at kun ledene hvor $n$ og $m$ er ens skal tælles med og vi kan reducere den dobbelte sum til en enkelt sum.
\begin{align*}
&= \sum_nc_n^*c_nE_nE^{i(E_n-E_n)t/\hbar}\braket{\psi_n}{\psi_n}\\
&= \sum_n \abs{c_n}^2E_n
\end{align*}
Her fosvandt tidsafhængigheden, så energiens forventningsværdi er konstant.
\end{opgave}
\newpage
\section*{Symmetri}
\begin{opgave}{Lige og ulige funktioner}{1}
En funktion så som $f(x) = x^2$ der opfylder kravet, $f(x) = f(-x)$, kaldes en lige funktion. En funktion som $f(x) = x$ der opfylder det lignende krav, $f(x)=-f(-x)$, kaldes for en ulige funktion.
Det vil sige, at en lige funktion er uændret, hvis man spejler den i $y$-aksen, mens en ulige funktion skifter fortegn ved den samme spejling.
Bemærk at de fleste funktioner er hverken lige eller ulige, og unikt er funktionen $f(x) = 0$ både lige og ulige.
Afgør om følgende funktioner er lige eller ulige.
\opg $\sin x$.
\opg $e^{x^2}$.
\opg $\cos x$.
\end{opgave}
%
\begin{opgave}{Mere om lige og ulige funktioner}{1}
Lad $f_g(x)$ være en lige funktion og $f_u(x)$ være en ulige funktion.\footnote{$g$ og $u$ står for gerate og ungerate, de tyske ord for lige og ulige.}
\opg Vis at produktet af lige og ulige funktioner fungerer på samme måde som produktet af lige og ulige tal, i forhold til hvorvidt produktet er lige eller ulige.
\opg Er $1/f_{g}(x)$ lige eller ulige?
\opg Hvad med $1/f_{u}(x)$?
\opg Hvad er reglen for division af lige og ulige funktioner?
\end{opgave}
\begin{opgave}{Sammensætning af lige og ulige funktioner}{2}
Alle funktioner kan skrives som en unik sum af en lige og en ulige funktion:
$$
f(x) = f_g(x)+f_u(x)
$$
\opg Skriv $f(-x)$ ud fra $f_g(x)$ og $f_u(x)$.
\opg Skriv $f_g(x)$ og $f_u(x)$ ud fra $f(x)$ og $f(-x)$.
\opg Hvad er den lige og den ulige del af eksponentialfunktionen $e^x$?
\end{opgave}
%
\begin{opgave}{Integraler af lige og ulige funktioner.}{3}
Vi vil her finde nogle meget praktiske regneregler for integraler af lige og ulige funktioner over et symmetrisk interval.
Lad $f_g(x)$ være en lige funktion og $f_u(x)$ være en ulige funktion.
Antag derudover at integralerne
$$
\integral{f_g(x)}{x}{0}{a}   \quad \text{og} \quad \integral{f_u(x)}{x}{0}{a} 
$$
er kendte.
\opg Vis at
$$
\integral{f_g(x)}{x}{-a}{a} = 2\integral{f_g(x)}{x}{0}{a} \, .
$$
\opg Vis at
$$\integral{f_u(x)}{x}{-a}{a} = 0 \, .
$$
\opg Brug dette til at løse integralet: $$\integral{x\cos(x)\sin(x)+x^2-x\exp(x^2)}{x}{-1}{1}$$
\end{opgave}

\end{document}


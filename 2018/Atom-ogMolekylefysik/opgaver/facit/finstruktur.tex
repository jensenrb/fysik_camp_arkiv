\documentclass[../../../Atom-ogMolekylefysik.tex]{subfiles}
\begin{document}

\begin{opgave}{Finstruktur for en elektron}{1}
Vi vil i denne opgave se på tilstanden n=2 for hydrogenatomet.
\opg For $s=\frac{1}{2}$, l=1, udregn $\beta_l$
$$\beta_1=\frac{g_s1^3e^8m_e}{c^22^3\hbar^44^5\pi^4\epsilon_0^4}\frac{1}{1(1+0.5)(1+1)}$$
\opg Find alle værdier af J for n=2 tilstanden.
$$J=|l+s|,|l+s-1|,..,|l-s|$$
hvilket giver $J=\frac{3}{2},\frac{1}{2}$
\opg Find energi opsplitningen for tilstandende n=2,l=0.
$$\left<\hat{s}\cdot\hat{l}\right>=\frac{1}{2}(j(j+1)-l(l+1)-s(s+1))$$
hvis l=0, er j=s, hvilket giver:
$$\left<\hat{s}\cdot\hat{l}\right>=\frac{1}{2}(j(j+1)-s(s+1))=0$$
dvs. for l=0 $H_{s-o}=0$
\opg Find energi opsplitningen for tilstandende n=2,l=1.
$$\left<\hat{s}\cdot\hat{l}\right>=\frac{1}{2}(j(j+1)-l(l+1)-s(s+1))$$
for $j=\frac{3}{2}$
$$\left<\hat{s}\cdot\hat{l}\right>=\frac{1}{2}(\frac{3}{2}(\frac{3}{2}+1)-1(1+1)-\frac{1}{2}(\frac{1}{2}+1))=\frac{1}{2}(\frac{15}{4}-\frac{8}{4}-\frac{3}{4})=\frac{1}{2}$$
for $j=\frac{1}{2}$
$$\left<\hat{s}\cdot\hat{l}\right>=\frac{1}{2}(\frac{1}{2}(\frac{1}{2}+1)-1(1+1)-\frac{1}{2}(\frac{1}{2}+1))=\frac{1}{2}(-1(1+1))=-1$$
dvs. at energiforskellen er givet ved:
$\Delta E=E_{j=\frac{3}{2}}-E_{j=\frac{1}{2}}=\frac{\beta_1}{2}-(-\beta_1)=\frac{3}{2}\beta_1$
\end{opgave}

\begin{opgave}{Finstruktur for flere elektroner}{3}
Vi vil i denne opgave se på et system med 2 elektronerhvor de hver har n=2.
\opg Find alle L og S værdier for $l_1=1$ og $l_2=1$,hvorfor er det ikke nødvendigt at specificere hvad $s_1$ og $s_2$ er?\\
\\
Der er ikke nødvendigt at specificere hvad $s_1$ og $s_2$ er, da elektroner er fermioner og de har pr. definitions altid spin $\frac{1}{2}$
$$L=|l_1+l_2|,|l_1+l_2-1|,..,|l_1-l_2|$$
hvilket giver $L=2,0$
$$S=|s_1+s_2|,|s_1+s_2-1|,..,|s_1-s_2|$$
hvilket giver $S=1,0$
\opg Find alle J værdier for L og S fra forrige opgave.
$$J=|L+S|,|L+S-1|,..,|L-S|$$
Her skal man kombinere begge værdier af L med begge værdier af S, hvilket giver:\\
$J=3,2,1,0$
\opg Find alle energiopsplitningerne for de J, L og S tilstande vi fandt i de forrige opgaver.\\
Her skal vi først tænke os om. Den eneste måde vi kan få J værdierne J=2,0 er hvis enten L eller S er 0. Dette betyder at J=S, J=L for de værdier hvilket gør at $\left<\hat{L}\cdot\hat{S}\right>=0$, Vi kan derfor se bort fra disse værdier af J.
J=1 kan vi få ved enten $|L-S|$ for L=2, S=1, eller ved L=0, S=1. hvis L=0, S=1 er J=S=1, får vi også at $\left<\hat{L}\cdot\hat{S}\right>=0$, så det er kun når L=2 og S=1 at vi får værdier der ikke er 0, og det er derfor kun de tilstande vi kan brug til noget. Vi skal så benytte formlen:
$$H_{S-O}=\Bigg[\beta_1\frac{(L(L+1)+l_1(l_1+1)-l_2(l_2+1))(S(S+1)+s_1(s_1+1)-s_2(s_2+1))}{4L(L+1)S(S+1)}$$
$$+\beta_2\frac{(L(L+1)-l_1(l_1+1)+l_2(l_2+1))(S(S+1)-s_1(s_1+1)+s_2(s_2+1))}{4L(L+1)S(S+1)}\Bigg]\hat{S}\cdot\hat{L}$$
Vi indser ret hurtigt at da $l_1=l_2$ og $s_1=s_2$ får vi reduceret det til:
$$H_{S-O}=\Bigg[\beta_1\frac{(L(L+1))(S(S+1))}{4L(L+1)S(S+1)}+\beta_2\frac{(L(L+1))(S(S+1))}{4L(L+1)S(S+1)}\Bigg]\hat{S}\cdot\hat{L}$$
$$=\left[\frac{\beta_1}{4}+\frac{\beta_2}{4}\right]\hat{S}\cdot\hat{L}$$
da $l_1=l_2$ bliver $\beta_1=\beta_2$ og vi får:
$$H_{S-O}=\frac{\beta_1}{2}\hat{S}\cdot\hat{L}$$
hvor $\hat{S}\cdot\hat{L}$ bliver:\\
for J=3
$$\hat{S}\cdot\hat{L}=\frac{1}{2}(3(3+1)-2(2+1)-1(1+1))=\frac{1}{2}(12-6-2)=2$$
så $E_{J=3}=\frac{\beta_1}{2}\hat{S}\cdot\hat{L}=\beta_1$\\
for J=1
$$\hat{S}\cdot\hat{L}=\frac{1}{2}(1(1+1)-2(2+1)-1(1+1))=\frac{1}{2}(2-6-2)=-3$$
så $E_{J=1}=\frac{\beta_1}{2}\hat{S}\cdot\hat{L}=\frac{-3}{2}\beta_1$\\
Vi har derfor til slut at $\Delta E=E_{J=3}-E_{J=1}=\beta_1-\frac{-3}{2}\beta_1=\frac{5}{2}\beta_1$
\end{opgave}

\end{document}
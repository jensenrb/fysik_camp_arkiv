\begin{opgave}{Dihelium}{1}
Vi skal nu undersøge om dihelium(He$_2$) kan eksistere.
\opg Hvad er elektronstrukturen for frit helium?

I frit helium er begge elektronerne i grundtilstanden.
\opg Opstil symmetritilpassede molekylorbitaler svarende til dem for H$_2$.

De symmetritilpassede molekyleorbitaler for He$_2$ er de samme som for H$_2$
\opg Lav et molekyleorbitaldiagram for dihelium.
\opg Hvad er elektronstrukturen for dihelium.
\opg Er der nogen energigevinst for helium ved at være i molekylet?
\end{opgave}
\begin{opgave}{Ilt}{2}
En af de mest almindelige molekyler i atmosfæren er ilt (O$_2$).
\opg Hvad er punktgruppen for ilt?
\opg Hvilke atomorbitaler indgår i bindingen?
\opg Find karakterer for atomorbitalerne.
\opg Find Symmetritilpassede atomorbitaler. Bemærk at de godt kan være udartede.
\opg Opskriv molekyleorbitaldiagrammet.
\opg Find elektronstrukturen.
\opg Er der uparede elektroner. Hvis der er det vil molekylet være magnetisk.
\end{opgave}
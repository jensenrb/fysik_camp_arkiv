\section*{Spin}
\begin{opgave}{Pauli princippet uden spin}{1}
Der er en fejl i denne opgave, første del skulle have været $\op O\Psi(x_1,x_2)$
\opg 
$$
\op O\Psi(x_1,x_2) = \op O(\psi(x_1)\psi(x_2))\psi(x_2)\psi(x_1) = \Psi(x_1,x_2)
$$
\opg Siden $\op O\Psi(x_1,x_2)=\Psi(x_1,x_2)
$
er bølgefunktionen symmetrisk, og dermed ikke antisymetrisk. Den er forbudt ifølge Pauli. Bølgefunktionen representerer to partikler i den samme tilstand, hvilket ikke kan lade sig gøre hvis spin ikke var en ting.
\end{opgave}
%
\begin{opgave}{Rumbølgefinktioner}{2}
\opg
$$
\op O \Psi_{12}=\op O(\psi_1(x_1)\psi_2(x_2))=\psi_1(x_2)\psi_2(x_1)=\Psi_{21}
$$
\opg
$$
\op O \Psi_{21}=\op O(\psi_2(x_1)\psi_1(x_2))=\psi_2(x_2)\psi_1(x_1)=\Psi_{12}
$$
\opg
$$
\op O \Psi^S=\op O\Psi_{12}+\op O\Psi_{21}=\Psi_{21}+\Psi_{12}=\Psi^S
$$
\opg
$$
\op O \Psi^A=\op O\Psi_{12}-\op O\Psi_{21}=\Psi_{21}-\Psi_{12}=-\Psi^A
$$
\opg $\Psi_{12}$ og $\Psi_{21}$ er ikke symmetriske under ombytning, $\Psi^S$ er symetrisk og $\Psi^A$ er  antisymmetrisk. Det er da også symmetrisk og antisymmetrisk som $S$ og $A$ står for.
\end{opgave}
%
\begin{opgave}{Pauli med spin}{3}
\opg
$$
\op O\alpha(1)\alpha(2) = \alpha(2)\alpha(1)
$$
\opg
$$
\op O \beta(1)\beta(2) = \beta(2)\beta(1)
$$
\opg
$$
\op O \alpha(1)\beta(2) = \alpha(2)\beta(1)
$$
\opg
$$
\op O \beta(1)\alpha(2) = \beta(2)\alpha(1)
$$
\opg Vi har allerede fundet to symmetriske spinbølgefunktioner, den sidste bølgefunktioner kan konstrueres på samme måde som $\Psi^S$.
\begin{align*}
\chi_{\alpha\alpha}&=\alpha(1)\alpha(2)\\
\chi_{\beta\beta}&=\beta(1)\beta(2)\\
\chi_{\alpha\beta}^S&=\alpha(1)\beta(2)+\beta(1)\alpha(2)
\end{align*} 
\opg Tilsvarende kan en antisymetrisk spin bølgefunktion konstrueres på samme måde som $\Psi^A$.
$$
\chi_{\alpha\beta}^A=\alpha(1)\beta-\beta(1)\alpha(2)
$$
\opg Når elektronerne er i samme rum tilstand kan de kun have en symmetrisk bølgefunktion, hvis den totale bølgefunktion skal være antisymmetrisk må spin bølgefunktionen være antisymmetrisk.
Da der kun er en mulighed:
\opg
$$
\Psi=\Psi_{11}\chi_{\alpha\beta}^A
$$
Dette kaldes en singlet tilstand.
\opg For elektroner i forskellige tilstande er der to muligheder for den rummelige bølgefunktion: $\Psi^S$ og $\Psi^A$. For at opfylde Pauli må $\Psi^S$ kombineres med $\chi_{\alpha\beta}^A$ og $\Psi^A$ kombineres med en af de 3 andre.
Det giver 4 ialt.
\begin{align*}
\Psi^S\chi_{\alpha\beta}^A\\
\Psi^A\chi_{\alpha\alpha}\\
\Psi^A\chi_{\beta\beta}\\
\Psi^A\chi_{\alpha\beta}^S
\end{align*}
\end{opgave}
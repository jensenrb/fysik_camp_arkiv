\section*{Hyperfinstruktur}
\begin{opgave}{Hyperfinstruktur for 21 cm linjen}{1}
Vi vil i denne opgave arbejde med hyperfinstruktur for en elektron i grundtilstanden af et brintatom.
\opg Kernen er her en proton med spin $\frac{1}{2}$. Brug dette til at finde alle mulige værdier af $F$. Hvorfor er det ikke nødvendigt at kende $l$ for elektronen?
\opg Udregn konstanten $A$ for grundtilstanden.
\opg Find energiopsplitningen mellem tilstandende.
\opg Udregn bølgelængden af det lys som har en energi svarende til energiforskellen mellem de to tilstande.
\end{opgave}
\begin{opgave}{Hyperfinstruktur for atomure}{2}
Vi vil i denne opgave arbejde med et cæsium atom, som det er beskrevet i teksten.
Spinnet for en cæsium kerne er $I=\frac{7}{2}$, og vi vil i denne opgave arbejde med $n=6$ tilstanden.
\opg Find alle $F$ tilstande for $l=0$.
\opg Udregn energiopsplitningen for tilstandende.
\opg Udregn frekvensen af det lys der bliver udsendt fra denne overgang.
\end{opgave}

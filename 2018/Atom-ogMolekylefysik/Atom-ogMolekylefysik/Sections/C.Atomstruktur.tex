\documentclass[../../Atom-ogMolekylefysik.tex]{subfiles}
\begin{document}
\section{Atomstruktur}
Det simpleste atom er brintatomet. Her er det rentfaktisk muligt at udregne analytiske udtryk for alle elektronbølgefunktionerne. Denne udledning er do lettere omfattende, så dem der har mod på det kan finde den i appendixet. Det vigtigste er at alle elektrontilstandene er beskrevet ved 3 kvantetal: $n$, $l$ og $m$. Det kan skrives: $\psi_{nlm}$
De er alle hele tal. $n$ kan gå fra 1 og opad, $l$ går fra 0 til $n$ og $m$ imellem $-l$ og $l$.
Grundtilstanden i brintatomet har en energi på:
\begin{equation}
E_1 = \SI{13.6}{eV}    
\end{equation}
De andre tilstande har en energi givet ved $n$ kvantetallet:
\begin{equation}
    E_n = \frac{E_1}{n^2}~~~~n=1,2,3,...
\end{equation}
Så længe der kun er en elektron er det fint, men langt de fleste atomer har mere end en elektron\footnote{Siden universet primært består af brint har de fleste atomer en elektron, men stort set alle andre atomer har mere end en elektron.}.
Når vi tilføjer flere elektroner vil de frastøde hinanden, hvilket giver et mere kompliceret system.
Den første effekt er at de indre elektroner vil frastøde de ydre elektroner, som derfor er svagere bundet. Derudover vil energien ikke længere kun være bestemt af $n$ kvantetallet, men også $l$ kvantetallet. Energien er dog stadig primært bestemt af $n$ kvantetallet, så f.eks har $\psi_{210}$ højere energi end $\psi_{200}$ men lavere energi end $\psi_{300}$.

Det er normal praksis at navngive orbitalerne med et bogstav der representerer $l$ kvantetallet. $s$ for $l=0$, $p$ for $l=1$, $d$ for $l=2$ og $f$ for $l=3$. Bogstaverne stammer fra spektroskopi, hvor det var muligt at observere energiniveauerne et godt stykke tid før kvantemekanikken blive udviklet. $s$ for sharp, $p$ for principal, $d$ for difuse og $f$ for fine. Der er flere, men herefter forsætter de bare alfabetisk. Siden et højt $l$ kvantetal kræver et højt $n$ kvantetal vil højere end $f$ orbitaler kun findes som eksiterede tilstande.

En tommelfingerregel for at finde rækkefølgen af orbitaler vil være så lavt $n$-kvantetal som muligt, derefter så lavt $l$-kvantetal som muligt. Det er dog ikke helt korekt. Den egentlige rækkefølge er:
$$
1s~~2s~~2p~~3s~~3p~~4s~~3d~~4p~~5s~~4d~~5p~~6s~~...
$$
For de høje energier begynder der at være meget kort imellem de enkelte niveauer.
\end{document}
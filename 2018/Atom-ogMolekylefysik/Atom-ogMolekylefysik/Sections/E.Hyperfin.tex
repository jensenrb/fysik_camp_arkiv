\documentclass[../../Atom-ogMolekylefysik.tex]{subfiles}
\begin{document}

\section{Hyperfinstruktur}
I det forgående afsnit har vi beskrevet hvordan elektronernes energier i hydrogenatomet bliver splittet op grundet koblingen mellem elektronens spin og dens angulære moment. Dette så vi blev ufattelig kompliceret hvis vi begyndte at regne på systemer af flere elektroner, så derfor kunne man fristes til at tro at der nu ikke var flere korrektioner til energierne for atomer, for hvordan kan det blive mere kompliceret end det allerede er? Som du sikkert har gættet er legen ikke slut her, for det er nemlig sådan at kernen i et atom jo består af protoner og neutroner og disse partikler har også et spin. Der opstår derfor en yderligere opsplitning af energierne, som er forårsaget af interaktionen mellem kernens spin og det totale angulære moment som elektronen har. Det totale angulære moment som elektronen producerer vil nemlig resultere i et magnetfelt (når $J\neq0$) og dette vil naturligvis interagere med magnetfeltet som kernens spin forårsager. 
For at udregne dette udtryk skal man se på det magnet felt som det totale angulære moment forårsager og så udregne hvor stort det magnetfelt er i midten af atomet der hvor kernen er (det er jo kun den del af magnetfeltet som er ved kernen, som kernen kan interagere med). Denne udledning vil vi ikke lave her, men i stedet bare state at den er givet ved:
\begin{equation}
    H_{hfs}=A\left<\hat{I}\cdot\hat{J}\right>
\end{equation}
hvilket med det samme også gælder hvis der er flere elektroner, da det jo bare kommer an på værdien af det totale angulære moment $J$.
A er så en kosntant der er givet ved faktorerne: $A=\frac{2}{3}\mu_0g_s\mu_Bg_I\mu_N\frac{Z^3}{\pi a_0^3n^3}$. Her er $\mu_0$ vacuum permabiliteten, $g_s$ er det samme som ved finstrukturen og $\mu_B$ er ligesom før Bohr magnetonen. $g_I$ er i dette regi lidt speciel, da man skulle forvente at protoner og neutroner ligesom elektroner har værdier der er ret "pæne" (tæt på heltallige værdier), sådan forholder det sig dog ikke, og det har eksperimentielt vist sig at disse partikler har $g$ værdierne $g_n=- 3.82608545$, $g_p=5.58569470$. Hvilket var et af de første tegn på at protoner og neutroner faktisk består af andre mere fundamentale partikler, som vi i dag har fundet ud af er quarkerne. Inde i en kerne har vi generelt en masse protoner og neutroner, så det er svært at sige noget om $g_I$ da vil være en kobling mellem $g$ værdierne for de partikler der er i kernen, dog vil størrelsesordenen for denne konstant nok ikke overstige 10, så man kan bare sætte den til 2 for at undgå forvirring. $\mu_N$ svarer til $\mu_B$ for elektronen bare hvor det i stedet er for en kerne, og den er defineret som:
\begin{equation}
    \mu_N=\frac{e\hbar}{2m_N}=\frac{e\hbar}{2m_e\cdot1836}=\frac{1}{1836}\frac{e\hbar}{2m_e}
\end{equation}
så vi ser at $\mu_N$ er ca. 2000 gange mindre en $\mu_B$, hvilket skyldes at massen af en proton er ca. 2000 gange større end en massen for en elektron.\\
\\
For at udregne opsplitningen af energierne givet koblingen mellem det totale angulære moment og kernens spin skal vi dog først finde ud af hvad udtrykket $\left<\hat{I}\cdot\hat{J}\right>$ giver. Set i lyset af hvad vi lige har lavet, så er det måske ikke så overraskende at løsningen findes ved at betragte det totale angulære moment af hele atomet (kerne + elektron) og definere dette som F, som er givet ved $F=J+I$, og denne ændrer sig nødvendigvis ikke hvis der ikke er nogen udefrakommende kræfter. Vi bruger nu samme trick som ved finstrukturen og får:
\begin{equation}
    \left<\hat{I}\cdot\hat{J}\right>=\frac{1}{2}(F(F+1)-I(I+1)-J(J+1))
\end{equation}
Den totale opsplitning af energierne givet ved den hyperfine struktur er derfor givet ved:
\begin{equation}
    H_{hfs}=\frac{2}{6}\mu_0g_s\mu_Bg_I\mu_N\frac{Z^3}{\pi a_0^3n^3}(F(F+1)-I(I+1)-J(J+1)).
\end{equation}
Men vent nu lige lidt, hvis denne opsplitning egentlig er mere generel end finstruktur opsplitningen, da vi jo her også tager højde for at elektronen interagerer med kernens spin, hvorfor brugte vi så rigtig lang tid på at udlede og regne på finstrukturen? Svaret til dette er faktisk ret simpelt. Det er nemlig sådan at den opsplitning af energierne som hyperfinstrukturen forårsager er i størrelsesordenen $10^{-5}$ hvilket er ca. 1000 gange mindre end den opsplitning af energierne som finstrukturen forårsager. Det man gør er derfor at man udregner korrektionerne fra finstrukturen og derefter udregner man korrektionerne til disse korrektioner som er givet ved hyperfinstrukturen, og når man har det hele samlet så har man alle de forskellige opsplitninger af energiniveauerne der er (eller ikke helt alligevel, der findes andre korrektioner til energierne, men disse vil ikke blive gennemgået her. Andet litteratur om atomfysik kan dog konsulteres hvis man er interesseret.).\\
\\
En ting man kan spørge sig selv om her er, hvorfor er det interessant at udregne alle disse ting? Svaret til dette er faktisk ret simpelt. For finstruktur tilstandene så vi at korrektionen til energierne for grundtilstanden simpelthen er 0 idet der jo ikke er noget angulært moment. For hyperfinstruktur ser vi dog at dette ikke er tilfældet. F er jo defineret som $F=I+J$, så hvis elektronen er i sin grundtilstand har den $l=0$ og $s=\frac{1}{2}$, hvilket vil sige at $J=s=\frac{1}{2}$, så $F=I+s$ hvilket giver følgende to værdier for F, $F=|I+\frac{1}{2}|,|I-\frac{1}{2}|$. Hvis vi nu betragter et hydrogenatom så vil I have værdien $I=\frac{1}{2}$, da protonen jo har spin $\frac{1}{2}$, og F vil derfor antage værdierne $F=1,0$. og vi har derfor at:
\begin{align*}
    E_{F=1}&=\frac{A}{2}(1(1+1)-\frac{1}{2}(\frac{1}{2}+1)-\frac{1}{2}(\frac{1}{2}+1))=\frac{A}{4}\\
    E_{F=0}&=\frac{A}{2}(0(0+1)-\frac{1}{2}(\frac{1}{2}+1)-\frac{1}{2}(\frac{1}{2}+1))=\frac{-3A}{4}
\end{align*}
denne opsplitning er ret specifik for hydrogenatomet og energiforskellen er så lille, så det lys der bliver udsendt fra denne overgang har derfor en helt specifik bølgelængde da der ikke er andet der udsender lys med så lav energi. Dette kan astronomer bruge til at kigge efter hydrogen ude i verdensrummet, for ligeså snart man ser lys med denne bølgelængde, så ved man med det samme at det stammer fra hydrogen. Det lys der bliver udsendt herfra er har nemlig en bølgelængde på ca. 21 centimeter og derfor kalder man det 21 centimerter linjen når man laver spektroskopi på lyset herfra. Grunden til at denne bølgelængde er ekstra god a kigge efter er at lys med en bølgelængde på 21 centimeter ligger i radiobølge regimet og dette lys intergerer ikke specielt meget med andet stof. dvs. at det går igennem det meste andet materiale ude i universet. derfor der det en god indikator hvis man vil se hvor meget stof der f.eks. ligger i den anden ende af vores galakse, da vi jo ikke kan se det med normalt synligt lys da alle stjernerne ligger i vejen, men da 21 centimer linjen trænger igennem relativt ubesværet kan vi bruge denne til at få en anelæse om hvordan stof generelt er fordelt bag stjerner og andre ting der blokere "normalt" lys.\\
\\
En anden virkelg sej ting ved den hyperfine opsplitning er faktisk at det også er denne der ligger til grund for hvordan atomure fungerer.  Her benytter man nemlig cæsium-133, hvor man ser på den hyperfine opsplitning i 6n tilstanden. Cæsium har nemlig 55 elektroner, hvoraf de første 54 elektroner fylder alle de nederste tilstande fra n=1 til n=5 og så der er kun 1 elektron i den yderste 6 skal. Dette gør cæsium atomet særligt stabilt da de andre korrektioner til energierne for elektronerne (som blev nævnt eksisteret men som ikke er blevet udspecificeret her), ikke har så stor effekt her. Da cæsium-133 kernen har spin $\frac{7}{2}$, så får atomet det totale angulære moment $F=3,4$, og det er overgangen mellem disse to tilstande man bruger som definitions på sekundet. Overgangen mellem de to tilstande har nemlig en energiforskel på $6.06\cdot10^{-6}ev$, som svarer til lys med en bølgelængde på ca. 3.26 cm. Man har så defineret sekundet ud fra den frekvens som lyset der bliver udsendt ved denne overgang har, hvilket er defineret som eksakt 9192631770 Hz, hvor enheden Hz jo er $\frac{1}{s}$, og det er derfor at man kan definere sekundet ud fra denne naturkonstant.





\end{document}
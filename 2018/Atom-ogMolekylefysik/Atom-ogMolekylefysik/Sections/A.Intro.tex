\documentclass[../../Atom-ogMolekylefysik.tex]{subfiles}
\begin{document}
Ordet atom kommer fra det græske ord $\alpha \tau o \mu o \varsigma$ (atomos) der betyder udelelig. Det skyldes den græske filosof Demokirt, der allerede omkring 400 f.v.t havde fremsat en atomteori. Udover navnet har Demokrits atomteori ikke meget at gøre med vores moderne forståelse af atomer. Hans atomer var som universitets udelelige byggesten mere lig elementarpartikler end det vi idag kalder for atomer. 
Det var dog komplet umuligt at undersøge, så Demokrits ideer spillede ikke den store rolle i den tidlege videnskab.

Først i begyndelsen af 1800-tallet vandt ideen om atomer indpas igen. John Dalton havde observeret at kemiske reaktioner altid brugte den samme andel af de forskellige reaktanter, og konkluderende at kemikalier måtte være opbygget af et helt antal atomer med forskellig masse. Daltons atomer var lige som Demokrits udellelelige, hvilket er grunden til at navnet blev genoplivet. 

Med opdagelsen af elektronen i 1897 begyndte jegten på en beskrivelse af atomernes struktur. Den første model fremsat af Thomson (der også havde opdaget elektronen) hvor elektronerne flød frit rundt i et positivt ladet medie. Denne model kaldes ofte for rosinbolle modellen. 
I 1909 udførte Geiger og Marsden et eksperiment hvor de sendte $\alpha$-partikler ind imod en tynd guldfolie, og observerede deres spredning. I Thomsons model er ladningen spredt over et relativt stort område, og $\alpha$-partiklerne brude passere igennem uden de store afbøjninger. Lagt de fleste passerede da også nogenlunde direkte igennem folien, men nogle få $\alpha$ partikler blev stærkt afbøjet. Det fik deres vejleder Ernest Rutherford til at konkludere at den positive ladning måtte være koncentreret i et meget lille område: atomkærnen. I Rutherfords model kredsede elektronerne omkring kærnen som planeter omring Solen. 

Når et ladet partikel accelereres udsendes der stråling, kaldet bremsestråling. Problemet med Rutherfords model er at den ikke er stabil. Elektronerne burde udsende bremsestråling og spiralere ind imod kærnen iløbet af meget kort tid. 

Bohrs atommodel fra 1913 var en forbedring, men på mange måder er modellen en mystisk mellemting imellem kvantemekanik og klassisk fysik. I Bohr modellen er elektronerne begrenset til bestemte energiniveauer, og kan springe imellem dem under absorption eller udsendelse af en foton. Man havde på det tidspunkt allerede fundet ud af at forskellige grundstoffer gav forskellige spektre, men Bohr fandt sammenhængen imellem spektroskopien og atomfysiken.
Bohrmodellen er framragende for brintatomer\footnote{Eller ioner med kun en elektron så som He$^+$ og Li$^{2+}$}, men så snart der er mere end en elektron bryder modellen sammen. Med schrödingerligningen blev en egentlig kvantemekanisk beskrivelse af atomer mulig, og med få modifikationer er det modellen vi bruger idag.

\end{document}
\documentclass[../../Atom-ogMolekylefysik.tex]{subfiles}
\begin{document}

\begin{opgave}{Hyperfinstruktur for 21 cm linjen}{1}
Vi vil i denne opgave arbejde med den hyperfine struktur for en elektron i grundtilstanden af et brintatom.
\opg Kernen er en proton med spin $\frac{1}{2}$, find alle værdier af F. Hvorfor er det ikke nødvendigt at kende $l$aq for elektronen?
\opg Udregn konstanten A for grundtilstanden.
\opg Find energiopsplitningen mellem de to tilstande
\opg udregn bølgelængden af det lys som har en energi svarende til energiforskellen mellem de to tilstande.
\end{opgave}

\begin{opgave}{Hyperfinstruktur for atomure}{2}
Vi vil i denne opgave arbejde med et cæsium atom, som det er beskrevet i teksten.
Spinnet for et cæsium kerne er I=$\frac{7}{2}$ og vi vil i denne opgave arbejde med n=6 tilstanden.
\opg Find alle F tilstande for l=0.
\opg Udregn energiopsplitningen for tilstandende.
\opg Udregn frekvensen af det lys der bliver udsendt fra denne overgang.
\end{opgave}

\end{document}
\documentclass[../../Atom-ogMolekylefysik.tex]{subfiles}
\begin{document}
\begin{opgave}{Pauli princippet uden spin}{1}
Lad $\op O$ være ombytningsoperatoren. Som navnet antyder er det den operator der for to elektroner til at bytte plads.
$$
\op O \Psi(x_1,x_2) = \Psi(x_2,x_1)
$$

Lad $\psi$ være en en elektron tilstand, og $\Psi(x_1,x_2) = \psi(x_1)\psi(x_2)$
\opg Hvad er $\op \Psi(x_1,x_2)$?

For at en tilstand skal opfylde Pauli princippet skal bølgefunktionen være antisymmetrisk under ombytning. Det vil sige $\Psi(x_1,x_2) = -\Psi(x_2,x_1)$.
\opg Er $\Psi(x_1,x_2)$ tilladt ifølge Pauli?
\end{opgave}

\begin{opgave}{Rumbølgefinktioner}{2}
Lad os istedet se på to elektroner i to forskellige tilstande: $\psi_1$ og $\psi_2$.
Lad $\Psi_{12}(x_1,x_2) = \psi_1(x_1)\psi_2(x_2)$ og $\Psi_{21}(x_1,x_2) = \psi_2(x_1)\psi_1(x_2)$
Udregn de følgende ombytninger.
\opg $\op O \Psi_{12}(x_1,x_2)$
\opg $\op O \Psi_{21}(x_1,x_2)$
\opg $\op O \Psi^S(x_1,x_2)=\op O (\Psi_{12}(x_1,x_2)+\Psi_{21}(x_1,x_2))$
\opg $\op O \Psi^A(x_1,x_2)=\op O (\Psi_{12}(x_1,x_2)-\Psi_{21}(x_1,x_2))$
\opg Er bølgefunktionerne symmetriske, antisymmetriske eller ingen af delene.
\end{opgave}

\begin{opgave}{Pauli med spin}{3}
Nu vil vi tage højde for spin. Det gøres ved at multiplicere bølgefunktionen med en spinfunktion. Elektronen kan have spin op eller spin ned. Det angives med $\alpha$ eller $\beta$. For at holde styr på hvilken partikel der har hvilket spin tilføjes et tal til spinfunktionen. Så har partikel et spin op og partikel to spin ned skrives det: $\chi=\alpha(1)\beta(2)$.
Ombytningsoperatoren ombytter også hvilken partikel der har hvilket spin.
Hvad er:
\opg $\op O \alpha(1)\alpha(2)$
\opg $\op O \beta(1)\beta(2)$
\opg $\op O \alpha(1)\beta(2)$
\opg $\op O \beta(1)\alpha(2)$
\opg Brug dette til at finde tre spinfunktioner der er symmetriske under ombytning.
\opg Find en spinfunktion der er antisymmetrisk under ombytning.

Tag Højde for både spin og rum delen af den totale bølgefunktion.
\opg Hvor mange muligheder er der der er tilladt af Pauli, når de to elektroner har samme tilstand?
\opg Hvad er denne tilstand?
\opg Hvor mange Paulitilladte bølgefunktioner er der når de to elektronbølgefunktioner er forskellige.
\end{opgave}
\end{document}
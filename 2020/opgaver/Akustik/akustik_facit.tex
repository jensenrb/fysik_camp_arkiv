\documentclass[crop=false, class=memoir]{standalone}
\usepackage[utf8]{inputenc}%Nødvendig for danske bogstaver
\usepackage[danish]{babel}%Sørger for at ting LaTeX gør automatisk er på dansk
\usepackage{csquotes}
\usepackage{geometry}%Til opsætning af siden
\geometry{lmargin = 2.5cm,rmargin = 2.5cm}%sætter begge magner
\usepackage{lipsum}%Fyldtekst, til brug under test af layoutet
\usepackage{float}
\usepackage{graphicx}%Tillader grafik
\usepackage{epstopdf}%Tillader eps filer
\usepackage{marginnote}% Noter i margen
\interfootnotelinepenalty=10000 %undgår at fodnoter bliver spilittet op.
\usepackage[sorting=none]{biblatex}
\addbibresource{litteratur.bib}
\usepackage[hidelinks]{hyperref}%Tillader links
\usepackage{subcaption} % Tillader underfigurer
\usepackage[font={small,sl}]{caption}	% Caption med skrå tekst ikke kursiv

\usepackage{xcolor} %Bruges til farver
\usepackage{forloop} %Bruges til nemmere for loops

\newcounter{opgave}[chapter] %Definerer opgavenumrene og hvornår de nulstilles
\renewcommand{\theopgave}{\thechapter.\arabic{opgave}} %Definerer udseende af opgavenummereringen
\newcounter{delopgave}[opgave] %Definerer delopgavenumrene
\newcounter{lvl} %Definerer en "variabel" til senere brug

\definecolor{markerColor}{rgb}{0.0745098039, 0.262745098, 0.584313725} %Definerer farven af markøren
\newcommand{\markerSymbol}{\ensuremath{\bullet}} %Definerer tegnet for markøren
\newlength{\markerLength} %Definerer en ny længde
\settowidth{\markerLength}{\markerSymbol} %Sætter den nye længde til bredden af markøren

\newenvironment{opgave}[2][0]{%Definerer det nye enviroment, hvor sværhedsgraden er den første parameter med en default på 0
\newcommand{\opg}{\refstepcounter{delopgave}\par\vspace{0.1cm}\noindent\textbf{\thedelopgave)\space}}%Definerer kommando til delopgave
\refstepcounter{opgave}%Forøger opgavenummer med 1 og gør den mulig at referere til
\setcounter{lvl}{#1}%Sætter "variablen" lvl lig med angivelsen af sværhedsgraden
\noindent\hspace*{-0.75em}\hspace*{-\value{lvl}\markerLength}\forloop{lvl}{0}{\value{lvl}<#1}{{\color{markerColor}\markerSymbol}}\hspace*{0.75em}%Sætter et antal af markører svarende til sværhedsgraden
\textbf{Opgave \theopgave : #2}\newline\nopagebreak\ignorespaces}{\bigskip} %Angiver udseende af titlen på opgaverne samt mellemrummet mellem opgaver



\usepackage{mathtools}%Værktøjer til at skrive ligninger
\renewcommand{\phi}{\varphi}%Vi bruger varphi
\renewcommand{\epsilon}{\varepsilon}%Vi bruger varepsilon
\usepackage{physics}%En samling matematikmakroer til brug i fysiske ligninger
\usepackage{braket}%Simplere kommandoer til bra-ket-notation
\usepackage{siunitx}%Pakke der håndterer SI enheder godt
\DeclareSIUnit\clight{\text{\ensuremath{c}}} % Lysets fart i vakuum som c og ikke c_0
\usepackage{chemmacros}
\usechemmodule{isotopes}
\usepackage{tikz}
\usepackage[danish]{cleveref}
\usepackage{nicefrac}
% \renewcommand{\ref}[1]{\cref{#1}}
\creflabelformat{equation}{#2(#1)#3}
\crefrangelabelformat{equation}{#3(#1)#4 to #5(#2)#6}
\crefname{equation}{ligning}{ligningerne}
\Crefname{equation}{Ligning}{Ligningerne}
\crefname{section}{afsnit}{afsnitene}
\Crefname{section}{Afsnit}{Afsnitene}
\crefname{figure}{figur}{figurene}
\Crefname{figure}{Figur}{Figurene}
\crefname{table}{tabel}{tabellerne}
\Crefname{table}{Tabel}{Tabellerne}
\crefname{opgave}{opgave}{opgaverne}
\Crefname{opgave}{Opgave}{Opgaverne}
\crefname{delopgave}{delopgave}{delopgaverne}
\Crefname{delopgave}{Delopgave}{Delopgaverne}

\newcommand{\eqbox}[1]{\begin{empheq}[box=\fbox]{align}
	\begin{split}
	#1
	\end{split}
\end{empheq}}

\newcommand{\kb}{\ensuremath{k_\textsc{b}}}

\DeclareSIUnit{\parsec}{pc}
\DeclareSIUnit{\lightyear}{ly}
\DeclareSIUnit{\astronomicalunit}{AU}
\DeclareSIUnit{\year}{yr}
\DeclareSIUnit{\solarmass}{M_\odot}
\DeclareSIUnit{\solarradius}{R_\odot}
\DeclareSIUnit{\solarluminosity}{L_\odot}
\DeclareSIUnit{\solartemperature}{T_\odot}
\DeclareSIUnit{\earthmass}{M_\oplus}
\DeclareSIUnit{\earthradius}{R_\oplus}
\DeclareSIUnit{\jupitermass}{M_J}

% Infobokse og lignende
% http://mirrors.dotsrc.org/ctan/graphics/awesomebox/awesomebox.pdf
% \usepackage{awesomebox}


% Egen infobokse (virker kun med begrænsede symboler)

\usepackage[framemethod=tikz]{mdframed}
\usetikzlibrary{calc}
\usepackage{kantlipsum}

\usepackage[tikz]{bclogo}

\tikzset{
    % lampsymbol/.style={scale=2,overlay}
    % lampsymbol/.pic={\centering\tikz[scale=5]\node[scale=10,rotate=30]{\bclampe}}.style={scale=2,overlay}
    infosymbol/.style={scale=2,overlay}
}

\newmdenv[
    hidealllines=true,
    nobreak,
    middlelinewidth=.8pt,
    backgroundcolor=blue!10,
    frametitlefont=\bfseries,
    leftmargin=.3cm, rightmargin=.3cm, innerleftmargin=2cm,
    roundcorner=5pt,
    % skipabove=\topsep,skipbelow=\topsep,
    singleextra={\path let \p1=(P), \p2=(O) in ($(\x2,0)+0.92*(1.1,\y1)$) node[infosymbol] {\bcinfo};},
    % singleextra={\path let \p1=(P), \p2=(O) in ($(\x2,0)+0.5*(2,\y1)$) node[infosymbol] {\bcinfo};},
]{info}

% Skal bruges som
% \begin{info}[frametitle={Titel}]
%     Tekst
% \end{info}
\usepackage{import}
\begin{document}
\section{Akustik}
%%
\begin{opgave}[1]{Lydens hastighed}
\opg Snorspændingen eller snorkraftens størrelse $F_s$
\opg $v_\mathrm{lyd}=\sqrt{\nicefrac{F_s}{\mu}}$ altså bliver lydhastigheden større, når snoren strammes.
\opg $v_\mathrm{lyd}=\sqrt{\nicefrac{F_s}{\mu}}$ altså bliver lydhastigheden mindre, hvis snoren gøres tungere.
\end{opgave}
%%
%%
\begin{opgave}[1]{Grundtonen og stemning af strenginstrumenter}
Grundtonens frekvens er givet i \cref{aku:eq:frekvens}.
\opg Grundtonens frekvens stiger.
\opg Grundtonens frekvens falder.
\opg Grundtonens frekvens falder.
\opg Dette kan besvares ved at beregne forholdet mellem de to grundtonefrekvenser
%
\begin{align*}
    \frac{\tilde \nu_0}{\nu_0} = \frac{\nicefrac{\pi}{2L}\cdot\sqrt{\nicefrac{2F_s}{\mu}}}{\nicefrac{\pi}{L}\cdot\sqrt{\nicefrac{F_s}{\mu}}} = \frac{1}{\sqrt{2}}.
\end{align*}
%
Grundtonenfrekvensen bliver altså mindre med en factor $\nicefrac{1}{\sqrt{2}}$, hvilket svarer til en ligesvævende kvint.
\opg Er instrumentet for højt, skal $\nu_0$ gøres mindre, hvorfor strengen skal slækkes.
\opg Er instrumentet for lavt, skal $\nu_0$ gøres større, hvorfor strengen skal strammes.
\end{opgave}
%%
%%
\begin{opgave}[1]{Temperering}
At strengene stemmes i kvinter, betyder at
%
\begin{align*}
    \nu_{d0} = \frac{3}{2}\nu_{g0}, \;\; \nu_{a0} = \frac{3}{2}\nu_{d0} \;\; \mathrm{og} \; \nu_{e0} = \frac{3}{2}\nu_{a0}.
\end{align*}
Derfor er
\opg $\dfrac{\nu_{d0}}{\nu_{g0}} = \dfrac{3}{2}$
\opg $\dfrac{\nu_{a0}}{\nu_{g0}} = \dfrac{\nu_{a0}}{\nu_{d0}}\cdot\dfrac{\nu_{d0}}{\nu_{g0}} = \dfrac{3}{2}\cdot\dfrac{3}{2} = \dfrac{9}{4}$
\opg Man vil høre to toner, med næsten samme grundtonefrekvens, hvilket giver stødtoner mellem hver partialtone i de to toners overtonerække.
\opg Stemmer violinisten i frekvensforholdet $2^{\nicefrac{7}{12}}$ ville
%
\begin{align*}
    \dfrac{\nu_{a0}}{\nu_{g0}} = \dfrac{\nu_{a0}}{\nu_{d0}}\cdot\dfrac{\nu_{d0}}{\nu_{g0}} = 2^{\nicefrac{7}{12}}\cdot2^{\nicefrac{7}{12}} = 2^{\nicefrac{14}{12}},
\end{align*}
%
hvilket er det samme som klaveret.
\opg $x_\mathrm{ulv} = \dfrac{2^7}{\left(\nicefrac{3}{2}\right)^{11}} = \dfrac{2^{18}}{3^{11}} = \num{1.47981}$ 
\opg $\mathrm{E}_\mathrm{ren} = \SI{660,000}{\hertz}$ og $\mathrm{E}_\mathrm{ulv} = \SI{651,117}{\hertz}$. Der er altså næsten \SI{9}{hertz} til forskel mellem de to kvinter, hvilket let kan høres, hvorfor ulvekvinten er tydeligt falsk for selv utrænede ører.
\end{opgave}
%%
%%
\begin{opgave}[2]{Vibrato}
\opg Ved at trykke strengen ned på fingerbrættet, kan længden af strengen ændres. Ved at bevæge fingeren frem og tilbage ændres længden af strengen tilsvarende. Af \cref{aku:eq:frekvens} får dette grundtonefrekvensen til at svinge i tid.
\opg Når hånden holdes over den ringende tast i en afstand $L$, så vil svingninger med bølgelængderne $\lambda = \nicefrac{L}{n}$, hvor $n=1,2,...$, danne stående bølger mellem hånden og tasten. Dette vil dæmpe bølger med andre bølgelængder mere end de stående bølger, hvorfor vi hører de stående bølger tydeligere. Når hånden bevæges, så ændres bølgelængderne, der giver stående bølger, hvilket giver en vibratoeffekt.
\end{opgave}
%%
\begin{opgave}[2]{Bølger og ubestemthed}
\opg Afstanden mellem to knudepunkter eller to bølgetoppe/bølgedale er $\pi$, hvorfor $\lambda = \pi$.
\opg Der er en bølgetop ved $x=0$ og en bølgedal halvanden bølgelængde derfra ved $x=\pm\pi$. Et bud på bølgelængden er derfor $\lambda=\nicefrac{2\pi}{3}$.
\opg Bølgen har toppunkt og dermed centrum i $x=0$.
\opg Nej. Sinus og cosinus er translationssymmetriske funktioner i hele multipla af $\pi$ -- dvs. eksempelvis $\sin(x) = \sin(x \pm n\pi)$ for $n=1,2,...$
\opg De funktioner, der har velbestemte bølgelængder, er sinus og cosinus, alle andre bølgers ``bølgelængde'' findes, ved at finde den sum af sinus- og cosinusfunktioner, der giver samme bølge. Grundet sinus' og cosinus' translationssymmetri har de intet centrum -- de er ikke lokaliseret i rummet noget sted. Det betyder, at hvis bølgen har en velbestemt bølgelængde, så er den ikke noget bestemt sted i rummet, men hvis den er et bestemt sted i rummet, så har den ikke en velbestemt bølgelængde. Kort sagt kan man ikke kende bølgens position og bølgelængde på samme tid med arbitrær præcision.
\end{opgave}
%%
%%
\begin{opgave}[2]{Løsninger på forskellige former}
k
\end{opgave}
%%
%%
\begin{opgave}[2]{oktaver}
k
\end{opgave}
%%
%%
\begin{opgave}[3]{Sammenhæng mellem hastighed og tryk for en lydbølge}
k
\end{opgave}
%%
%%
\begin{opgave}[1]{Grundfrekvenser af forskellige instrumenter}
k
\end{opgave}
%%
%%
\begin{opgave}[2]{streng og fløjte}
k
\end{opgave}
%%
%%
\begin{opgave}[3]{overtoner på klarinet}
k
\end{opgave}
%%
%%
\begin{opgave}{Stående bølger i en fløjte}
\end{opgave}
%%
%%
\begin{opgave}[1]{Stødtoner (pløg-ind-opgave)}
    k
\end{opgave}
%%
%%
\begin{opgave}[2]{Stødtoner}
    k
\end{opgave}
\end{document}
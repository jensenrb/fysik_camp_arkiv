\subsection*{Integralregning}
\begin{opgave}[1]{Integraler og arealer under funktioner}
I matematikafsnittet i kompendiet nævnes det, at når man regner et bestemt integral
\begin{equation*}
\int^a_b{f(x)}\dd{x},
\end{equation*}
så er svaret det samme som arealet under grafen for $f(x)$ i intervallet $[a,b]$ på $x$-aksen. Brug dette faktum til at diskutere den fysiske forståelse af følgende udsagn.
\opg Hvis man integrerer en hastighed $v(t)$ ift. tiden $t$, så får man en position $x(t)$.

Hastigheden er $\dv{x}{t}$, så når man integrerer får man et udtryk for positionen. Grænserne sørger for at det netop er den afstand der er tilbagelagt imellem $a$ og $b$
\opg Hvis man integrerer en acceleration $a(t)$ ift. tiden $t$, så får man en hastighed $v(t)$. 

Tilsvarende er accelerationen $\dv{v}{t}$, så når man integrerer den får man en hastighed, hvor grenserne giver den specifikke ændring i hastighed imellem $a$ og $b$.
\end{opgave}

\begin{opgave}[1]{Ubestemte integraler}
	Udregn det ubestemte integral af følgende funktioner:
	\opg $f(x) = x^3$:
    	\begin{align*}
    	    \int x^3\dd{x}=\frac{1}{4}x^4+k \: .
    	\end{align*}
	\opg $f(x) = x^2 + 4x$:
    	\begin{align*}
    	    \int x^2+4x\dd{x}=\frac{1}{3}x^3+2x^2+k \: .
    	\end{align*}
	\opg $f(x) = x^2 + \frac{1}{x^2}$:
	    \begin{align*}
	        \int x^2+\frac{1}{x^2}\dd{x}=\frac{1}{3}x^3-\frac{1}{x}+k \: .
	    \end{align*}
	\opg $f(x) = \cos (x)$:
	    \begin{align*}
	        \int \cos(x)\dd{x}=\sin(x)+k \: .
	    \end{align*}
	\opg $f(x) = \frac{1}{x}$:
	    \begin{align*}
	        \int\frac{1}{x}\dd{x}=\ln|x|+k \: .
	    \end{align*}
    
    Husk at man kan se på hver led i en sum separat, og løsningerne til disse integraler kan slåes op.
\end{opgave}

\begin{opgave}[2]{Størrelsen af en port}
    En tømmrer er blevet hyret til at lave tre porte.
    Toppen af alle portene kan beskrives med følgende funktioner
    \begin{align*}
        f(x)&=1-x^2,\\
        g(x)&=2-\frac{e^x+e^{-x}}{2},\\
        h(x) &= 1-\abs{x}.
    \end{align*}
    Alle portene er i intervallet $[-\frac{1}{2},\frac{1}{2}]$.
    \opg Opstil bestemte integraler til at udregne arealet af de tre porte.
    
    For at finde arealet tages det bestemte integral fra $\frac{-1}{2}$ til $\frac{1}{2}$. Det bliver
    \begin{align*}
        A_f&=\int_{-1/2}^{1/2}f(x)\dd{x}\\
        A_g&=\int_{-1/2}^{1/2}g(x)\dd{x}\\
        A_h&=\int_{-1/2}^{1/2}h(x)\dd{x}
    \end{align*}
    \opg Hvilken port er størst?
    
    For at finde den største port skal vi kende alle deres arealer. De findes ved at løse integralerne
    \begin{gather*}
        A_f=\int_{\-1/2}^{1/2}1-x^2\dd{x}=\left[x-\frac{x^3}{3}\right]_{-1/2}^{1/2}=1-\frac{1}{12}=0,91667\\
        A_g=\int_{\-1/2}^{1/2}2-\frac{e^x+e^{-x}}{2}\dd{x}=\left[2x+\frac{e^x-e^{-1}}{2}\right]_{-1/2}^{1/2}=2-e^{1/2}+e^{-1/2}=0.9578...
    \end{gather*}
    Til $h(x)$ er der den ekstra spidsfindighed at Det i virkeligheden er funktionen $h(x)=x$ for positive $x$ og $h(x)=-x$ for negative $x$.
    Så man bliver nød til at splitte integralet op i to intervaller
    $$
    A_h=\int_{\-1/2}^{1/2}1-\abs{x}\dd{x}=\int_{-1/2}^{0}1-\abs{x}\dd{x}+\int_{0}^{1/2}1-\abs{x}\dd{x}
    $$
    Vi kan dog udnytte at funktionen kan spejles i $y$-aksen, så arealerne på begge sidder er ens, det betyder at vi kan skrive:
    $$
    A_h=2\int_0^{1/2}1-x\dd{x}=2\left[x-\frac{x^2}{2}\right]_0^{1/2}=2(\frac{1}{2}-\frac{1}{8})=0,75
    $$
    Vi kan nu konkludere at $A_g$ er størst.
    \opg Hvilken er mindst?
    
    og at $A_h$ er mindst.
\end{opgave}
%%
%%
\begin{opgave}[3]{Lige og ulige funktioner}
En funktion så som $f(x) = x^2$ der opfylder kravet, $f(x) = f(-x)$, kaldes en lige funktion. En funktion som $f(x) = x$ der opfylder det lignende krav, $f(x)=-f(-x)$, kaldes for en ulige funktion.
Det vil sige, at en lige funktion er uændret, hvis man spejler den i $y$-aksen, mens en ulige funktion skifter fortegn ved den samme spejling.
Bemærk at de fleste funktioner er hverken lige eller ulige, og unikt er funktionen $f(x) = 0$ både lige og ulige.
Afgør om følgende funktioner er lige eller ulige.
\opg $\sin(x)$
Ulige, sinden $\sin(-x)=-\sin(x)$.
\opg $e^{x^2}$.
Lige siden $e^{(-x)^2}=e^{x^2}$.
\opg $\cos(x)$.
Lige, siden $\cos(-x)=\cos(x)$.
\opg Sinus er den eneste ulige funktion:
%
\begin{align*}
    \int_{-a}^{a}\sin(x)\dd{x} = \Big[-\cos(x)\Big]_{-a}^{a} = -\Big[\cos(a) - \cos(-a)\Big] = -\Big[\cos(a) - \cos(a)\Big] = 0,
\end{align*}
%
da cosinus er en lige funktion.
\opg Lad $f(x)$ betegne en reel ulige funktion omkring origo. Af midtpunktsreglen er
%
\begin{align*}
    \int_{-a}^{a}f(x)\dd{x} &= \int_{-a}^{0}f(x)\dd{x} + \int_{0}^{a}f(x)\dd{x} = -\int_{-a}^{0}f(-x)\dd{x} + \int_{0}^{a}f(x)\dd{x} \\
    &= \int_{a}^{0}f(y)\dd{y} + \int_{0}^{a}f(x)\dd{x} = -\int_{0}^{a}f(y)\dd{y} + \int_{0}^{a}f(x)\dd{x} \\
    &= -\int_{0}^{a}f(x)\dd{x} + \int_{0}^{a}f(x)\dd{x} = 0
\end{align*}
\end{opgave}
%%

% \begin{opgave}{Ubestemte højere ordens integraler}
%     Udregn følgende ubestemte højere ordens integraler.  Sæt alle integrationskonstanter til nul.
    
%     Den enkelte integraler tages en af gange, hvor alle andre variable betragtes som konstanter.
%     Hvilken rækkefølge man tager integralerne i er underordnet.
%     \opg $$\iint ye^{-x}\dd{x}\dd{y}=\int -ye^{-x}\dd{y}=-\frac{1}{2}y^2e^{-x}$$
%     \opg $$\iint y\cos(xy)\dd{x}\dd{y} =\int \frac{y\sin(xy)}{y}\dd{y}=\int \sin(xy)+k_x\dd{x}=\frac{-\cos(xy)}{y}$$
%     \opg $$\iiint (x^2+y^2)z\dd{x}z\dd{y}\dd{z}=\iint (x^2y+\frac{y^3}{3})z\dd{x}\dd{z}=\frac{1}{3}\int (x^3y+y^3x)z\dd{z} =\frac{1}{6}(x^3y+xy^3)z^2$$
%     \opg \begin{gather*}\iiint \sin(x)\sin(y)\sin(z)\dd{x}\dd{y}\dd{z}=-\iint\sin(x)\sin(y)\cos(z)\dd{x}\dd{y}\\=\int\sin(x)\cos(y)\cos(z)\dd{x}=-\cos(x)\cos(y)\cos(z)\end{gather*}
% \end{opgave}
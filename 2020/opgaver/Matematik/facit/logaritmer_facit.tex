\begin{opgave}[1]{Logaritmer og tal}
    Find værdien af de følgende udtryk
    \opg $\log_4(8) = \log_4(4\cdot 2)=\log_4(4)+\log_4(2)=\log_4(4)+\log_4(4^{1/2})=1+\frac{1}{2}=\frac{3}{2}$
    \opg $\log_\frac{1}{9} \left(\sqrt{27}\right)=\frac{1}{2}\log_\frac{1}{9}\left(27\right)=\frac{1}{2}\log_\frac{1}{9}\left(3^3\right)=\frac{3}{2}\log_\frac{1}{9}\left(3\right)=\frac{3}{2}\log_\frac{1}{9}\left(9^{1/2}\right)=\frac{3}{4}\log_\frac{1}{9}\left(\left(\frac{1}{9}\right)^{-1}\right)=\frac{-3}{4}\log_\frac{1}{9}\left(\frac{1}{9}\right)=\frac{-3}{4}$
    \opg $\ln(e^{2/3})=\frac{2}{3}\ln(e)=\frac{2}{3}$
    \opg $\ln(\frac{e^5}{e^3})=\ln(e^2)=2\ln(e)=2$
    %\opg $2^{\log_2(5 \cdot x)}=5x$
\end{opgave}

\begin{opgave}[2]{Ligningsløsning med logaritmer}
Find talværdien for den ukendte variabel, således at ligningen er sand
    \opg $\log[b](16) = 4/3 \iff b^{4/3}=16\iff b=16^{3/4}=4^{3/2}=2^3=8$
    \opg $\ln(x) = -1 \iff x=e^{-1}=0,36787944117$
    \opg $\log[2](1/x)=\frac{1}{5}\iff 2^{1/5}=\frac{1}{x}\iff x=2^{-1/5}=\frac{1}{32}$
\end{opgave}

%\begin{opgave}{Logaritmeapproximationer}
%Brug approximationerne $log_10(2) = 0,3010$ og $log_10(3)= 0,4771$ til at udregne værdien af %de følgende udtryk
%    \opg $\log[10](24)=\log_{10}(2^3\cdot 4)=3\log_{10}(2)+\log_{10}(3)=3\cdot 0,301+0,4771=1,3801$
%    \opg $\log[10](5)=\log_{10}\left(\frac{10}{2}\right)=\log_{10}(10)-\log_{10}(2)=1-0,301=0,699$
%    \opg $\log[10](4^{\frac{1}{3}})=\log_{10}(3^{-4})=-4\log_{10}(3)=-4\cdot 0,4771=1,9084$
%\end{opgave}

\begin{opgave}[3]{En ekstra logaritmeregneregel}
    Ud fra de tre logaritme regneregler, \eqref{mat:log}, vis;
    $$
    -b\log(a)=\log(a^{-1})
    $$
    Det følger direkte af (1.56.c)
    $$
        \log(a^{-b})=\log({a^b}^{-1})=-\log(a^b)=-b\log(a)
    $$
\end{opgave}
%%
%%
\begin{opgave}[3]{Eksponentregneregler}
    k
\end{opgave}
%%
%%
\begin{opgave}[4]{Logaritmeregneregler}
    k
\end{opgave}
\subsection*{Differentialligninger}
%%
\begin{opgave}[1]{Differentialligninger i ord}
Denne opgaver handler om at beskrive differentialligninger i egne ord.
\opg Hvad er en differentialligning?
\opg Løsninger til almindelige ligninger er tal. Hvad er løsninger til differentialligninger?
\opg Hvis man har et gæt på en løsning til en almindelig ligning, hvordan kan man finde ud af om gættet er en løsning?
\opg Hvis man har et gæt på en løsning til en differentialligning, hvordan kan man finde ud af om gættet er en løsning?
\end{opgave}
%%
%%
\begin{opgave}[2]{Hvornår er det en løsning?}
	I denne opgave skal vi finde ud af, hvornår nogle forskellige funktioner er løsninger til de givne differentialligninger. Sagt med andre ord skal vi finde de specifikke talværdier for alle de konstanter, der indgår i funktionerne, således at funktionerne løser deres respektive differentialligningerne. Bemærk at der kan være mere end 1 værdi.
	\opg For hvilke værdier af $k$ løser funktionen
	\begin{align*}
	f(x) = \cos(kx)
	\end{align*}
	differentialligningen
	\begin{align*}
	4 \dv[2]{f(x)}{x} = - 25f(x) \, .
	\end{align*}
	\opg Tjek for de værdier af $k$ I fandt i 1), at funktionen $g(x) = A \sin (kx) + B \cos (kx)$ også løser differentialligningen
	\begin{align*}
	4 \dv[2]{g(x)}{x} = -25 g(x) \, .
	\end{align*}
	\opg For hvilke værdier af $r$ løser funktionen
	\begin{align*}
	h(x) = e^{rx}
	\end{align*}
	differentialligningen
	\begin{align*}
	2 \dv[2]{h(x)}{x} + \dv{h(x)}{x} - h(x) = 0 \, .
	\end{align*}
	\opg Lad $r_1$ og $r_2$ være de konstanter du fandt i \textbf{3)}. Tjek at funktionen $k (x) = ae^{r_1x} + be^{r_2x} $ også løser differentialligningen
	\begin{align*}
	2 \dv[2]{k(x)}{x} + \dv{k(x)}{x} - k(x) = 0 \, .
	\end{align*}
\end{opgave}
%%
%%
\begin{opgave}[3]{Generelle 1. ordens differentialligninger}
I denne opgave skal du også vise, at nogle forskellige funktioner er løsninger til de givne differentialligninger. Denne gang er funktionerne og differentialligningerne dog skrevet op på en mere generel form, dvs. at de kan indeholde arbitrære konstanter.
\opg Vis at alle funktioner på formen
\begin{align*}
	h(x) = \frac{1}{x + A} 
\end{align*}
løser differentialligningen
\begin{align*}
	\dv{h}{x} = - h(x)^2 \; .
\end{align*}
\opg Vis at alle funktioner på formen
\begin{align*}
	k(x) = \left( c - x^2 \right)^{-1/2}
\end{align*}
løser differentialligningen
\begin{align*}
	\dv{k}{x} = x k(x)^3 \; .
\end{align*}
\opg Vis at alle funktioner på formen
\begin{align*}
	g(x) = \frac{\ln (x) + C}{x}
\end{align*}
løser differentialligningen
\begin{align*}
	x^2 \dv{g}{x} + xg(x) = 1 \; .
\end{align*} 
\opg Vis at alle funktioner på formen
\begin{align*}
	f(x) = \frac{1 + ce^x}{1-ce^x}
\end{align*}
løser differentialligningen
\begin{align*}
	\dv{f}{x} = \frac{1}{2} \left( f(x)^2 - 1 \right) \; .
\end{align*} 
\end{opgave}
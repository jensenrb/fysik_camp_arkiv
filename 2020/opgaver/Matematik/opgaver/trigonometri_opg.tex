\subsection*{Trigonometri}
%%
\begin{opgave}[1]{Retvinklede trekanter}
En retvinklet trekant har to kateter $a,b$ og en hypotenuse $c$. Vi antager at længderne af $a$ og $c$ er hhv. $5$ og $13$.
\opg Bestem vinklen mellem $a$ og $c$ vha. cosinus.
\opg Bestem længden af $b$ vha. tangens.
\opg Bestem vinklen mellem $b$ og $c$ vha. sinus.
\end{opgave}
%%
%%
\begin{opgave}[1]{Tangens}
Vi betragter en retvinklet trekant med to kateter, den hosliggende katete $a$ og den modstående katete $b$, samt en hypotenuse $c$.
\opg Vis hvorfor $\tan(\theta)=\nicefrac{b}{a}$ ud fra definitionen af sinus og cosinus.
\opg Løs ligningen $\tan(\theta)=1$ med den inverse tangensfunktion.
\opg Vis at ligningerne $\tan(\theta)=1$ og $\sin(\theta) = \cos(\theta)$ er ens.
\opg Løs ligningen $\sin(\theta) = \cos(\theta)$ ud fra en tegning som \cref{mat:fig:pol_koor}.
\opg Sammenlign løsningerne fra spørgsmål 2 og 4.
\end{opgave}
%%
%%
\begin{opgave}[1]{Koordinatskift}
Omskriv følgende punkter skrevet i kartesiske koordinater til polære koordinater. Dvs. find $r$ og $\theta$ vha. de givne $x$ og $y$. Lav også en lille skitse af punkterne i et $xy$-koordinatsystem, hvor $r$ og $\theta$ indtegnes.
\opg $x=2$ og $y=2$.
\opg $x=1$ og $y=-2$.\\
I 2) vælger man selv, om man vil regne $\theta$ som positiv (mod uret) eller negativ (med uret) ift. $x$-aksen.
\end{opgave}
%%
%%
\begin{opgave}[1]{Koordinatskift 2}
Omskriv følgende punkter skrevet i polære koordinater til kartetiske koordinater, og lav en lille skitse af punkterne i et $xy$-koordinatsystem.
\opg $r=5$ og $\theta=\dfrac{\pi}{4}$.
\opg $r=5$ og $\theta=\dfrac{7\pi}{4}$.
\opg $r=13$ og $\theta=\dfrac{11\pi}{6}$.
\end{opgave}
%%
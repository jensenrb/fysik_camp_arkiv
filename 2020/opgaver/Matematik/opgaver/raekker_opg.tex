\section*{Rækkeudviklinger}
%%
%%
\begin{opgave}[1]{Approksimation af funktion}
	I denne opgave skal I prøve at kigge lidt nærmere på Taylorapproksimationer, og hvad det egentligt er for en størrelse. Dette skal gøres på baggrund af formel \eqref{mat:eq:k-Taylor_pol} fra matematikafsnittet i kompendiet.
	\opg Diskuter betydningen af de enkelte led i summen. Hvordan ser de
	ud som funktioner af $x$.
	\opg Hvorfor bliver approksimationen bedre af at tage flere led med?
	\opg Hvilke led er de vigtigste, når $x$ er tæt på $a$?
	\opg Brug disse observationer til at forklare \cref{mat:fig:Taylorseries_figure} med dine egne ord.
\end{opgave}	
%%
%%
\begin{opgave}[2]{Taylorpolynomier for simple funktioner}{2}
I denne opgave skal I bestemme Taylorpolynomierne $T_0(x)$, $T_1(x)$ og $T_2(x)$ for følgende funktioner vha. \cref{mat:eq:k-Taylor_pol} i kompendiet for $a=0$.
\opg $f(x) = \cos(x)$.
\opg $f(x) = \sin(x)$.
\opg $f(x) = e^x$.
\opg Tjek om jeres resultater for $T_2(x)$ stemmer med Taylorpolynomierne i tabel \ref{k-Taylorseries_table} i matematikafsnittet i kompendiet. 
\end{opgave}
%%
%%
\begin{opgave}[3]{Flere gode Taylorudviklinger}
Betragt funktionerne $f(x) = (1+x)^{a}$ og $g(x) = \ln(x)$.
\opg Bestemt Taylorpolynomierne $T_0(x)$ og $T_1(x)$ for $f(x)$ med $a=0$.
\opg Brug dette til at vise, at $\sqrt(x) \approx 1+\nicefrac{x}{2}$ for $x\ll1$.
\opg Bestemt Taylorpolynomierne $T_0(x)$ og $T_1(x)$ for $g(x)$ med $a=1$.
\end{opgave}
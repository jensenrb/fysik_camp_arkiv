Den beskrivelse af lyd, de fleste kender, er at lyd er bølger, der udbredes i luften omkring os. Lydbølger skabes ved, at eksempelvis en højtalermembran svinger frem og tilbage, hvilket skiftevis øger og mindsker lufttrykket, hvorved trykket i luften bølger frem og tilbage -- der er skab en trykbølge. At lyd er bølger skaber også en masse interessante fænomener, der i forskellige grader kendes fra musik. Formålet med dette emne er derfor tvefoldigt: ud fra Newtons love vil vi vise, at lyd faktisk er bølger, og så vil vi undersøge nogle af disse spændende fænomener.
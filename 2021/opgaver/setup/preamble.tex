\usepackage[utf8]{inputenc}%Nødvendig for danske bogstaver
\usepackage[danish]{babel}%Sørger for at ting LaTeX gør automatisk er på dansk
\usepackage{csquotes}
\usepackage{geometry}%Til opsætning af siden
\geometry{lmargin = 2.5cm,rmargin = 2.5cm}%sætter begge magner
\usepackage{lipsum}%Fyldtekst, til brug under test af layoutet
\usepackage{float}
\usepackage{graphicx}%Tillader grafik
\usepackage{epstopdf}%Tillader eps filer
\usepackage{marginnote}% Noter i margen
\interfootnotelinepenalty=10000 %undgår at fodnoter bliver spilittet op.
\usepackage{biblatex}
\addbibresource{litteratur.bib}
\usepackage[font={small,sl}]{caption}	% Caption med skrå tekst ikke kursiv
\usepackage{subcaption} % Tillader underfigurer

% Wrapfigure 
\usepackage{wrapfig}

% Tikz-ting til kredsløbsanalyse
\usepackage{circuitikz}

% Equal sign with condition
\usepackage{extarrows}
\newcommand{\xleq}{\xlongequal}

\usepackage{xcolor} %Bruges til farver
\usepackage{forloop} %Bruges til nemmere for loops

\newcounter{opgave}[chapter] %Definerer opgavenumrene og hvornår de nulstilles
\renewcommand{\theopgave}{\thechapter.\arabic{opgave}} %Definerer udseende af opgavenummereringen
\newcounter{delopgave}[opgave] %Definerer delopgavenumrene
\newcounter{lvl} %Definerer en "variabel" til senere brug

\definecolor{markerColor}{rgb}{0.0745098039, 0.262745098, 0.584313725} %Definerer farven af markøren
\newcommand{\markerSymbol}{\ensuremath{\bullet}} %Definerer tegnet for markøren
\newlength{\markerLength} %Definerer en ny længde
\settowidth{\markerLength}{\markerSymbol} %Sætter den nye længde til bredden af markøren

\newenvironment{opgave}[2][0]{%Definerer det nye enviroment, hvor sværhedsgraden er den første parameter med en default på 0
\newcommand{\opg}{\refstepcounter{delopgave}\par\vspace{0.1cm}\noindent\textbf{\thedelopgave)\space}}%Definerer kommando til delopgave
\refstepcounter{opgave}%Forøger opgavenummer med 1 og gør den mulig at referere til
\setcounter{lvl}{#1}%Sætter "variablen" lvl lig med angivelsen af sværhedsgraden
\noindent\hspace*{-0.75em}\hspace*{-\value{lvl}\markerLength}\forloop{lvl}{0}{\value{lvl}<#1}{{\color{markerColor}\markerSymbol}}\hspace*{0.75em}%Sætter et antal af markører svarende til sværhedsgraden
\textbf{Opgave \theopgave : #2}\newline\nopagebreak\ignorespaces}{\bigskip\par\ignorespacesafterend} %Angiver udseende af titlen på opgaverne samt mellemrummet mellem opgaver

% Eksempelenvironment
\newcounter{example}[chapter]
% Eksempelnummer tælles med romertal
\renewcommand{\theexample}{\Roman{example}}
% Selve environmentet defines
\newenvironment{example}[1][]{\refstepcounter{example}\par\bigskip\noindent
  \textbf{Eksempel~\theexample: #1} \newline\nopagebreak\ignorespaces}{}

\newcommand{\el}[1]{\dv{}{t}\left(\pdv{L}{\dot{#1}}\right)}

\usepackage{mathtools}%Værktøjer til at skrive ligninger
\renewcommand{\phi}{\varphi}%Vi bruger varphi
\renewcommand{\epsilon}{\varepsilon}%Vi bruger varepsilon
\usepackage{physics}%En samling matematikmakroer til brug i fysiske ligninger
\usepackage{braket}%Simplere kommandoer til bra-ket-notation
\usepackage{siunitx}%Pakke der håndterer SI enheder godt
\DeclareSIUnit\clight{\text{\ensuremath{c}}} % Lysets fart i vakuum som c og ikke c_0
\sisetup{separate-uncertainty=true}
\sisetup{output-decimal-marker={,}}
\DeclareSIUnit{\parsec}{pc}
\DeclareSIUnit{\lightyear}{ly}
\DeclareSIUnit{\astronomicalunit}{AU}
\DeclareSIUnit{\year}{yr}
\DeclareSIUnit{\solarmass}{M_\odot}
\DeclareSIUnit{\solarradius}{R_\odot}
\DeclareSIUnit{\solarluminosity}{L_\odot}
\DeclareSIUnit{\solartemperature}{T_\odot}
\DeclareSIUnit{\earthmass}{M_\oplus}
\DeclareSIUnit{\earthradius}{R_\oplus}
\DeclareSIUnit{\jupitermass}{M_J}


\usepackage{tikz}
\usetikzlibrary{decorations.pathmorphing,patterns,math}

\newcommand{\eqbox}[1]{\begin{empheq}[box=\fbox]{align}
	\begin{split}
	#1
	\end{split}
\end{empheq}}

\usepackage{listings}

% Custom colors
\usepackage{color}
\definecolor{deepblue}{rgb}{0,0,0.5}
\definecolor{deepred}{rgb}{0.6,0,0}
\definecolor{deepgreen}{rgb}{0,0.5,0}
%------------ Python style -------------%

\lstset{literate=%
{æ}{{\ae}}1
{å}{{\aa}}1
{ø}{{\o}}1
{Æ}{{\AE}}1
{Å}{{\AA}}1
{Ø}{{\O}}1
}
\lstset{extendedchars=\true}
\lstset{inputencoding=ansinew}

% Python style for highlighting
\newcommand{\pythonstyle}{\lstset{
language=Python,
basicstyle=\ttfamily,
morekeywords={self},              % Add keywords here
keywordstyle=\bfseries\color{deepblue},
emph={MyClass,__init__},          % Custom highlighting
emphstyle=\bfseries\color{deepred},    % Custom highlighting style
stringstyle=\color{deepgreen},
frame=tb,                         % Any extra options here
showstringspaces=false,
upquote=true
}}

% Python environment
\lstnewenvironment{python}[1][]%
{
\pythonstyle
\lstset{#1}
}
{}

% Python for external files
\newcommand\pythonexternal[2][]{{
\pythonstyle
\lstinputlisting[#1]{#2}}}

% Python for inline
\newcommand\pythoninline[1]{{\pythonstyle\lstinline!#1!}}
%------------ Python style end -------------%


%--- Opsætning til referencer til kompendiet ---%
%
\usepackage{xr}
\usepackage[hidelinks]{hyperref}%Tillader links\usepackage{xr} %Tillader refferencer paa tvaers af dokumenter
% \externaldocument{../kompendium/main} %Indsæt relativ sti til kompendium og kompiler lokalt
\externaldocument[k-]{main_kompendium} %Indsæt relativ sti til kompendium og kompiler lokalt
\usepackage[danish]{cleveref}
\creflabelformat{equation}{#2(#1)#3}
\crefrangelabelformat{equation}{#3(#1)#4 to #5(#2)#6}
\crefname{equation}{ligning}{ligningerne}
\Crefname{equation}{Ligning}{Ligningerne}
\crefname{section}{afsnit}{afsnitene}
\Crefname{section}{Afsnit}{Afsnitene}
\crefname{figure}{figur}{figurene}
\Crefname{figure}{Figur}{Figurene}
\crefname{table}{tabel}{tabellerne}
\Crefname{table}{Tabel}{Tabellerne}
\crefname{example}{eksempel}{eksemplerne}
\Crefname{example}{Eksempel}{Eksemplerne}
\crefname{opgave}{opgave}{opgaverne}
\Crefname{opgave}{Opgave}{Opgaverne}
\crefname{delopgave}{delopgave}{delopgaverne}
\Crefname{delopgave}{Delopgave}{Delopgaverne}


\newcommand{\kb}{\ensuremath{k_\textsc{b}}}

\DeclareSIUnit{\parsec}{pc}
\DeclareSIUnit{\lightyear}{ly}
\DeclareSIUnit{\astronomicalunit}{AU}
\DeclareSIUnit{\year}{yr}
\DeclareSIUnit{\solarmass}{M_\odot}
\DeclareSIUnit{\solarradius}{R_\odot}
\DeclareSIUnit{\solarluminosity}{L_\odot}
\DeclareSIUnit{\solartemperature}{T_\odot}
\DeclareSIUnit{\earthmass}{M_\oplus}
\DeclareSIUnit{\earthradius}{R_\oplus}
\DeclareSIUnit{\jupitermass}{M_J}
\documentclass[crop=false, class=memoir]{standalone}
\documentclass[a4paper,hidelinks,12pt]{memoir}
\usepackage[utf8]{inputenc} % Do not change or remove!
\usepackage[T1]{fontenc} % Do not change or remove
\usepackage[danish]{babel} % Sproget, vi skriver på
\renewcommand\danishhyphenmins{22} % Kun hvis vi skriver på dansk

%%%%%%%%%%%%%%%%%%%%%%%%%%%%%%%%%%%%%%%%%%%%%%%%%%%%%
% Niels Jakob Søe Loft                              %
% nsl@phys.au.dk                                    %
%%%%%%%%%%%%%%%%%%%%%%%%%%%%%%%%%%%%%%%%%%%%%%%%%%%%%

% Denne skabelon er baseret på Rasmus Villemoes' veldokumenterede
% phd-afhandling i matematik, som jeg har ændret på, så den passer til
% et bachelorprojekt i fysik. Som hovedregel er ting kommenteret på
% engelsk fra Rasmus' skabelon, mens jeg har skrevet på dansk. De
% væsentligste ændringer er, at skabelonen er gjort mere egnet til et
% mindre projekt som et bachelorprojekt er i forhold til en
% phd-afhandling, hvorfor nogle ting er skåret væk, og jeg har
% inkluderet en liste fysik-relaterede makroer. Desuden er
% bibliografien konverteret fra BibTeX til BibLateX pr. marts 2014.

% Pr. 29. marts 2014 har jeg ændret skabelonen, så den kan bruges til
% kompendiet til UNFs Fysik Camp 2014.

%%%%%%%%%%%%%%
%% Generelt %%
%%%%%%%%%%%%%%
% ***************** UNF Science camp  kompendie ***************** %
% Dette dokument indeholder enviroments, comannds, makroer og
% layot specifikt til UNF science camp kompendier

% Pakker der anvendes. Kendte 'issues:
%	- xcolor skal loades før pdfpages, da den ellers loades uden dvipsnames
\usepackage[dvipsnames]{xcolor}		% Farver
\usepackage{xparse}							% Mere flexibel definition af makroer
\usepackage{marginnote}					% Noter i margen
\usepackage{forloop}						% Mulighed for forløkker



% ***************** Opgave enviroment ***************** %
% Sætter en opgave op og angiver sværhedsgraden. Opgavenummereringen nulstilles
% efter hvert ny kapitel.
% Anvenedelse: 
%		\begin{opgave}[farve]{Titel}{Sværhedgrad}
%			Introduktion
%			\opg
%			Delopgave 1
%			\opg
%			Delopgave 2
%			...
%		\end{opgave}
%
% Definer selve enviromentet. i´
\newcounter{opgave}[chapter]
\newcounter{delOpgave}[opgave]
\newenvironment{opgave}[3][NavyBlue]
	{\newcommand{\opg}{{{\refstepcounter{delOpgave}\smallskip\newline\textbf\thedelOpgave})\,}}
	\noindent\ignorespaces\refstepcounter{opgave}\newline\textbf{Opgave \theopgave:\,#3 #2}\newline}
	{\newline\bigskip}
% Definer 
%\newcommand{\lvl}[2][NavyBlue]{
%	\setcounter{nBullets}{#2}
%	\addtocounter{nBullets}{1}
%	\checkoddpage
%	\ifoddpages
%		\normalmarginpar
%		\marginnote{\textcolor{#1}{\lvltoken{\value{nBullets}}}}
%	\else
%		\reversemarginpar
%		\marginnote{\textcolor{#1}{\lvltoken{\value{nBullets}}}}
%	\fi
%}
\NewDocumentCommand{\lvl}{ O{NavyBlue} O{$ \bullet $} m}{
	\setcounter{nBullets}{#3}
	\addtocounter{nBullets}{1}
	\checkoddpage
	\ifoddpage
	\normalmarginpar
	{\textcolor{#1}{\lvltoken[#2]{\value{nBullets}}}}
	\else
	\reversemarginpar
	{\textcolor{#1}{\lvltoken[#2]{\value{nBullets}}}}
	\fi
}
\newcounter{lvl}
\newcounter{nBullets}
\newcommand{\lvltoken}[2][$ \bullet $]{
	\forloop{lvl}{1}{\value{lvl} < #2}{#1}} % load UNF-layout
\usepackage{graphicx} % Billeder
\usepackage{float}
\usepackage{epstopdf} % Så vi kan indsætte eps-filer
\usepackage{lipsum} % Dummytekst
\usepackage{pdfpages} % Indsættelse af pdf-sider
\usepackage{url} % Håndtering af URL'er
\usepackage{subfiles}
\usepackage{xspace} % Smarte mellemrum i egne makroer
\usepackage[final]{fixme} % Indsæt kommentarer i margin
%\usepackage{xstring} % Til sværhedsgrad-makro (se old/macros)
\usepackage[misc]{ifsym} % Til sværhedsgrad, skriv \Cube{n} hvor n=1,2,3
%\setcounter{secnumdepth}{3}
\setsecnumdepth{subsection}
\usepackage{newtxtext}
\usepackage{newtxmath}
\usepackage{subcaption} %sub-figurer
\usepackage{framed} % tekst-bokse
\usepackage{wrapfig}
\usepackage{enumitem}
\usepackage{microtype} % Mellemrumsjustering
\usepackage{xcolor} % flere farver
\usepackage{csquotes}%pæne citater
\usepackage{tikz} % tegninger i latex
\usepackage{empheq}
\usetikzlibrary{decorations.pathmorphing,patterns} % til tikz
\usetikzlibrary{calc}
\usetikzlibrary{decorations.pathmorphing}
\usetikzlibrary{decorations.markings}
\usetikzlibrary{positioning, shapes, snakes, arrows}
\tikzset{
fermion/.style={very thick,postaction={decorate},
  decoration={markings,mark=at position .6 with {\arrow[#1]{latex}}}},
boson/.style={very thick,dashed,postaction},
gluon/.style={thick,decorate,
 decoration={coil,amplitude=4pt, segment length=5pt,  pre length=.05cm, post length=.05cm}},
photon/.style={very thick,decorate, decoration={snake, segment length=8pt, amplitude=2pt, pre length=.05cm, post length=.05cm}},
}
\newcommand{\aq}[1]{$\bar{\mathrm{#1}}$}
\newcommand{\vertex}[1]{\fill (#1) circle (1 mm)}
%For at gøre det lettere at tegne Feynman diagrammer.

\interfootnotelinepenalty=10000 %undgår at fodnoter bliver spilittet op.

%\usepackage{cleveref}
%\creflabelformat{equation}{#2(#1)#3}
%\crefrangelabelformat{equation}{#3(#1)#4 to #5(#2)#6}
%\renewcommand{\ref}[1]{\eqref{#1}}
%\Crefname{equation}{ligning}{ligningerne}
%\Crefname{section}{afsnit}{afsnitene}
%\Crefname{table}{tabel}{tabellerne}

%% Bibliografi og referencer

%\usepackage{natbib} % Til biblografi, hvis man IKKE bruger BibLaTeX

%\usepackage[style=alphabetic,  % alternativt: style=numeric
%            backend=biber]{biblatex} % BibLaTeX, kræver installering
                                % af biber-pakken
%\addbibresource{kompendie.bib} % BibLaTeX tager referencer fra bach.bib

% \usepackage{cleveref} % Smarte referencer: skriv \cref{...} for småt forbogstav og \Cref{...} for stort forbogstav
% \crefname{equation}{ligning}{ligningerne}
% \Crefname{equation}{Ligning}{Ligningerne}
% \crefname{figure}{figur}{figurerne}
% \Crefname{figure}{Figur}{Figurerne}
% \crefname{table}{tabel}{tabellerne}
% \Crefname{table}{Tabel}{Tabellerne}
% \crefname{chapter}{kapitel}{kapitlerne}
% \Crefname{chapter}{Kapitel}{Kapitlerne}
% \crefname{section}{afsnit}{afsnittene}
% \Crefname{section}{Afsnit}{Afsnittene}

\usepackage[colorlinks=true, hidelinks]{hyperref} % Farvede links

% Glossary setup af Benjamin Muntz
\let\printglossary\relax 
\let\theglossary\relax
\let\endtheglossary\relax
%
\usepackage[toc,section=chapter]{glossaries}
\newglossary{symboler}{sym}{sbl}{Symbolliste}
\makeglossary
\newglossaryentry{Multiplicitet}{
    type=symboler,
    name={\ensuremath{\Omega}},
    sort=fnc,
    description={Multiplicitet}
}

%%%%%%%%%%%%%%%%%%%%%%
%% Tekst og formler %%
%%%%%%%%%%%%%%%%%%%%%%

%\usepackage[osf]{mathpazo} % Skrift

\usepackage{wasysym} % Font til smileys \smiley og \frownie

%\usepackage[sf]{libertine} % Til slanted skrift NJ's emacs er pigesur
\usepackage{libertine}

\linespread{1.06} % Større linjeafstand pga. font
\usepackage{fourier-orns} % Sjove symboler NJ's emacs er pigesur igen
\usepackage{textcomp} % Tilføjer flere tegn
\renewcommand\ttdefault{txtt} % Pænere teletype-skrift
\usepackage{physics}%En stor samling makroer
\renewcommand{\epsilon}{\varepsilon} %Skriver epsilon som varepsilon
\renewcommand{\varphi}{\phi} %Skriver varphi som phi
%Et par ekstra makroer
\newcommand{\xhat}{\vu x}
\newcommand{\yhat}{\vu y}
\newcommand{\zhat}{\vu z}
\newcommand{\xyz}[3]{\begin{pmatrix}#1\\#2\\#3\end{pmatrix}}
%\renewcommand{\Vec}[1]{\va{#1}}
\usepackage{mathtools} % Matematiktricks
\usepackage{cancel} % Ting der går ud med hinanden
\usepackage{siunitx} %SI-enheder
\sisetup{separate-uncertainty=true % gør at siunitx skriver +/- i
  % stedet for at bruge parentes til
  % at angive usikkerheder.
  ,output-decimal-marker={,}, % gør at der bruges komma til komma og
  % ikke punktum som i USA.
  ,load=abbr, % så vi kan bruge \keV
  ,exponent-product = \cdot, output-product = \cdot, % skift gangetegn fra \times til \cdot
}
%%% VI LAVER NOGLE FYSIK- OG MATEMATIK-MAKROER:


%% Generelt
%\newcommand{\g}{\cdot} % Prikprodukt, gangetegn
\newcommand{\subv}[2]{\gv{#1}_{\text{#2}}} % Pæn subscript til vektorer
\newcommand{\sub}[2]{#1_{\text{#2}}} % Pæn subscript til
\newcommand{\e}{\mathcal{E}} % Skrevet E
\newcommand{\abs}[1]{\left| #1 \right|} % Numerisk værdi
\newcommand{\N}{\ensuremath{\mathbb{N}}} % Naturlige tal
\newcommand{\Z}{\ensuremath{\mathbb{Z}}} % Hele tal
\newcommand{\Q}{\ensuremath{\mathbb{Q}}} % Rationelle tal
\newcommand{\R}{\ensuremath{\mathbb{R}}} % Reelle tal
\newcommand{\C}{\ensuremath{\mathbb{C}}} % Komplekse tal
\newcommand{\F}{\ensuremath{\mathbb{F}}} % Legeme tal
\newcommand{\A}{\ensuremath{\mathbb{A}}} % Algebraiske tal

%% Angiv sværhedsgrad til opgaver (benytter \usepackage{xstring})
%\newcommand{\lvl}[1]{%
%\IfStrEqCase{#1}{{1}{\ensuremath{\star}}
%    {2}{\ensuremath{\star\star}}
%    {3}{\ensuremath{\star\star\star}}}
%    [nada]
%}

%% Infinitesimalregning

\let\underdot=\d % omdøb indbygget kommando \d{} til \underdot{}
%\renewcommand{\d}[2]{\partial_{#2} \, #1} % afledt
%\newcommand{\dd}[2]{\partial_{#2}^2 \, #1} % dbl.afledt

%differentierings d
\renewcommand{\d}{\mathrm{d}}

%haard differentiering
\newcommand{\dif}[3][]{\frac{\d^{#1}{#3}}{{\d {#2}}^{#1}}}

%partiel differentiering
\newcommand{\pdif}[3][]{\frac{\partial^{#1}{#3}}{\partial {#2}^{#1}}}

\newcommand{\dt}[1]{\dot{#1}} % afledt mht. t (dot-notation)
\newcommand{\ddt}[1]{\ddot{#1}} % dbl.afledt mht. t (dbl.dot)

\newcommand{\integral}[4]{\int_{#3}^{#4} \, #1 \, \textrm{d}#2} % integrere



% Vektorer

\newcommand{\xyz}[3]{\begin{bmatrix} #1 \\ #2 \\ #3 \end{bmatrix}} %3D-vektor
\newcommand{\xy}[2]{\begin{bmatrix} #1 \\ #2 \end{bmatrix}} %2D-vektor
\let\vaccent=\v % Omdøb \v{} til \vaccent{}

\newcommand{\gv}[1]{{\vec{\mathbf{#1}}}} % Vektor med græske bogstaver
\renewcommand{\v}[1]{\gv{#1}} % Vektor med fed
\newcommand{\hatvec}[1]{\hat{\mathbf{#1}}} % Hatvektor
\newcommand{\ihat}{\boldsymbol{\hat{\textbf{\i}}}} % Enhedsvektor i
\newcommand{\jhat}{\boldsymbol{\hat{\textbf{\j}}}} % .. j
\newcommand{\khat}{\mathbf{\hat{k}}}  % .. k
\newcommand{\xhat}{\mathbf{\hat{x}}} % Enhedsvektor x
\newcommand{\yhat}{\mathbf{\hat{y}}} % .. y
\newcommand{\zhat}{\mathbf{\hat{z}}} % .. z
\newcommand{\grad}[1]{\gv{\nabla} #1} % Gradient
\let\divsymb=\div % Omdøb \div til \divsymb
\renewcommand{\div}[1]{\gv{\nabla} \cdot \v{#1}} % Divergens
\newcommand{\curl}[1]{\gv{\nabla} \times \v{#1}} % Curl
% Vil man tage div eller curl af græske bogstaver,
% skal man lade argumentetet være fx \gv{\mu} for µ-vektor

% Kvantemekanik

\newcommand{\op}[1]{\hat #1} % operator

\newcommand{\expect}[1]{\left< #1 \right>} % Forventningsværdi
\newcommand{\trace}{\ensuremath{\text{Tr}}\xspace}
\newcommand{\Hilbert}{\ensuremath{\mathcal{H}}}
\newcommand{\lag}{\ensuremath{{L}}}
\newcommand{\tr}[1]{\text{Tr}\left(#1\right)} % Trace
\newcommand{\ptr}[2]{\text{Tr}_{#1}\left(#2\right)} % Partial trace
\newcommand{\ket}[1]{\left| #1 \right>} % Dirac-notation: ket
\newcommand{\bra}[1]{\left< #1 \right|} % bra
\newcommand{\braket}[2]{\left< #1 \vphantom{#2} \, \right|
  \left. \! #2 \vphantom{#1} \right>} % bracket
\newcommand{\matrixel}[3]{\left< #1 \vphantom{#2#3} \right|
  #2 \left| #3 \vphantom{#1#2} \right>} % Bracket med ekstra streg
 % En masse matematik- og fysikmakroer

%%%%%%%%%%%%
%% Layout %%
%%%%%%%%%%%%

%\newcommand{\anonbreak}{\fancybreak{$* * *$}} % Break med stjerner
%\let\bar\overline % Gør at en bar over et symbol kan skalere efter symbolet

%% Sidehoved- og fod

\makepagestyle{tket}
\makeevenfoot{tket}{\thepage}{}{}
\makeoddfoot{tket}{}{}{\thepage}
\makeevenfoot{plain}{\thepage}{}{}
\makeoddfoot{plain}{}{}{\thepage}
\makeevenhead{tket}{\leftmark}{}{}


%% Margin

% Man kan sætte margins ved enten at specificere marginstørrelsen
% eller ved at specificere tekstblokken. Man skal vælge én og kun én
% af mulighederne.

% Specificer marginstørrelsen
%\setulmarginsandblock{2.7cm}{*}{1}
%\setlrmarginsandblock{1.6cm}{1.6cm}{*} 
%\setlength{\oddsidemargin}{-1cm} % Giver mere plads på siden
%\setlength{\topmargin}{-1.2cm} % Gør topmargin behagelig at se på
%\setlength{\columnsep}{1.5\columnsep}  % Afstand mellem søjlerne


\setlrmarginsandblock{2.5cm}{2.5cm}{*}

\usepackage[font={small,it}]{caption}	% Italic captions

% Tekstblok: Følgende er fra Rasmus Villemoes' thesis-layout.tex
%\setlxvchars[\normalfont] % standardbredden af tekstblok er ca. 65 tegn
%\settypeblocksize{*}{1.2\lxvchars}{1.61803} % højde, bredde, forhold
%\setulmargins{*}{*}{1.3} % lav bundmargin lidt større end topmargin
\checkandfixthelayout % memoir tjekker, at alt er ok og konsistent

\usepackage{ctable} % Tillader fede linjer i tabeller

%%%%%%%%%%%%%%%%%
%% Bibliografi %%
%%%%%%%%%%%%%%%%%

\usepackage[style=ieee]{biblatex}
\addbibresource{litteratur.bib}

%%%%%%%%%%%%%%%%%%
%% Definitioner %%
%%%%%%%%%%%%%%%%%%

% Definer titlen på projektet
 \newcommand{\thesistitle}{Kompendie til UNF Fysik Camp 2019}

%%%%%%%%%%%%%%%%%%%%%%
%% Slut på preamble %%
%%%%%%%%%%%%%%%%%%%%%%
\usepackage{import}
\begin{document}
\chapter{Speciel Relativitetsteori} \label{chap:rel_opg}


\begin{opgave}[1]{Det Galileiske Relativitetsprincip}
Det Galileiske Relativitetsprincip siger, at Newtons bevægelseslove er ens i alle inertielle referencesystemer.\\
Vi forestiller os nu et tog, der kører med en konstant hastighed $v$ ift. sporet. En passager i toget tager så en sten og slipper den fra hvile.  
\opg Brug Galileis relativitetsprincip til at beskrive stenens bevægelse set fra en observatør i toget.
\opg Brug Galilei-transformationen \cref{k-rel:eq:galilei} i kompendiet til at give en beskrivelse af stenens bevægelse set fra en observatør på Jorden. 
\end{opgave}

\begin{opgave}[1]{Kombination af Galileiske transformationer}	
	To tog kører parallelt med hinanden i $x$-retningen, det ene med hastighed $V$ og det andet med hastighed $U$,
	relativt til jorden. Lad det første tog være $S'$ og det andet $S''$, med deres respektive koordinater. Jorden betegnes $S$.
	\opg Hvad er den Galileiske transformation fra $S \left( x,y,z,t \right)$ til $S'' \left( x'',y'',z'',t'' \right)$?
	\opg Hvad er den Galileiske transformation fra $S'$ til $S''$?
	\opg Hvad svarer størrelsen $\left( U- V \right)$ til?	 
\end{opgave}

\begin{opgave}[1]{Løbetur i Regnvejr}
En dag hvor det regner, falder dråberne ned med $\SI{2}{\metre\per\second}$. En person løber vandret af sted med $\SI{3}{\metre\per\second}$. Ved hvilken vinkel ift. vandret skal personen holde sin paraply for bedst muligt at skærme for regnen? (hint: Kig på regndråbernes bevægelse set fra løberens referencesystem).
\end{opgave}

\begin{opgave}[2]{Flyvetur i Blæsevejr}
I denne opgave skal Gallileitransformationerne for hastighed bruges. Disse er
%
\begin{subequations} \label{rel:eq:galilei_hastighed}
\begin{align}
    u'_x &= u_x-v, \\
	u'_y &= u_y, \\
	u'_z &= u_z,
\end{align}
\end{subequations}
%
for et objekt med hastigheden $(u_x, u_y, u_z)$ i det umærkede koordinatsystem, der bevæger sig med farten $v$ i forhold til det mærkede system. En flyvemaskine kan flyve med  $\SI{500}{\km\per\hour}$, og der er en vindhastighed på $\SI{200}{\km\per\hour}$ fra vest mod øst.
\opg Flyets pilot styrer flyet mod nord. I hvilken retning bevæger flyet sig, og hvad er flyets hastighed set fra en observatør på Jorden, som betegnes $S$? \emph{Hint}: Kig på et referencesystem $S'$, der bevæger sig med vinden og benyt hastighedstransformationerne, \cref{rel:eq:galilei_hastighed}, til at oversætte flyets bevægelse i dette system til $S$.
\opg I hvilken retning skal piloten styre, hvis flyet skal flyve mod nord? Hvad er flyets hastighed set fra en observatør på Jorden i dette tilfælde?
\end{opgave}

\begin{opgave}[2]{Flodræset}
En flod er $\SI{20}{m}$ bred og vandet i floden strømmer af sted med en hastighed på $\SI{1}{m/s}$. To svømmere Arthur og Barbara arrangerer et ræs. Arthur skal svømme $\SI{20}{m}$ ned af floden og tilbage, mens Barbara skal svømme lige over floden og tilbage. Både Arthur og Barbare kan svømme med $\SI{2}{m/s}$.
\opg I hvilken retning skal Barbara svømme, for at hun kommer lige over floden?
\opg Hvem vinder ræset og med hvor meget?
\end{opgave}

\begin{opgave}[1]{Hvornår er relativitetsteori virkelig nødvendig?}
	Som det ses indgår $\gamma$ meget ofte i relativitetsteori. Når $\gamma$ er væsentligt større end 1, er det nødvendigt at bruge de relativistiske udtryk, frem for de Galileiske. For hvilken hastighed (i enheder af $c$) er
	værdien af $\gamma$:
	\opg \SI{1}{\percent} større end 1?
	\opg \SI{10}{\percent} større end 1?
	\opg \SI{100}{\percent} større end 1?
\end{opgave}

% Opgave kræver formel for relativistisk masse. Ikke givet i årets kompendium
% \begin{opgave}[1]{Et lille tankeeksperiment}
% 	De relativistiske effekter ses ikke i hverdagen, fordi $c$ er så stor, sammenlignet med hastigheder vi oplever i
% 	hverdagen. Hvad nu hvis lysets hastighed var meget mindre? Lad os se hvad der sker, hvis nu $c = \SI{44,72}{km/t}$.
% 	\opg Usain Bolts topfart er $\SI{44,72}{km/t}$. Hans hvilemasse er \SI{94}{\kg}. Hvad er hans masse, når han når topfart?
% \end{opgave}

\begin{opgave}[1]{Muoner i Jordens atmosfære}
	Muoner er ustabile sub-atomare partikler, der med en levetid på \SI{2,2}{\micro\second} (\SI{2,2e-6}{s}) henfalder til elektroner.
	Muoner produceres omkring \SI{10}{\km} over Jordens overflade, hvor energirige partikler fra rummet rammer
	atmosfæren, og de rejser med en hastighed tæt på lysets i forhold til Jorden, lad os sige $v = 0,999c$.
	\opg Hvad er den længste afstand en muon kan nå at rejse i sin levetid på \SI{2,2}{\micro\second}?
	\opg Fra ovenstående lader det til, at muonerne aldrig vil nå os på overfladen. Ikke desto mindre detekterer vi
	dem! Men levetiden angivet er i muonens hvilesystem. Hvad er dens levetid målt for en observatør på
	Jorden?
	\opg Hvor langt vil muonen nå nu?
	\opg Fra muonens synspunkt lever den stadig kun \SI{2,2}{\micro\second}. Hvad er tykkelsen af \SI{10}{\km} atmosfære, set fra
	muonens system?
\end{opgave}

\begin{opgave}[1]{Relativitet og rumfart}
	For nyligt valgte NASA at pensionere deres rumfærger. Indtil da var rumfærgen en forholdsvis billig måde at
	fragte udstyr og mennesker ud i rummet, fordi færgen og det meste af det man brugte til at sende den op med
	kunne genanvendes. Efter endt mission kunne rumfærgen lande som et fly.\\
	\indent
	En observatør på Jorden måler en landingsbane til at være \SI{3600}{\m}. En rumfærge befinder sig i kredsløb om
	Jorden med en hastighed af \SI{4,00e-7}{\m\per\s} relativt til Jorden. Vi antager, at dens bane er en ret linje under hele
	opgaven, og at den flyver parallelt med landingsbanen.
	\opg Hvad er længden af landingsbanen målt af piloten på rumfærgen?
	\opg En observatør på Jorden måler tiden der går, fra rumfærgen er direkte over den ene ende af landingsbanen,
	og til den er over den anden ende. Hvor lang tid får vedkommende?
	\opg Piloten på rumfærgen måler den tid, det tager ham at flyve længden af landingsbanen. Hvilken værdi får
	han?
	\opg Rumfærgen har en vægt på godt $2000$ tons. Hvad ville rumfærgen veje, hvis en observatør på Jorden
	kunne veje den, mens den var i kredsløb, dvs. hvad er dens relativistiske masse?
	\opg Rumfærgen er \SI{60}{\m} lang og \SI{10}{\m} meter høj i dens eget referencesystem. Hvor lang og høj er rumfærgen
	for en observatør på Jorden?
\end{opgave}

\begin{opgave}[2]{TvillingeParadokset}
	Tvillinge-paradokset er et af de mest kendte paradokser inden for speciel relativitetsteori. Egentligt er det ikke
	et paradoks, da Einstein allerede løste det tilbage i 1905. I denne opgave skal I også løse det. Det kræver, at man
	lige tænker sig lidt om.
	
	Barbara og Arthur er tvillinger. Arthur bliver på Jorden, mens Barbara rejser af sted med et rumskib, med
	hastighed nær $c$. På et tidspunkt vender rumskibet hurtigt, og flyver tilbage til Jorden. Da Barbara kommer
	tilbage, mødes hun med Arthur til en sammenligning. For Arthur har Barbara rejst ud og hjem igen med nær
	lysets hastighed. Derfor er tiden for hende gået langsommere, og hun vil derfor se yngre ud end Arthur. Men
	fra Barbaras synspunkt er det jo Arthur, som har bevæget sig i forhold til hende. Derfor bruger hun samme
	argument til at konkludere, at han vil se yngre ud end hende. Samtidigt er der jo ikke noget referencesystem,
	som er bedre end andre, så et argument der bygger på dette vil komme frem til, at alle resultater må være
	symmetriske mellem de to tvillinger. De er altså lige gamle.
	\opg Ud fra ovenstående lader det til, at der er tre muligheder, men kun en kan være rigtig. Hvilken mulighed
	er det?
	\opg Hvis en af tvillingerne er ældst, kan du så sige noget om, hvor lang tid der er gået for den yngste i
	forhold til den ældste?
\end{opgave}

% Opgave specifik til kompendiet i 2017. Giver ikke mening.
% \begin{opgave}[2]{Lorentz-transformationens udledelse}
% 	I \cref{k-rel:sec:trans} i kompendiet udledte vi Lorentz-transformationen. I ligning 3.15 så vi på højresiden, hvor vi konkluderede at $x$'s koefficient skulle være 1, og herved fandt frem til gamma-funktionen $\gamma$. Vi var dog ikke helt færdig med udledelsen i dette tilfælde.
% 	\opg I skal nu færdiggøre udledelsen af Lorentz-transformationen, ved at undersøge om højresiden af ligning 3.15 stemmer overens med ligning 3.11, når vi kender $\gamma$.
% \end{opgave}

\begin{opgave}[2]{Lorentztransformationen på differensform} \label{rel:opg:lorentz_diff}%
	Lad os betragte to begivenheder $P_1$ og $P_2$, som i inertialsystemet $S$ har koordinaterne $(x_1,y_1,z_1,t_1)$ og $(x_2,y_2,z_2,t_2)$. Svarende hertil har vi de fire koordinatdifferencer
	\begin{align}
		\Delta t=t_2-t_1, \	 \Delta x=x_2-x_1, \ \Delta y=y_2-y_1, \ \Delta z= z_2-z_1 \nonumber
	\end{align}
	\opg I skal nu finde de tilsvarende størrelser,
	\begin{math}
		\Delta t',  \Delta x',  \Delta y',  \Delta z'
	\end{math}
	i inertialsystemet $S'$, som bevæger sig i forhold til $S$ med hastigheden $v$, ved hjælp af Lorentz-transformationen.
\end{opgave}

\begin{opgave}[2]{Tidsforlængelse og Længdeforkortelse vha. Lorenz-transformationen}
	I denne opgave skal I prøve at udlede formlen for tidsforlængelse og længdeforkortelse vha. Lorentz-transformationen. Det gøres ved at kigge på en proces set fra to referencesystemer $S$ og $S'$ i standardkonfigurationen. Processens start og slutning i tid og rum beskrives ved $(x_1,t_1)$ og $(x_2,t_2)$ i $S$ og $(x'_1,t'_1)$ og $(x'_2,t'_2)$ i $S'$.
	\opg Udtrykket for $\Delta t'$ I fandt i \cref{rel:opg:lorentz_diff} indeholder både $t_1,t_2$ og $x_1,x_2$. Hvad må man kræve omkring processens start-- og slutkoordinater $x_1$ og $x_2$ i $S$, for at udtrykket bliver lig udtrykket for tidsforlængelse?
	\opg Forklar hvorfor kravet fra 1) sikre, at vi kigger på en "ren" tidsforlængelse, hvor rum og tid ikke bliver blandet sammen.
	\opg Forklar hvordan udtrykket for $\Delta t'$ fra \cref{rel:opg:lorentz_diff}, hvis man ikke bruger kravet fra 1), viser at rum og tid bliver blandet sammen i relativitetsteori.
	\opg Gennemgå de samme trin som I har gjort ovenfor, denne gang hvor I kigger på længdeforkortelse. Start derfor med $\Delta x'$, og undersøg hvad man må kræve omkring $t_1$ og $t_2$.
\end{opgave}

\begin{opgave}[2]{Cæsars død og Kristi fødsel}
	Cæsar blev myrdet år 44 f.Kr., og afstanden fra Rom til Betlehem kan sættes til \SI{2300}{\km}.
	\opg Findes der nogen iagttager, for hvem Cæsars død og Kristi fødsel er samtidige? Hvorfor/hvorfor ikke?
\end{opgave}

\begin{opgave}[2]{Samtidighed}
	To begivenheder har i inertialsystemet $S$ koordinaterne $(t_1,x_1,y_1,z_1)=(L/c,L,0,0)$ og $(t_2,x_2,y_2,z_2)=(L/2c,2L,0,0)$.
	\opg Der findes et inertialsystem, $S'$, i hvilket disse begivenheder er samtidige. Find hastigheden af $S'$ i forhold til $S$.
	\opg Hvad er den fælles tidskoordinat, $t'$, for disse begivenheder i $S'$?
\end{opgave}

\begin{opgave}[2]{En stangs hastighed}
	En stang med hvilelængde $l_0$ bevæger sig med jævn hastighed i sin længderetning. Set fra $S$ tager det tiden $\tau$ for stangen at passere et fast punkt i $S$. 
	\opg Find stangens hastighed som en brøkdel af lysets hastighed, $c$.
\end{opgave}

\begin{opgave}[2]{Invarians af lyspuls bevægelse}
	Et referencesystem $S'$ bevæger sig i $x$-retningen med hastigheden $v$ relativt til et andet referencesystem $S$. Til tiden $t'=t=0$ krydser de to referencesystemer hinanden (deres origo er samme sted), og i netop dette øjeblik udsendes en lyspuls fra origo i $S'$. Efter en tid $t'$ er lyspulsens afstand $x'$ i $S'$ givet ved $x'^2 = c^2 t'^2$. 
	\opg Vis at afstanden $x$ i $S$ er givet ved $x^2 = c^2 t^2$ (\emph{hint}: Brug Lorentztransformationerne). 
\end{opgave}

\begin{opgave}[3]{Referencesystemer -- samme sted og samme tid}
	To begivenheder er observeret i et referencesystem $S$ og kan beskrives ved $\left( x_1 , t_1 \right)$ og $\left( x_2 , t_2 \right)$. Et andet referencesystem $S'$ bevæger sig langs $x$-aksen med en hastighed $v$, således at de to begivenheder sker samme sted på $x$-aksen set fra $S'$.
	\opg Vis at tidsforskellen $\Delta t'$ mellem begivenhederne i $S'$ er givet ved:
	$$\Delta t' = \sqrt{\left( \Delta t \right)^2 - \left( \frac{\Delta x}{c} \right)^2}$$
	(\emph{hint}: Brug $x'_1 = x'_2$ og Lorentztransformationerne).
	\opg Brug ovenstående resultat til at vise, at såfremt $\Delta x > c \Delta t$, så vil der ikke eksistere et referencesystem $S'$, hvor begivenhederne sker samme sted.
	\opg Hvis $\Delta x > c \Delta t$, så findes der i stedet et andet referencesystem $S'$, hvor de to begivenheder sker samtidigt. Vis at afstanden $\Delta x'$ mellem de to begivenheder i dette referencesystem er givet ved:
	$$\Delta x' = \sqrt{\left( \Delta x \right)^2 - c^2 \left( \Delta t \right)^2}$$
	(\emph{hint}: Brug $t'_1 = t'_2$ og Lorentztransformationerne).
	\opg Brug ovenstående resultat til at vise, at såfremt $c \Delta t > \Delta x$, så vil der ikke eksistere et referencesystem $S'$, hvor begivenhederne sker samtidigt.
\end{opgave}


I de følgende to opgaver, vil det være nødvendigt at kigge på funktioner af to variable, og hvordan sådanne funktioner ændre sig, når begge variable ændre sig på samme tid. 
Forestil jer, at  vi har en funktion $z$, der afhænger både af $y$ og $x$. Dette skriver man typisk $z=f \left( x,y \right)$. Hvis man har en ændring $\Delta x$ samt en ændring $\Delta y$ kan man approksimere ændringen $\Delta z$ på følgende måde:
%
$$\Delta z \approx \dv{z}{x} \cdot \Delta x + \dv{z}{y} \cdot \Delta y$$
%
Såfremt man lader ændringerne $\Delta x$ og $\Delta y$ gå mod nul, vil ovenstående ikke længere være en approksimation, og man skriver at:
%
$$dz = \dv{z}{x} \cdot dx + \dv{z}{y} \cdot dy$$
%
hvor $dz$, $dy$ og $dx$ er infinitesimale (meget små) versioner af $\Delta z$, $\Delta y$ og $\Delta x$.\\

\begin{opgave}[2]{Lorentz-transformationen for hastighed}
	I denne opgave skal i prøve at udlede Lorentz-transformationen for hastighed. Dette kan gøres ved at finde et udtryk for $\dd{x}'$ og $\dd{t}'$, som det er beskrevet ovenfor. Da $\dd{x}'$ og $\dd{t}'$ vil være de øjeblikkelige ændringer i $x'$ og $t'$, må det derfor gælde at $v'_x = \dv*{x'}{t'}$.
\end{opgave}

\begin{opgave}[3]{Lorentz-transformationen for acceleration}
	Lad $S'$ være et referencesystem der bevæger sig med hastighed $v$ i forhold til et andet referencesystem $S$. Et objekt bevæger sig relativt i forhold til $S$ langs $x$-aksen med en øjeblikkelig hastighed $v_x$ og øjeblikkelig acceleration $a_x$. Målet med de følgende trin er at finde et udtryk for accelerationen $a'_x$ i referencesystemet $S'$.
	\opg Find et udtryk for $\dd{t'}$.
	\opg Find et udtryk for $\dd{v'_x}$.
	
	
	Før det sidste trin er det værd at sikre jer, at i har fået de rigtige resultater. I skulle gerne have fundet at:
	$$\dd{t'} = \gamma \left( \dd{t} - v \dd{x} / c^2 \right)$$
	$$\dd{v'_x} = \left( \frac{1 - v^2/c^2}{ \left( 1-vv_x/c^2 \right)^2 } \right) \dd{v_x}$$
	\opg Brug at $a'_x = dv'_x / dt'$, $a_x = dx/dt$ samt at $v_x = dx/dt$ til at vise, at $a'_x$ er givet ved:
	$$a'_x = a_x \left( 1- \frac{v^2}{c^2} \right)^{3/2} \left( 1 - \frac{vv_x}{c^2}  \right)^{-3}$$
\end{opgave}



\end{document}

\documentclass[crop=false, class=memoir]{standalone}
\documentclass[a4paper,hidelinks,12pt]{memoir}
\usepackage[utf8]{inputenc} % Do not change or remove!
\usepackage[T1]{fontenc} % Do not change or remove
\usepackage[danish]{babel} % Sproget, vi skriver på
\renewcommand\danishhyphenmins{22} % Kun hvis vi skriver på dansk

%%%%%%%%%%%%%%%%%%%%%%%%%%%%%%%%%%%%%%%%%%%%%%%%%%%%%
% Niels Jakob Søe Loft                              %
% nsl@phys.au.dk                                    %
%%%%%%%%%%%%%%%%%%%%%%%%%%%%%%%%%%%%%%%%%%%%%%%%%%%%%

% Denne skabelon er baseret på Rasmus Villemoes' veldokumenterede
% phd-afhandling i matematik, som jeg har ændret på, så den passer til
% et bachelorprojekt i fysik. Som hovedregel er ting kommenteret på
% engelsk fra Rasmus' skabelon, mens jeg har skrevet på dansk. De
% væsentligste ændringer er, at skabelonen er gjort mere egnet til et
% mindre projekt som et bachelorprojekt er i forhold til en
% phd-afhandling, hvorfor nogle ting er skåret væk, og jeg har
% inkluderet en liste fysik-relaterede makroer. Desuden er
% bibliografien konverteret fra BibTeX til BibLateX pr. marts 2014.

% Pr. 29. marts 2014 har jeg ændret skabelonen, så den kan bruges til
% kompendiet til UNFs Fysik Camp 2014.

%%%%%%%%%%%%%%
%% Generelt %%
%%%%%%%%%%%%%%
% ***************** UNF Science camp  kompendie ***************** %
% Dette dokument indeholder enviroments, comannds, makroer og
% layot specifikt til UNF science camp kompendier

% Pakker der anvendes. Kendte 'issues:
%	- xcolor skal loades før pdfpages, da den ellers loades uden dvipsnames
\usepackage[dvipsnames]{xcolor}		% Farver
\usepackage{xparse}							% Mere flexibel definition af makroer
\usepackage{marginnote}					% Noter i margen
\usepackage{forloop}						% Mulighed for forløkker



% ***************** Opgave enviroment ***************** %
% Sætter en opgave op og angiver sværhedsgraden. Opgavenummereringen nulstilles
% efter hvert ny kapitel.
% Anvenedelse: 
%		\begin{opgave}[farve]{Titel}{Sværhedgrad}
%			Introduktion
%			\opg
%			Delopgave 1
%			\opg
%			Delopgave 2
%			...
%		\end{opgave}
%
% Definer selve enviromentet. i´
\newcounter{opgave}[chapter]
\newcounter{delOpgave}[opgave]
\newenvironment{opgave}[3][NavyBlue]
	{\newcommand{\opg}{{{\refstepcounter{delOpgave}\smallskip\newline\textbf\thedelOpgave})\,}}
	\noindent\ignorespaces\refstepcounter{opgave}\newline\textbf{Opgave \theopgave:\,#3 #2}\newline}
	{\newline\bigskip}
% Definer 
%\newcommand{\lvl}[2][NavyBlue]{
%	\setcounter{nBullets}{#2}
%	\addtocounter{nBullets}{1}
%	\checkoddpage
%	\ifoddpages
%		\normalmarginpar
%		\marginnote{\textcolor{#1}{\lvltoken{\value{nBullets}}}}
%	\else
%		\reversemarginpar
%		\marginnote{\textcolor{#1}{\lvltoken{\value{nBullets}}}}
%	\fi
%}
\NewDocumentCommand{\lvl}{ O{NavyBlue} O{$ \bullet $} m}{
	\setcounter{nBullets}{#3}
	\addtocounter{nBullets}{1}
	\checkoddpage
	\ifoddpage
	\normalmarginpar
	{\textcolor{#1}{\lvltoken[#2]{\value{nBullets}}}}
	\else
	\reversemarginpar
	{\textcolor{#1}{\lvltoken[#2]{\value{nBullets}}}}
	\fi
}
\newcounter{lvl}
\newcounter{nBullets}
\newcommand{\lvltoken}[2][$ \bullet $]{
	\forloop{lvl}{1}{\value{lvl} < #2}{#1}} % load UNF-layout
\usepackage{graphicx} % Billeder
\usepackage{float}
\usepackage{epstopdf} % Så vi kan indsætte eps-filer
\usepackage{lipsum} % Dummytekst
\usepackage{pdfpages} % Indsættelse af pdf-sider
\usepackage{url} % Håndtering af URL'er
\usepackage{subfiles}
\usepackage{xspace} % Smarte mellemrum i egne makroer
\usepackage[final]{fixme} % Indsæt kommentarer i margin
%\usepackage{xstring} % Til sværhedsgrad-makro (se old/macros)
\usepackage[misc]{ifsym} % Til sværhedsgrad, skriv \Cube{n} hvor n=1,2,3
%\setcounter{secnumdepth}{3}
\setsecnumdepth{subsection}
\usepackage{newtxtext}
\usepackage{newtxmath}
\usepackage{subcaption} %sub-figurer
\usepackage{framed} % tekst-bokse
\usepackage{wrapfig}
\usepackage{enumitem}
\usepackage{microtype} % Mellemrumsjustering
\usepackage{xcolor} % flere farver
\usepackage{csquotes}%pæne citater
\usepackage{tikz} % tegninger i latex
\usepackage{empheq}
\usetikzlibrary{decorations.pathmorphing,patterns} % til tikz
\usetikzlibrary{calc}
\usetikzlibrary{decorations.pathmorphing}
\usetikzlibrary{decorations.markings}
\usetikzlibrary{positioning, shapes, snakes, arrows}
\tikzset{
fermion/.style={very thick,postaction={decorate},
  decoration={markings,mark=at position .6 with {\arrow[#1]{latex}}}},
boson/.style={very thick,dashed,postaction},
gluon/.style={thick,decorate,
 decoration={coil,amplitude=4pt, segment length=5pt,  pre length=.05cm, post length=.05cm}},
photon/.style={very thick,decorate, decoration={snake, segment length=8pt, amplitude=2pt, pre length=.05cm, post length=.05cm}},
}
\newcommand{\aq}[1]{$\bar{\mathrm{#1}}$}
\newcommand{\vertex}[1]{\fill (#1) circle (1 mm)}
%For at gøre det lettere at tegne Feynman diagrammer.

\interfootnotelinepenalty=10000 %undgår at fodnoter bliver spilittet op.

%\usepackage{cleveref}
%\creflabelformat{equation}{#2(#1)#3}
%\crefrangelabelformat{equation}{#3(#1)#4 to #5(#2)#6}
%\renewcommand{\ref}[1]{\eqref{#1}}
%\Crefname{equation}{ligning}{ligningerne}
%\Crefname{section}{afsnit}{afsnitene}
%\Crefname{table}{tabel}{tabellerne}

%% Bibliografi og referencer

%\usepackage{natbib} % Til biblografi, hvis man IKKE bruger BibLaTeX

%\usepackage[style=alphabetic,  % alternativt: style=numeric
%            backend=biber]{biblatex} % BibLaTeX, kræver installering
                                % af biber-pakken
%\addbibresource{kompendie.bib} % BibLaTeX tager referencer fra bach.bib

% \usepackage{cleveref} % Smarte referencer: skriv \cref{...} for småt forbogstav og \Cref{...} for stort forbogstav
% \crefname{equation}{ligning}{ligningerne}
% \Crefname{equation}{Ligning}{Ligningerne}
% \crefname{figure}{figur}{figurerne}
% \Crefname{figure}{Figur}{Figurerne}
% \crefname{table}{tabel}{tabellerne}
% \Crefname{table}{Tabel}{Tabellerne}
% \crefname{chapter}{kapitel}{kapitlerne}
% \Crefname{chapter}{Kapitel}{Kapitlerne}
% \crefname{section}{afsnit}{afsnittene}
% \Crefname{section}{Afsnit}{Afsnittene}

\usepackage[colorlinks=true, hidelinks]{hyperref} % Farvede links

% Glossary setup af Benjamin Muntz
\let\printglossary\relax 
\let\theglossary\relax
\let\endtheglossary\relax
%
\usepackage[toc,section=chapter]{glossaries}
\newglossary{symboler}{sym}{sbl}{Symbolliste}
\makeglossary
\newglossaryentry{Multiplicitet}{
    type=symboler,
    name={\ensuremath{\Omega}},
    sort=fnc,
    description={Multiplicitet}
}

%%%%%%%%%%%%%%%%%%%%%%
%% Tekst og formler %%
%%%%%%%%%%%%%%%%%%%%%%

%\usepackage[osf]{mathpazo} % Skrift

\usepackage{wasysym} % Font til smileys \smiley og \frownie

%\usepackage[sf]{libertine} % Til slanted skrift NJ's emacs er pigesur
\usepackage{libertine}

\linespread{1.06} % Større linjeafstand pga. font
\usepackage{fourier-orns} % Sjove symboler NJ's emacs er pigesur igen
\usepackage{textcomp} % Tilføjer flere tegn
\renewcommand\ttdefault{txtt} % Pænere teletype-skrift
\usepackage{physics}%En stor samling makroer
\renewcommand{\epsilon}{\varepsilon} %Skriver epsilon som varepsilon
\renewcommand{\varphi}{\phi} %Skriver varphi som phi
%Et par ekstra makroer
\newcommand{\xhat}{\vu x}
\newcommand{\yhat}{\vu y}
\newcommand{\zhat}{\vu z}
\newcommand{\xyz}[3]{\begin{pmatrix}#1\\#2\\#3\end{pmatrix}}
%\renewcommand{\Vec}[1]{\va{#1}}
\usepackage{mathtools} % Matematiktricks
\usepackage{cancel} % Ting der går ud med hinanden
\usepackage{siunitx} %SI-enheder
\sisetup{separate-uncertainty=true % gør at siunitx skriver +/- i
  % stedet for at bruge parentes til
  % at angive usikkerheder.
  ,output-decimal-marker={,}, % gør at der bruges komma til komma og
  % ikke punktum som i USA.
  ,load=abbr, % så vi kan bruge \keV
  ,exponent-product = \cdot, output-product = \cdot, % skift gangetegn fra \times til \cdot
}
%%% VI LAVER NOGLE FYSIK- OG MATEMATIK-MAKROER:


%% Generelt
%\newcommand{\g}{\cdot} % Prikprodukt, gangetegn
\newcommand{\subv}[2]{\gv{#1}_{\text{#2}}} % Pæn subscript til vektorer
\newcommand{\sub}[2]{#1_{\text{#2}}} % Pæn subscript til
\newcommand{\e}{\mathcal{E}} % Skrevet E
\newcommand{\abs}[1]{\left| #1 \right|} % Numerisk værdi
\newcommand{\N}{\ensuremath{\mathbb{N}}} % Naturlige tal
\newcommand{\Z}{\ensuremath{\mathbb{Z}}} % Hele tal
\newcommand{\Q}{\ensuremath{\mathbb{Q}}} % Rationelle tal
\newcommand{\R}{\ensuremath{\mathbb{R}}} % Reelle tal
\newcommand{\C}{\ensuremath{\mathbb{C}}} % Komplekse tal
\newcommand{\F}{\ensuremath{\mathbb{F}}} % Legeme tal
\newcommand{\A}{\ensuremath{\mathbb{A}}} % Algebraiske tal

%% Angiv sværhedsgrad til opgaver (benytter \usepackage{xstring})
%\newcommand{\lvl}[1]{%
%\IfStrEqCase{#1}{{1}{\ensuremath{\star}}
%    {2}{\ensuremath{\star\star}}
%    {3}{\ensuremath{\star\star\star}}}
%    [nada]
%}

%% Infinitesimalregning

\let\underdot=\d % omdøb indbygget kommando \d{} til \underdot{}
%\renewcommand{\d}[2]{\partial_{#2} \, #1} % afledt
%\newcommand{\dd}[2]{\partial_{#2}^2 \, #1} % dbl.afledt

%differentierings d
\renewcommand{\d}{\mathrm{d}}

%haard differentiering
\newcommand{\dif}[3][]{\frac{\d^{#1}{#3}}{{\d {#2}}^{#1}}}

%partiel differentiering
\newcommand{\pdif}[3][]{\frac{\partial^{#1}{#3}}{\partial {#2}^{#1}}}

\newcommand{\dt}[1]{\dot{#1}} % afledt mht. t (dot-notation)
\newcommand{\ddt}[1]{\ddot{#1}} % dbl.afledt mht. t (dbl.dot)

\newcommand{\integral}[4]{\int_{#3}^{#4} \, #1 \, \textrm{d}#2} % integrere



% Vektorer

\newcommand{\xyz}[3]{\begin{bmatrix} #1 \\ #2 \\ #3 \end{bmatrix}} %3D-vektor
\newcommand{\xy}[2]{\begin{bmatrix} #1 \\ #2 \end{bmatrix}} %2D-vektor
\let\vaccent=\v % Omdøb \v{} til \vaccent{}

\newcommand{\gv}[1]{{\vec{\mathbf{#1}}}} % Vektor med græske bogstaver
\renewcommand{\v}[1]{\gv{#1}} % Vektor med fed
\newcommand{\hatvec}[1]{\hat{\mathbf{#1}}} % Hatvektor
\newcommand{\ihat}{\boldsymbol{\hat{\textbf{\i}}}} % Enhedsvektor i
\newcommand{\jhat}{\boldsymbol{\hat{\textbf{\j}}}} % .. j
\newcommand{\khat}{\mathbf{\hat{k}}}  % .. k
\newcommand{\xhat}{\mathbf{\hat{x}}} % Enhedsvektor x
\newcommand{\yhat}{\mathbf{\hat{y}}} % .. y
\newcommand{\zhat}{\mathbf{\hat{z}}} % .. z
\newcommand{\grad}[1]{\gv{\nabla} #1} % Gradient
\let\divsymb=\div % Omdøb \div til \divsymb
\renewcommand{\div}[1]{\gv{\nabla} \cdot \v{#1}} % Divergens
\newcommand{\curl}[1]{\gv{\nabla} \times \v{#1}} % Curl
% Vil man tage div eller curl af græske bogstaver,
% skal man lade argumentetet være fx \gv{\mu} for µ-vektor

% Kvantemekanik

\newcommand{\op}[1]{\hat #1} % operator

\newcommand{\expect}[1]{\left< #1 \right>} % Forventningsværdi
\newcommand{\trace}{\ensuremath{\text{Tr}}\xspace}
\newcommand{\Hilbert}{\ensuremath{\mathcal{H}}}
\newcommand{\lag}{\ensuremath{{L}}}
\newcommand{\tr}[1]{\text{Tr}\left(#1\right)} % Trace
\newcommand{\ptr}[2]{\text{Tr}_{#1}\left(#2\right)} % Partial trace
\newcommand{\ket}[1]{\left| #1 \right>} % Dirac-notation: ket
\newcommand{\bra}[1]{\left< #1 \right|} % bra
\newcommand{\braket}[2]{\left< #1 \vphantom{#2} \, \right|
  \left. \! #2 \vphantom{#1} \right>} % bracket
\newcommand{\matrixel}[3]{\left< #1 \vphantom{#2#3} \right|
  #2 \left| #3 \vphantom{#1#2} \right>} % Bracket med ekstra streg
 % En masse matematik- og fysikmakroer

%%%%%%%%%%%%
%% Layout %%
%%%%%%%%%%%%

%\newcommand{\anonbreak}{\fancybreak{$* * *$}} % Break med stjerner
%\let\bar\overline % Gør at en bar over et symbol kan skalere efter symbolet

%% Sidehoved- og fod

\makepagestyle{tket}
\makeevenfoot{tket}{\thepage}{}{}
\makeoddfoot{tket}{}{}{\thepage}
\makeevenfoot{plain}{\thepage}{}{}
\makeoddfoot{plain}{}{}{\thepage}
\makeevenhead{tket}{\leftmark}{}{}


%% Margin

% Man kan sætte margins ved enten at specificere marginstørrelsen
% eller ved at specificere tekstblokken. Man skal vælge én og kun én
% af mulighederne.

% Specificer marginstørrelsen
%\setulmarginsandblock{2.7cm}{*}{1}
%\setlrmarginsandblock{1.6cm}{1.6cm}{*} 
%\setlength{\oddsidemargin}{-1cm} % Giver mere plads på siden
%\setlength{\topmargin}{-1.2cm} % Gør topmargin behagelig at se på
%\setlength{\columnsep}{1.5\columnsep}  % Afstand mellem søjlerne


\setlrmarginsandblock{2.5cm}{2.5cm}{*}

\usepackage[font={small,it}]{caption}	% Italic captions

% Tekstblok: Følgende er fra Rasmus Villemoes' thesis-layout.tex
%\setlxvchars[\normalfont] % standardbredden af tekstblok er ca. 65 tegn
%\settypeblocksize{*}{1.2\lxvchars}{1.61803} % højde, bredde, forhold
%\setulmargins{*}{*}{1.3} % lav bundmargin lidt større end topmargin
\checkandfixthelayout % memoir tjekker, at alt er ok og konsistent

\usepackage{ctable} % Tillader fede linjer i tabeller

%%%%%%%%%%%%%%%%%
%% Bibliografi %%
%%%%%%%%%%%%%%%%%

\usepackage[style=ieee]{biblatex}
\addbibresource{litteratur.bib}

%%%%%%%%%%%%%%%%%%
%% Definitioner %%
%%%%%%%%%%%%%%%%%%

% Definer titlen på projektet
 \newcommand{\thesistitle}{Kompendie til UNF Fysik Camp 2019}

%%%%%%%%%%%%%%%%%%%%%%
%% Slut på preamble %%
%%%%%%%%%%%%%%%%%%%%%%
\usepackage{import}
\begin{document}
\chapter{Elektromagnetisme} \label{chap:elektro_opg}
\subsection*{Elektrostatik}
\begin{opgave}[1]{Coulombkraften}
    To ladninger, $q$ og $Q$ har en indbyrdes afstand $r$ til hinanden.
    \opg Bestem Coulombkraften fra den ene ladning på den anden i hvert af følgende tilfælde:
    \begin{enumerate}
        \item $r = \SI{1,00}{\metre}$, $q = \SI{1,00}{\coulomb}$ og $Q = \SI{1,00}{\coulomb}$
        \item $r = \SI{1,00}{\metre}$, $q = \SI{1,00}{\coulomb}$ og $Q = \SI{-1,00}{\coulomb}$
        \item $r = \SI{1,00}{\metre}$, $q = \SI{1,00}{\coulomb}$ og $Q = \SI{2,00}{\coulomb}$
        \item $r = \SI{2,00}{\metre}$, $q = \SI{1,00}{\coulomb}$ og $Q = \SI{1,00}{\coulomb}$
    \end{enumerate}
    \opg Beskriv med ord hvad der sker med Coulombkraften når den ene ladning skifter fortegn, dens størrelse fordobles og når ladningernes indbyrdes afstand fordobles.
    \opg Brug Newtons anden lov, $\va{F}=m\va{a}$, til at bestemme accelerationen af hver ladning i tilfælde 1. Ladningernes masse er $m_1 = \SI{1,00}{\kilo\gram}$ og $m_2 = \SI{2,00}{\kilo\gram}$.
    \opg Bestem accelerationen af ladning 1 i tilfælde 2-4.
    \opg Beskriv med ord hvad der sker når ladningerne frigives i de fire tilfælde og sammenlign dem.
\end{opgave}

\begin{opgave}[2]{Tre ladninger på en linje}
    To punktladninger med ladning $+q$ befinder sig ved positioner $x=-a$ og $x=a$.
    \opg Bestem kraften på en anden ladning $+Q$ som funktion af dens position langs $x$-aksen.
    \opg Tegn en graf over kraften som funktion af position, og kommenter den generelle opførsel.
    \opg Hvordan vil ladningen bevæge sig når den er ved $x=0$?
    \opg For $x\gg a$, approksimer udtrykket for kraften. \textit{Hint: Hvad sker der hvis man lægger et lille tal til et stort tal?}
\end{opgave}

\begin{opgave}[1]{Firkantet ladningskonfiguration}
Betragt fire punktladninger sat i et kvadratisk mønster. Vi vil nu undersøge hvilken retning en testladning $Q$ skubbes. Tegn retningen som $Q$ vil bevæge sig i de tre tilfælde vist i \cref{em:fig:ladning_i_firkant}.
\begin{figure}[]
    \centering
    \includegraphics[width=.9\textwidth]{Elektro/Elekfig/firkant.JPG}
    \caption{Tre forskellige positioner for Q.}
    \label{em:fig:ladning_i_firkant}
\end{figure}
\end{opgave}

\begin{opgave}[2]{Fluxopgave I} \label{em:opg:flux1}%
Betragt en uendelig lang linjeladning $\lambda$. Linjeladningen udspænder E-feltet $\va{E}=\lambda/(2\pi r \epsilon_0)\vu{s}$, hvor $\vu{s}$ er en enhedsvektor, der peger vinkelret væk fra linjen (og variablen $r$, er også den vinkelrette distance væk fra linjen). For illustration af Gaussoverfladen se \cref{em:fig:flux1}.
\opg Bestem fluxen igennem en cylinder Gaussoverflade med radius $R$ og længde $L$. \emph{Hint: Er E-feltet konstant langs noget af overfladen?}
\opg Bestem dernæst fluxen igennem en cylinder Gaussoverflade med radius $2R$ og længde $L$
\opg Bestem tilsidst fluxen igennem en cylinder Gaussoverflade med radius $R$ og længde $2L$
\opg Forklar hvorfor fluxen er større i det ene tilfælde og uændret i det andet tilfælde.
    \begin{figure}[]
        \centering
        \includegraphics[width=.52\textwidth]{Elektro/Elekfig/cylinder.PNG}
        \caption{Cylinderen til spørgsmål 1 og 2 i \cref{em:opg:flux1}.}
        \label{em:fig:flux1}
    \end{figure}
\end{opgave}

\begin{opgave}[2]{Fluxopgave II}
Betragt to parallele plader, der hver har en konstant overfladeladningstæthed $\sigma$. En enkelt plade udspænder E-feltet $\va{E}=(\sigma/2)\vu{n}$.
\opg Bestem E-feltet under, imellem og over de to plader.

\opg Vi vil nu prøve at regne fluxen på dette E-felt. Brug kassen markeret 1) i \cref{em:fig:flux2} som  Gaussoverflade og bestem fluxen.
\opg Brug kassen markeret 2) i \cref{em:fig:flux2} som  Gaussoverflade og bestem fluxen.
\opg Forklar hvorfor fluxen er dobbelt så stor i 2) end i 1).
    \begin{figure}[]
        \centering
        \includegraphics[width=\textwidth]{Elektro/Elekfig/tokapacitor.JPG}
        \caption{To forskellige Gaussoverflader.}
        \label{em:fig:flux2}
    \end{figure}
\end{opgave}

\begin{opgave}[3]{Uendelig lang ladet linje}
    For at beregne det elektriske felt overalt i rummet fra en uendelig lang ladet linje med konstant linjeladning $\lambda$ bruges Gauss' lov.
    \opg Argumenter for at det elektriske felt peger direkte væk fra linjen overalt i rummet. (Anvend hvad vi har lært omkring en linjeladning i 2D)
    \opg Argumenter for at E-feltet er rotationssymmetrisk omkring linjeladningen.
    \opg Hvis $\vu{s}$ er en vektor, som peger vinkelret væk fra linjeladningen, så er E-feltet $\va{E}=|E|\vu{s}$. Anvend dette og rotationssymmetrien til at argumentere for, at en cylinder Gaussoverflade omkring linjeladningen er en god ide. 
    \opg Giv din cylinder radius r og længde L. Det lukkede integral i Gauss' lov $\oint\va{E}\cdot\dd{\va{a}}$ kan skrives som en sum af hver de 3 overflade en cylinder har. Lad $S_1$ angive den krumme overflade, og henholdsvis $S_2$ og $S_3$ angive de to endestykker. Løs integralet
    \begin{align*}
        \oint_S \va E \cdot \dd{\va a} = \int_{S_1} \va E \cdot \dd{\va a} + \int_{S_2} \va E \cdot \dd{\va a} + \int_{S_3} \va E \cdot \dd{\va a}.
    \end{align*}
    \opg Brug nu Gauss' lov til at bestemme det elektriske felt overalt i rummet. \\
    \emph{Hint}: Længden af cylinderen skulle gerne gå ud, så det endelige udtryk kun afhænger af $\lambda$, $r$ og nogle konstanter.
\end{opgave}

\begin{opgave}[3]{Ladet kugle}
    En solid ladet kugle med radius $R$ har en konstant volumeladning $\rho$ for $r<R$ og $\rho=0$ for $r>R$, hvor $r$ er afstanden fra kuglens centrum.
    \opg Anvend hvad vi ved om en sfære overfladningsladning til at redegøre for hvilken vej E-feltet peger udenfor og indenfor kuglen.
    \opg Brug Gauss' lov til at bestemme det elektriske felt $\va{E}$ for $r>R$.
    \opg Brug Gauss' lov til at bestemme det elektriske felt $\va{E}$ for $r<R$. \emph{Hint}: Det er vigtigt, at huske at $Q_{enc}$ ikke er konstant inde i kuglen.
\end{opgave}

\begin{opgave}[3]{Cylinder ladningskonfigurationer}
    En uendelig lang hul cylinder med radius $R$ har en konstant overfladeladning $\sigma$.
    \opg Split overfladen af cylinderen op i uendelig lange linjeladninger og argumenter for retningen af E-feltet udenfor og inden i cylinderen. \emph{Hint: Det kan hjælpe at kigge 2-dimensionalt på cylinderen (altså en cirkel) og se hvert punkt på cirklen som en uendelig lang linjeladning}.
    \opg Anvend, at cylinderen er rotationssymmetrisk og bestem E-feltet for $r>R$.
    \opg Bestem dernæst E-feltet for $r<R$.
    \newline
    Dernæst vil vi kigge på en uendelig lang solid cylinder med radius R, der har en konstant volumeladning $\rho$.
    \opg Split den solide cylinder op i cylinder-overflader og argumenter for E-feltets retning udenfor og inden i den solide cylinder.
    \opg Anvend rotationssymmetri og bestem E-feltet for $r>R$.
    \opg Bestem til sidst E-feltet for $r>R$. \emph{Hint}: $Q_\textup{inde}$ er ikke konstant inde i den solide cylinder.
\end{opgave}

\begin{opgave}[4]{Endelig linjeladning}
I denne opgave vil vi prøve at kaste os over de ladningstæthedsintegraler, der kræver at man kender til integration. Betragt en konstant linjeladning $\lambda$ langs x-aksen med en længde på L, sådan at linjeladningens venstre ende er i origo. Vi vil nu finde E-feltet i en position $(y,0)$. Se figur \ref{endeliglinje}.
\opg Opskriv $\va{R}$.
\opg Opskriv nu $\vu{R}$. \emph{Hint}: En enhedsvektor kan altid skrives som $\vu{n}=\va{n}/|n|$.
\opg Opskriv integralet for E-feltet.
\opg Integralet kan deles op i to (i $\vu{x}$ og $\vu{y}$ retningen). Løsningen af det ene integrale er lidt nemmere, hvis man kan ens integrations tricks. Anvend integration af substitution til at bestemme integralet langs x-aksen.
\opg Det andet integrale er mere indviklet. Anvend substitutionen $x=y\tan({u})$, og at $1+\tan({u})^2=1/cos({u})^2$ og $\sin({\tan^{-1}(L/y)})=L/(\sqrt{L^2+y^2})$ til at bestemme integralet langs z-aksen. Hvis man har løst begge integraler, så har man fundet et udtryk for E-feltet: tillykke.
    \begin{figure}
        \centering
        \includegraphics[scale=0.5]{Elektro/Elekfig/endeliglinje.JPG}
        \caption{}
        \label{endeliglinje}
    \end{figure}
\end{opgave}


\subsection*{Magnetisme}
\begin{opgave}[3]{Solenoiden}
    En måde at fremstille et (approksimativt) uniformt magnetfelt er ved brug af en solenoide, som illustreret i figur \ref{fig:solenoide}, hvori en strøm med strømstyrken $I$ løber. Det antages at solenoiden befinder sig i vakuum, at den er meget lang, og at antallet af  vindinger pr. længde, $n$, er så stor at solenoiden kan beskrives som ringe af strøm, der ligger helt tæt op ad hinanden. Der vælges en Ampereløkke, som den i figur \ref{fig:solenoide_ampere}, hvor punkterne $a$ og $b$ placeres midt i solenoiden og punkterne $c$ og $d$ placeres uendeligt langt væk fra solenoiden. Bredden af løkken kaldes $L$ og hele løkken er placeret ved midten af solenoiden.
    \opg Brug højrehåndsreglen til at bestemme retningen på det magnetiske felt midt i solenoiden.
    \opg På baggrund af antagelserne er størrelsen af det magnetiske felt på hele linjestykket $cd$ 0. Forklar hvorfor det er tilfældet.
    \opg Argumenter for at magnetfeltet udenfor solenoiden omkring Ampereløkken er parallelt med feltet indeni solenoiden.
    \opg Brug ovenstående argumenter til at vise at
    \begin{align*}
        \oint_P \va B \cdot \dd{\va l} = BL.
    \end{align*}
    \opg Vis at den indesluttede strøm er
    \begin{align*}
        I_\textup{inde} = nLI.
    \end{align*}
    \opg Brug Amperes lov til at vise at magnetfeltet inde i midten af solenoiden er
    \begin{align} \label{eq:b_solenoide}
        B = \mu_0nI.
    \end{align}
    \opg Argumenter, ud fra antagelserne, for at magnetfeltet er uniformt omkring midten af solenoiden med feltstyrken givet i ligning \ref{eq:b_solenoide}.

\begin{figure}
    \centering
    \begin{subfigure}[t]{.47\textwidth}
        \includegraphics[width=\columnwidth]{Elektro/Elekfig/Solenoid-1.png}
        \caption{Illustration af en solenoide. Kilde: \cite{SolenoidWikipedia2019}}
        \label{fig:solenoide}
    \end{subfigure}
    %
    \hfill
    %
    \begin{subfigure}[t]{.47\textwidth}
        \includegraphics[width=\columnwidth]{Elektro/Elekfig/solenoid.png}
        \caption{Tegning af en solenoide med $n$ vindinger pr. længde, hvori der løber en strøm med strømstyrke $I$. Kilde: \cite{UY1ApplicationsAmpere2015}}
        \label{fig:solenoide_ampere}
    \end{subfigure}
    \caption{ }
\end{figure}
\end{opgave}
\begin{opgave}[3]{Lang lige leder med udstrækning}
    I Magnetostatikafsnittet blev det magnetiske felt fra en uendelig tynd lang lige leder med strømstyrke $I$ bestemt. Situationen ændrer sig en smule, hvis lederen har en endelig tykkelse $R$, og strømstyrken uniformt fordelt over tværsnittet.
    \opg Argumenter for at udregningerne ikke ændrer sig i en afstand $r>R$, hvorfor magnetfeltet er det samme.
    \opg Argumenter for at integralet i Amperes lov er det samme indenfor og udenfor lederen:
    %
    \begin{align}
        \oint \va B \cdot \dd \va l = 2\pi B r,
    \end{align}
    %
    hvor $r$ er afstanden lederens centrum.
    \opg Vis at den indesluttede strøm i tilfældet $r<R$ er
    %
    \begin{align}
        I_\textup{inde} = I \frac{r^2}{R^2}.
    \end{align}
    %
    \opg Brug dette til at vise at
    %
    \begin{align}
        B = \begin{cases}
            \mu_0I/2\pi r, \quad &r>R \\
            \mu_0Ir/2\pi R^2, \quad &r<R
        \end{cases}
        .
    \end{align}
    %
    \opg Skitser funktionen $B(r)$.
    \opg Argumenter kvalitativt for hvordan funktionen $B(r)$ ville se ud hvis lederen var hul med en indre radius $R_0$ og skitser funktionen.
\end{opgave}

\end{document}
\subsection*{Logaritmer}
%%
\begin{opgave}[1]{Logaritmer og tal}
    Find værdien af de følgende udtryk
    \opg $\log_{4}(8)$
    \opg $\log_{1/9}\left(\sqrt{27}\right)$
    \opg $\ln(e^{2/3})$
    %\opg $\ln(\dfrac{e^5}{e^3})$
\end{opgave}
%%
%%
\begin{opgave}[2]{Ligningsløsning med logaritmer}
Find talværdien for den ukendte variabel, således at ligningen er sand
    \opg $\log_{b}(16) = \dfrac{4}{3}$
    \opg $\ln(x) = -1$
    \opg $\log_{2}(1/x)=\dfrac{1}{5}$
\end{opgave}
%%
\begin{opgave}[3]{En ekstra logaritmeregneregel}
    Ud fra de tre logaritmeregneregler, \cref{k-mat:eq:log} i kompendiet, vis at
    $$
    -c\log_a(b)=\log_a\left(\frac{1}{b^c}\right).
    $$
\end{opgave}
%%
\begin{opgave}[4]{Eksponentregneregler}
    Bevis \cref{k-mat:eq:exp_regel2,k-mat:eq:exp_regel3} i kompendiet ud fra \cref{k-mat:eq:exp_regel1} i kompendiet. 
\end{opgave}
%%
\begin{opgave}[4]{Logaritmeregneregel}
    Bevis logaritmeregnereglerne, \cref{k-mat:eq:log} i kompendiet, ud fra definitionen af logaritmen, \cref{k-mat:eq:log_def} og eksponentregnereglerne, \cref{k-mat:eq:exp_regel} i kompendiet.
\end{opgave}
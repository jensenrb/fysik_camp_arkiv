\subsection*{Trigonometri}
%%
\begin{opgave}[1]{Retvinklede trekanter}
\opg Det vides, at $\cos(\theta)$ er givet som længden af den hosliggende katete divideret med længden af hypotesen. Det giver her at
\begin{align*}
\cos(\theta_{ac}) = \frac{a}{c} = \frac{5}{13} \quad \implies \quad \theta_{ac} = \cos^{-1} \left( \frac{5}{13} \right) \approx 1,176  \approx 67,38^\circ \, .
\end{align*}
\opg Ideen er her, at man for trekanten i denne opgave kan skrive $\tan(\theta_{ac}) = b/a$. Da $\theta_{ac}$ kendes fra 1) og $a$ er opgivet i opgaven, kan man nu finde $b$. Det giver
\begin{align*}
b = a \tan(\theta_{ac}) = 5 \cdot \tan \left( \cos^{-1} \left( \frac{5}{13}  \right) \right) = 12 \, .
\end{align*}
\opg Det vides, at $\sin(\theta)$ er givet som længden af den modstående katete divideret med længden af hypotesen. Det giver her at
\begin{align*}
\sin(\theta_{bc}) = \frac{a}{c} = \frac{5}{13} \quad \implies \quad \theta = \sin^{-1} \left( \frac{5}{13} \right) \approx 0,395 \approx 22,62 ^\circ \, . 
\end{align*}
\end{opgave}
%%
%%
\begin{opgave}[1]{Tangens}
\opg Fra definitionen af tangens har man, at $\tan(\theta) = \sin(\theta)/\cos(\theta)$. Men i matematikafsnittet er det også introduceret, at $\cos(\theta)$ er givet som længden af den hosliggende katete divideret med længden af hypotenusen, og at $\sin(\theta)$ er givet som længden af den modstående katete divideret med længden af hypotenusen. For trekanten i denne opgave har man altså 
\begin{align*}
\cos(\theta) = \frac{a}{c} \quad \text{og} \quad \sin(\theta) = \frac{b}{c}  \quad \implies  \quad \tan(\theta) = \frac{\sin(\theta)}{\cos(\theta)} = \frac{b/c}{a/c} = \frac{b}{a} \, . 
\end{align*}
\opg Man finder at: $\quad \tan(\theta) = 1 \quad \implies \quad \theta =  \tan^{-1}(1) = \frac{\pi}{4} = 45^\circ$.
\end{opgave}
%%
%%
\begin{opgave}[1]{Koordinatskift}
Her bruges formlerne i \cref{k-mat:eq:kartesisk/polaer} fra kompendiet, altså $r = \sqrt{x^2 + y^2}$ og $\theta = \tan^{-1}\left(y/x\right)$.
\opg Man får
\begin{align*}
r = \sqrt{2^2+2^2} = \sqrt{8} \, , \quad \theta = \tan^{-1} \left( \frac{2}{2} \right) = \frac{\pi}{4} = 45 ^\circ \, .
\end{align*} 
\opg Man får
\begin{align*}
r = \sqrt{1^2 + (-2)^2} = \sqrt{5} \, , \quad \theta = \tan^{-1} \left( \frac{-2}{1} \right) \approx -1,107 \approx -63,43^\circ \, .
\end{align*}
Det positive svar findes ved at lægge $2\pi$ radianer eller $360^\circ$ til det negative resultat.
\end{opgave}
%%
%%
\begin{opgave}[1]{Koordinatskift 2}
Her bruges formlerne i \cref{k-mat:eq:kartesisk/polaer} fra kompendiet, altså $x = r \cos(\theta)$ og $y = r \sin(\theta)$.
\opg Man får
\begin{align*}
x = 5 \cdot \cos\left( \frac{\pi}{4} \right) = \frac{5}{\sqrt{2}} \, , \quad y = 5 \cdot \sin\left(\frac{\pi}{4}\right) = \frac{5}{\sqrt{2}} \, .
\end{align*}
\opg Man får
\begin{align*}
x = 5 \cdot \cos\left(\frac{7\pi}{4}\right) =  \frac{5}{\sqrt{2}} \, , \quad y = 5 \cdot \sin\left(\frac{7\pi}{4}\right) = - \frac{5}{\sqrt{2}} \, .
\end{align*}
\opg Man får
\begin{align*}
x = 13 \cdot \cos\left(\frac{11\pi}{6}\right) = 13 \cdot \frac{\sqrt{3}}{2} \, , \quad y = 13 \cdot \sin\left(\frac{11\pi}{6}\right) =  -13 \cdot \frac{1}{2} \, .
\end{align*}
\end{opgave}
%%

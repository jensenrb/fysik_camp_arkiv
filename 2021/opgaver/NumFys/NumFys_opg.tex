\documentclass[crop=false, class=memoir]{standalone}
\documentclass[a4paper,hidelinks,12pt]{memoir}
\usepackage[utf8]{inputenc} % Do not change or remove!
\usepackage[T1]{fontenc} % Do not change or remove
\usepackage[danish]{babel} % Sproget, vi skriver på
\renewcommand\danishhyphenmins{22} % Kun hvis vi skriver på dansk

%%%%%%%%%%%%%%%%%%%%%%%%%%%%%%%%%%%%%%%%%%%%%%%%%%%%%
% Niels Jakob Søe Loft                              %
% nsl@phys.au.dk                                    %
%%%%%%%%%%%%%%%%%%%%%%%%%%%%%%%%%%%%%%%%%%%%%%%%%%%%%

% Denne skabelon er baseret på Rasmus Villemoes' veldokumenterede
% phd-afhandling i matematik, som jeg har ændret på, så den passer til
% et bachelorprojekt i fysik. Som hovedregel er ting kommenteret på
% engelsk fra Rasmus' skabelon, mens jeg har skrevet på dansk. De
% væsentligste ændringer er, at skabelonen er gjort mere egnet til et
% mindre projekt som et bachelorprojekt er i forhold til en
% phd-afhandling, hvorfor nogle ting er skåret væk, og jeg har
% inkluderet en liste fysik-relaterede makroer. Desuden er
% bibliografien konverteret fra BibTeX til BibLateX pr. marts 2014.

% Pr. 29. marts 2014 har jeg ændret skabelonen, så den kan bruges til
% kompendiet til UNFs Fysik Camp 2014.

%%%%%%%%%%%%%%
%% Generelt %%
%%%%%%%%%%%%%%
% ***************** UNF Science camp  kompendie ***************** %
% Dette dokument indeholder enviroments, comannds, makroer og
% layot specifikt til UNF science camp kompendier

% Pakker der anvendes. Kendte 'issues:
%	- xcolor skal loades før pdfpages, da den ellers loades uden dvipsnames
\usepackage[dvipsnames]{xcolor}		% Farver
\usepackage{xparse}							% Mere flexibel definition af makroer
\usepackage{marginnote}					% Noter i margen
\usepackage{forloop}						% Mulighed for forløkker



% ***************** Opgave enviroment ***************** %
% Sætter en opgave op og angiver sværhedsgraden. Opgavenummereringen nulstilles
% efter hvert ny kapitel.
% Anvenedelse: 
%		\begin{opgave}[farve]{Titel}{Sværhedgrad}
%			Introduktion
%			\opg
%			Delopgave 1
%			\opg
%			Delopgave 2
%			...
%		\end{opgave}
%
% Definer selve enviromentet. i´
\newcounter{opgave}[chapter]
\newcounter{delOpgave}[opgave]
\newenvironment{opgave}[3][NavyBlue]
	{\newcommand{\opg}{{{\refstepcounter{delOpgave}\smallskip\newline\textbf\thedelOpgave})\,}}
	\noindent\ignorespaces\refstepcounter{opgave}\newline\textbf{Opgave \theopgave:\,#3 #2}\newline}
	{\newline\bigskip}
% Definer 
%\newcommand{\lvl}[2][NavyBlue]{
%	\setcounter{nBullets}{#2}
%	\addtocounter{nBullets}{1}
%	\checkoddpage
%	\ifoddpages
%		\normalmarginpar
%		\marginnote{\textcolor{#1}{\lvltoken{\value{nBullets}}}}
%	\else
%		\reversemarginpar
%		\marginnote{\textcolor{#1}{\lvltoken{\value{nBullets}}}}
%	\fi
%}
\NewDocumentCommand{\lvl}{ O{NavyBlue} O{$ \bullet $} m}{
	\setcounter{nBullets}{#3}
	\addtocounter{nBullets}{1}
	\checkoddpage
	\ifoddpage
	\normalmarginpar
	{\textcolor{#1}{\lvltoken[#2]{\value{nBullets}}}}
	\else
	\reversemarginpar
	{\textcolor{#1}{\lvltoken[#2]{\value{nBullets}}}}
	\fi
}
\newcounter{lvl}
\newcounter{nBullets}
\newcommand{\lvltoken}[2][$ \bullet $]{
	\forloop{lvl}{1}{\value{lvl} < #2}{#1}} % load UNF-layout
\usepackage{graphicx} % Billeder
\usepackage{float}
\usepackage{epstopdf} % Så vi kan indsætte eps-filer
\usepackage{lipsum} % Dummytekst
\usepackage{pdfpages} % Indsættelse af pdf-sider
\usepackage{url} % Håndtering af URL'er
\usepackage{subfiles}
\usepackage{xspace} % Smarte mellemrum i egne makroer
\usepackage[final]{fixme} % Indsæt kommentarer i margin
%\usepackage{xstring} % Til sværhedsgrad-makro (se old/macros)
\usepackage[misc]{ifsym} % Til sværhedsgrad, skriv \Cube{n} hvor n=1,2,3
%\setcounter{secnumdepth}{3}
\setsecnumdepth{subsection}
\usepackage{newtxtext}
\usepackage{newtxmath}
\usepackage{subcaption} %sub-figurer
\usepackage{framed} % tekst-bokse
\usepackage{wrapfig}
\usepackage{enumitem}
\usepackage{microtype} % Mellemrumsjustering
\usepackage{xcolor} % flere farver
\usepackage{csquotes}%pæne citater
\usepackage{tikz} % tegninger i latex
\usepackage{empheq}
\usetikzlibrary{decorations.pathmorphing,patterns} % til tikz
\usetikzlibrary{calc}
\usetikzlibrary{decorations.pathmorphing}
\usetikzlibrary{decorations.markings}
\usetikzlibrary{positioning, shapes, snakes, arrows}
\tikzset{
fermion/.style={very thick,postaction={decorate},
  decoration={markings,mark=at position .6 with {\arrow[#1]{latex}}}},
boson/.style={very thick,dashed,postaction},
gluon/.style={thick,decorate,
 decoration={coil,amplitude=4pt, segment length=5pt,  pre length=.05cm, post length=.05cm}},
photon/.style={very thick,decorate, decoration={snake, segment length=8pt, amplitude=2pt, pre length=.05cm, post length=.05cm}},
}
\newcommand{\aq}[1]{$\bar{\mathrm{#1}}$}
\newcommand{\vertex}[1]{\fill (#1) circle (1 mm)}
%For at gøre det lettere at tegne Feynman diagrammer.

\interfootnotelinepenalty=10000 %undgår at fodnoter bliver spilittet op.

%\usepackage{cleveref}
%\creflabelformat{equation}{#2(#1)#3}
%\crefrangelabelformat{equation}{#3(#1)#4 to #5(#2)#6}
%\renewcommand{\ref}[1]{\eqref{#1}}
%\Crefname{equation}{ligning}{ligningerne}
%\Crefname{section}{afsnit}{afsnitene}
%\Crefname{table}{tabel}{tabellerne}

%% Bibliografi og referencer

%\usepackage{natbib} % Til biblografi, hvis man IKKE bruger BibLaTeX

%\usepackage[style=alphabetic,  % alternativt: style=numeric
%            backend=biber]{biblatex} % BibLaTeX, kræver installering
                                % af biber-pakken
%\addbibresource{kompendie.bib} % BibLaTeX tager referencer fra bach.bib

% \usepackage{cleveref} % Smarte referencer: skriv \cref{...} for småt forbogstav og \Cref{...} for stort forbogstav
% \crefname{equation}{ligning}{ligningerne}
% \Crefname{equation}{Ligning}{Ligningerne}
% \crefname{figure}{figur}{figurerne}
% \Crefname{figure}{Figur}{Figurerne}
% \crefname{table}{tabel}{tabellerne}
% \Crefname{table}{Tabel}{Tabellerne}
% \crefname{chapter}{kapitel}{kapitlerne}
% \Crefname{chapter}{Kapitel}{Kapitlerne}
% \crefname{section}{afsnit}{afsnittene}
% \Crefname{section}{Afsnit}{Afsnittene}

\usepackage[colorlinks=true, hidelinks]{hyperref} % Farvede links

% Glossary setup af Benjamin Muntz
\let\printglossary\relax 
\let\theglossary\relax
\let\endtheglossary\relax
%
\usepackage[toc,section=chapter]{glossaries}
\newglossary{symboler}{sym}{sbl}{Symbolliste}
\makeglossary
\newglossaryentry{Multiplicitet}{
    type=symboler,
    name={\ensuremath{\Omega}},
    sort=fnc,
    description={Multiplicitet}
}

%%%%%%%%%%%%%%%%%%%%%%
%% Tekst og formler %%
%%%%%%%%%%%%%%%%%%%%%%

%\usepackage[osf]{mathpazo} % Skrift

\usepackage{wasysym} % Font til smileys \smiley og \frownie

%\usepackage[sf]{libertine} % Til slanted skrift NJ's emacs er pigesur
\usepackage{libertine}

\linespread{1.06} % Større linjeafstand pga. font
\usepackage{fourier-orns} % Sjove symboler NJ's emacs er pigesur igen
\usepackage{textcomp} % Tilføjer flere tegn
\renewcommand\ttdefault{txtt} % Pænere teletype-skrift
\usepackage{physics}%En stor samling makroer
\renewcommand{\epsilon}{\varepsilon} %Skriver epsilon som varepsilon
\renewcommand{\varphi}{\phi} %Skriver varphi som phi
%Et par ekstra makroer
\newcommand{\xhat}{\vu x}
\newcommand{\yhat}{\vu y}
\newcommand{\zhat}{\vu z}
\newcommand{\xyz}[3]{\begin{pmatrix}#1\\#2\\#3\end{pmatrix}}
%\renewcommand{\Vec}[1]{\va{#1}}
\usepackage{mathtools} % Matematiktricks
\usepackage{cancel} % Ting der går ud med hinanden
\usepackage{siunitx} %SI-enheder
\sisetup{separate-uncertainty=true % gør at siunitx skriver +/- i
  % stedet for at bruge parentes til
  % at angive usikkerheder.
  ,output-decimal-marker={,}, % gør at der bruges komma til komma og
  % ikke punktum som i USA.
  ,load=abbr, % så vi kan bruge \keV
  ,exponent-product = \cdot, output-product = \cdot, % skift gangetegn fra \times til \cdot
}
%%% VI LAVER NOGLE FYSIK- OG MATEMATIK-MAKROER:


%% Generelt
%\newcommand{\g}{\cdot} % Prikprodukt, gangetegn
\newcommand{\subv}[2]{\gv{#1}_{\text{#2}}} % Pæn subscript til vektorer
\newcommand{\sub}[2]{#1_{\text{#2}}} % Pæn subscript til
\newcommand{\e}{\mathcal{E}} % Skrevet E
\newcommand{\abs}[1]{\left| #1 \right|} % Numerisk værdi
\newcommand{\N}{\ensuremath{\mathbb{N}}} % Naturlige tal
\newcommand{\Z}{\ensuremath{\mathbb{Z}}} % Hele tal
\newcommand{\Q}{\ensuremath{\mathbb{Q}}} % Rationelle tal
\newcommand{\R}{\ensuremath{\mathbb{R}}} % Reelle tal
\newcommand{\C}{\ensuremath{\mathbb{C}}} % Komplekse tal
\newcommand{\F}{\ensuremath{\mathbb{F}}} % Legeme tal
\newcommand{\A}{\ensuremath{\mathbb{A}}} % Algebraiske tal

%% Angiv sværhedsgrad til opgaver (benytter \usepackage{xstring})
%\newcommand{\lvl}[1]{%
%\IfStrEqCase{#1}{{1}{\ensuremath{\star}}
%    {2}{\ensuremath{\star\star}}
%    {3}{\ensuremath{\star\star\star}}}
%    [nada]
%}

%% Infinitesimalregning

\let\underdot=\d % omdøb indbygget kommando \d{} til \underdot{}
%\renewcommand{\d}[2]{\partial_{#2} \, #1} % afledt
%\newcommand{\dd}[2]{\partial_{#2}^2 \, #1} % dbl.afledt

%differentierings d
\renewcommand{\d}{\mathrm{d}}

%haard differentiering
\newcommand{\dif}[3][]{\frac{\d^{#1}{#3}}{{\d {#2}}^{#1}}}

%partiel differentiering
\newcommand{\pdif}[3][]{\frac{\partial^{#1}{#3}}{\partial {#2}^{#1}}}

\newcommand{\dt}[1]{\dot{#1}} % afledt mht. t (dot-notation)
\newcommand{\ddt}[1]{\ddot{#1}} % dbl.afledt mht. t (dbl.dot)

\newcommand{\integral}[4]{\int_{#3}^{#4} \, #1 \, \textrm{d}#2} % integrere



% Vektorer

\newcommand{\xyz}[3]{\begin{bmatrix} #1 \\ #2 \\ #3 \end{bmatrix}} %3D-vektor
\newcommand{\xy}[2]{\begin{bmatrix} #1 \\ #2 \end{bmatrix}} %2D-vektor
\let\vaccent=\v % Omdøb \v{} til \vaccent{}

\newcommand{\gv}[1]{{\vec{\mathbf{#1}}}} % Vektor med græske bogstaver
\renewcommand{\v}[1]{\gv{#1}} % Vektor med fed
\newcommand{\hatvec}[1]{\hat{\mathbf{#1}}} % Hatvektor
\newcommand{\ihat}{\boldsymbol{\hat{\textbf{\i}}}} % Enhedsvektor i
\newcommand{\jhat}{\boldsymbol{\hat{\textbf{\j}}}} % .. j
\newcommand{\khat}{\mathbf{\hat{k}}}  % .. k
\newcommand{\xhat}{\mathbf{\hat{x}}} % Enhedsvektor x
\newcommand{\yhat}{\mathbf{\hat{y}}} % .. y
\newcommand{\zhat}{\mathbf{\hat{z}}} % .. z
\newcommand{\grad}[1]{\gv{\nabla} #1} % Gradient
\let\divsymb=\div % Omdøb \div til \divsymb
\renewcommand{\div}[1]{\gv{\nabla} \cdot \v{#1}} % Divergens
\newcommand{\curl}[1]{\gv{\nabla} \times \v{#1}} % Curl
% Vil man tage div eller curl af græske bogstaver,
% skal man lade argumentetet være fx \gv{\mu} for µ-vektor

% Kvantemekanik

\newcommand{\op}[1]{\hat #1} % operator

\newcommand{\expect}[1]{\left< #1 \right>} % Forventningsværdi
\newcommand{\trace}{\ensuremath{\text{Tr}}\xspace}
\newcommand{\Hilbert}{\ensuremath{\mathcal{H}}}
\newcommand{\lag}{\ensuremath{{L}}}
\newcommand{\tr}[1]{\text{Tr}\left(#1\right)} % Trace
\newcommand{\ptr}[2]{\text{Tr}_{#1}\left(#2\right)} % Partial trace
\newcommand{\ket}[1]{\left| #1 \right>} % Dirac-notation: ket
\newcommand{\bra}[1]{\left< #1 \right|} % bra
\newcommand{\braket}[2]{\left< #1 \vphantom{#2} \, \right|
  \left. \! #2 \vphantom{#1} \right>} % bracket
\newcommand{\matrixel}[3]{\left< #1 \vphantom{#2#3} \right|
  #2 \left| #3 \vphantom{#1#2} \right>} % Bracket med ekstra streg
 % En masse matematik- og fysikmakroer

%%%%%%%%%%%%
%% Layout %%
%%%%%%%%%%%%

%\newcommand{\anonbreak}{\fancybreak{$* * *$}} % Break med stjerner
%\let\bar\overline % Gør at en bar over et symbol kan skalere efter symbolet

%% Sidehoved- og fod

\makepagestyle{tket}
\makeevenfoot{tket}{\thepage}{}{}
\makeoddfoot{tket}{}{}{\thepage}
\makeevenfoot{plain}{\thepage}{}{}
\makeoddfoot{plain}{}{}{\thepage}
\makeevenhead{tket}{\leftmark}{}{}


%% Margin

% Man kan sætte margins ved enten at specificere marginstørrelsen
% eller ved at specificere tekstblokken. Man skal vælge én og kun én
% af mulighederne.

% Specificer marginstørrelsen
%\setulmarginsandblock{2.7cm}{*}{1}
%\setlrmarginsandblock{1.6cm}{1.6cm}{*} 
%\setlength{\oddsidemargin}{-1cm} % Giver mere plads på siden
%\setlength{\topmargin}{-1.2cm} % Gør topmargin behagelig at se på
%\setlength{\columnsep}{1.5\columnsep}  % Afstand mellem søjlerne


\setlrmarginsandblock{2.5cm}{2.5cm}{*}

\usepackage[font={small,it}]{caption}	% Italic captions

% Tekstblok: Følgende er fra Rasmus Villemoes' thesis-layout.tex
%\setlxvchars[\normalfont] % standardbredden af tekstblok er ca. 65 tegn
%\settypeblocksize{*}{1.2\lxvchars}{1.61803} % højde, bredde, forhold
%\setulmargins{*}{*}{1.3} % lav bundmargin lidt større end topmargin
\checkandfixthelayout % memoir tjekker, at alt er ok og konsistent

\usepackage{ctable} % Tillader fede linjer i tabeller

%%%%%%%%%%%%%%%%%
%% Bibliografi %%
%%%%%%%%%%%%%%%%%

\usepackage[style=ieee]{biblatex}
\addbibresource{litteratur.bib}

%%%%%%%%%%%%%%%%%%
%% Definitioner %%
%%%%%%%%%%%%%%%%%%

% Definer titlen på projektet
 \newcommand{\thesistitle}{Kompendie til UNF Fysik Camp 2019}

%%%%%%%%%%%%%%%%%%%%%%
%% Slut på preamble %%
%%%%%%%%%%%%%%%%%%%%%%
\usepackage{import}
\begin{document}
\chapter{Numerisk Fysik} \label{chap:numfys_opg}

\subsection{Del 1: Kan løses i hånden} \label{sec:numfys_opg:haanden}

\begin{opgave}[1]{Trapezmetoden}
Vi vil integrere en funktion $f(x)$ ved hjælp af trapezmetoden. Vi integrerer fra 3 til 10 i skridt af størrelse 1. Værdierne af $x$ og $f(x)$ fremgår af \cref{numfys:table:trapez}.

\setlength{\tabcolsep}{1.2 em}
\def\arraystretch{1.3}
\begin{table*}[]
    \centering
    \begin{tabular}{ccccccccc}
        \toprule
        \textbf{$x$}      & 3 & 4 & 5 & 6 & 7 & 8 & 9 & 10   \\
        % \midrule
        \textbf{$f(x)$} & 1 & 2 & 4 & 2 & 6 & 5 & -6 & 2   \\
        \bottomrule
      \end{tabular}
      \caption{ Nogle værdier fra en funktion $f(x)$.}
      \label{numfys:table:trapez}
\end{table*}{} 

\opg Brug formlen for trapezmetoden nedenfor til at beregne integralet af $f(x)$ fra 3 til 10. 

\begin{align}
    \int_a^b f(x) dx \approx \frac{\Delta x}{2} \left( f(x_0) + f(x_n) + 2 \sum_{i=1}^{N-1} f(x_i) \right) %eq:numfys:trapezmetoden i det andet dokument
\end{align}


\end{opgave}

\begin{opgave}[2]{Trapezmetoden 2}
Lad os bruge trapezmetoden til at approksimere integralet
\begin{align} \label{numfys:eq:trapez2}
    \int_0^5 \frac{1}{4} x^3 - x^2+3 \dd{x}.
\end{align}

\begin{figure}
    \centering
    \tikzset{axes/.style={->,>=stealth}}
    \tikzset{vector/.style={->,>=stealth}}
    
    
    \begin{tikzpicture}[scale=0.5, domain=0:5, smooth, yscale=1, xscale=2.5, declare function={fcnf(\x)=0.25*\x^3-\x^2+3;}]
    
    \path (-0.2,0) node(xline){} (5,0);
    \path (0,-0.2) node(yline){} (0,10);
    
    %\draw [step=1.0,light grey, very thin] (0,0) grid (5,10);
    
    \foreach \x in {0,1,2,3,4,5}
       \draw (\x cm,1pt) -- (\x cm,-1pt) node[anchor=north] {$\x$};
    \foreach \y in {0,1,2,3,4,5,6,7,8,9,10}
        \draw (1pt,\y cm) -- (-1pt,\y cm) node[anchor=east] {$\y$};
        
    \draw plot (\x,{fcnf(\x)}) node[right] {$f(x)$};
    
    \draw[axes] (-0.2,0) -- (5,0) node[right] {$x$};
    \draw[axes] (0,-0.2) -- (0,10) node[above] {$y$};
    \end{tikzpicture}
    \caption{Graf for funktionen, der integreres i \cref{numfys:eq:trapez2}.}
    \label{numfys:fig:trapez2}
\end{figure}



\opg Løs det først analytisk, som beskrevet i matematikafsnittet, så vi har et svar at sammenligne med.

\opg Tegn på figuren ovenfor hvordan hvert areal i trapezmetoden ville se ud for skridt af længde $\Delta x = 1$. Vil du forvente det numeriske integral bliver lidt for stort eller lidt for småt?

\opg Før vi bruger trapezmetoden må vi vide hvilke punkter vi bruger den på. Opskriv værdierne af $x$ og $f(x)$ i integralet for skridt af længde $\Delta x = 1$.

\opg Brug nu trapezmetoden til at estimere integralet. Blev resultatet som forventet?


\end{opgave}

\begin{opgave}[1]{Euler Metoden}
Lad os prøve at bruge Eulers metode til at bestemme værdien af en funktion der er løsningen til en given differential , givet at vi kender funktionsværdien i et andet punkt.
\\
Differentialligningen, vi vil forsøge at estimere, er:
\begin{align*}
    y' &= y \\
    y(0) &= 1 
\end{align*}

\opg Forsøg at finde $y(1)$ ved at bruge en skridtlængde på $h = 1$, dvs. kun en enkelt iteration i Euler metoden.
\opg Forsøg at finde $y(1)$ ved at bruge en skridtlængde på $h = 0.2$, dvs. at I skal regne fem iterationer i Euler metoden.
\opg Find den analytiske løsning af differentialligningen\footnote{Slå det op i en formelsamling eller spørg en regnelære hvis I ikke umiddelbart kender løsningen af differentialligningen} og sammenlign resultaterne fra de to første underopgaver med den ``korrekte'' værdi.

\end{opgave}



\subsection{Del 2: Løses i Python} \label{sec:numfys_opg:python}

Opgaverne til Python vil blive udleveret elektronisk. Vi anbefaler at kode online via \textcolor{blue}{cocalc.com/doc/jupyter-notebook}.
\\

\iffalse
    \begin{opgave}[1]{Python}
    
    Lad os først lige komme i gang med at programmere i Python.
    
    \opg Opret et dokument i Jupytor Notebook via hjemmesiden ovenfor. På hjemmesiden vil du blive spurgt om hvilken "kernel" du ønsker. Vælg Python 3.
    
    \opg Opret en variabel kaldet 
    \emph{årstal} %Ville bruge \pythoninline{} men den virker ikke her??
    og sæt værdien til hvilket år vi er i.
    
    \opg Bed python om at printe variablen.
    
    \opg Opret en variabel kaldet \emph{camp} og sæt værdien til hvilken camp du er på. Husk at sætte anførselstegn omkring navnet, så python ved den skal tolke det som en tekststreng. Ellers tror den det er en ukendt variabel.
    
    \opg Bed nu python om at printe campens navn.
    
    \opg Bed python om lægge 17 til årstallet (gøre variablen \emph{årstal} 17 større).
    
    \opg Print nu \emph{årstal} igen for at se om det virkede.
    
    \opg Bed python om at printe årstallet i anden.
    
    \opg Ekstra: Lav en funktion der tager et årstal som input og outputter årstallet plus 17 i anden.
    
    \end{opgave}
    
    
    
    
    
    
    
    \begin{opgave}[1]{Taylor}
    Plot følgende funktioner, sammen med deres taylorrække, udviklet omkring a = 0, for forskellige ordner (n = 0,1,2,3)
    
    \opg $f(x) = \frac{1}{4}x^3 - x^2 + 3 $
    \opg $f(x) = \sin(x)$
    \opg $f(x) = e^x$ 
    
    \opg Overvej hvordan man kan vurdere hvor stor fejl man begår med Taylor approksimationen, kan formlen for restleddet løses numerisk. Prøv at implementere en funktion der bestemmer hvor stor fejl man begår.
    
    
    \end{opgave}
    
    
    
    
    
     
    
    \begin{opgave}[2]{3 legeme problemet}
    Bestem planetære bevægelser ud fra Newtons gravitationslov, samt Newtons anden lov:
    
    \begin{align}
        \va{F} &= \frac{G \cdot m_1\cdot m_2}{|\va{r}|^2} \vu{r} \\
        \va{F} &= m\cdot \va{a}
    \end{align}
    
    Hvor $\va{r}$ er vektoren fra objekt 1 til objekt 2, og $\vu{r}$ er enhedsvektoren  $\vu{r} = \frac{\va{r}}{|\va{r}|}$
    
    \opg Sæt G = 1 og bestem bevægelsen af 3 objekter, med start hastighed, start position og masse: 
    \begin{align}
        \begin{matrix}
        &m_0 = 1  &v_0 = [0,0]  &x_0 = [0,0] \\
        &m_1 = 0.01 &v_1 = [0,1]  &x_1 = [1,0] \\
        &m_2 = 0.01 &v_2 = [0,-1]  &x_2 = [-1,0] 
    \end{matrix}
    \end{align}
    
    \end{opgave}

\fi


% Partikel i EB-felt (cycloiden) JEP 

% Bølge i væske

% Kosmologi opgaver (forskellige modeller af universet)

% Gletcher profiler (Geo) JEP 

% Ising model (magnetisme)

% Frit fald

% Frit fald med luftmodstand (laminar/turbulent)

% Skråt kast

% Skråt kast med luftmodstand

% Pendulet

% Spredning af elektriske partikler

% Faktisk regne på astro data?

% Project Euler (som en fodnote)





\end{document}
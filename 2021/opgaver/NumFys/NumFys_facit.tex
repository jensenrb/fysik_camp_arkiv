\documentclass[crop=false, class=memoir]{standalone}
\usepackage[utf8]{inputenc}%Nødvendig for danske bogstaver
\usepackage[danish]{babel}%Sørger for at ting LaTeX gør automatisk er på dansk
\usepackage{csquotes}
\usepackage{geometry}%Til opsætning af siden
\geometry{lmargin = 2.5cm,rmargin = 2.5cm}%sætter begge magner
\usepackage{lipsum}%Fyldtekst, til brug under test af layoutet
\usepackage{float}
\usepackage{graphicx}%Tillader grafik
\usepackage{epstopdf}%Tillader eps filer
\usepackage{marginnote}% Noter i margen
\interfootnotelinepenalty=10000 %undgår at fodnoter bliver spilittet op.
\usepackage[sorting=none]{biblatex}
\addbibresource{litteratur.bib}
\usepackage[hidelinks]{hyperref}%Tillader links
\usepackage{subcaption} % Tillader underfigurer
\usepackage[font={small,sl}]{caption}	% Caption med skrå tekst ikke kursiv

\usepackage{xcolor} %Bruges til farver
\usepackage{forloop} %Bruges til nemmere for loops

\newcounter{opgave}[chapter] %Definerer opgavenumrene og hvornår de nulstilles
\renewcommand{\theopgave}{\thechapter.\arabic{opgave}} %Definerer udseende af opgavenummereringen
\newcounter{delopgave}[opgave] %Definerer delopgavenumrene
\newcounter{lvl} %Definerer en "variabel" til senere brug

\definecolor{markerColor}{rgb}{0.0745098039, 0.262745098, 0.584313725} %Definerer farven af markøren
\newcommand{\markerSymbol}{\ensuremath{\bullet}} %Definerer tegnet for markøren
\newlength{\markerLength} %Definerer en ny længde
\settowidth{\markerLength}{\markerSymbol} %Sætter den nye længde til bredden af markøren

\newenvironment{opgave}[2][0]{%Definerer det nye enviroment, hvor sværhedsgraden er den første parameter med en default på 0
\newcommand{\opg}{\refstepcounter{delopgave}\par\vspace{0.1cm}\noindent\textbf{\thedelopgave)\space}}%Definerer kommando til delopgave
\refstepcounter{opgave}%Forøger opgavenummer med 1 og gør den mulig at referere til
\setcounter{lvl}{#1}%Sætter "variablen" lvl lig med angivelsen af sværhedsgraden
\noindent\hspace*{-0.75em}\hspace*{-\value{lvl}\markerLength}\forloop{lvl}{0}{\value{lvl}<#1}{{\color{markerColor}\markerSymbol}}\hspace*{0.75em}%Sætter et antal af markører svarende til sværhedsgraden
\textbf{Opgave \theopgave : #2}\newline\nopagebreak\ignorespaces}{\bigskip} %Angiver udseende af titlen på opgaverne samt mellemrummet mellem opgaver



\usepackage{mathtools}%Værktøjer til at skrive ligninger
\renewcommand{\phi}{\varphi}%Vi bruger varphi
\renewcommand{\epsilon}{\varepsilon}%Vi bruger varepsilon
\usepackage{physics}%En samling matematikmakroer til brug i fysiske ligninger
\usepackage{braket}%Simplere kommandoer til bra-ket-notation
\usepackage{siunitx}%Pakke der håndterer SI enheder godt
\DeclareSIUnit\clight{\text{\ensuremath{c}}} % Lysets fart i vakuum som c og ikke c_0
\usepackage{chemmacros}
\usechemmodule{isotopes}
\usepackage{tikz}
\usepackage[danish]{cleveref}
\usepackage{nicefrac}
% \renewcommand{\ref}[1]{\cref{#1}}
\creflabelformat{equation}{#2(#1)#3}
\crefrangelabelformat{equation}{#3(#1)#4 to #5(#2)#6}
\crefname{equation}{ligning}{ligningerne}
\Crefname{equation}{Ligning}{Ligningerne}
\crefname{section}{afsnit}{afsnitene}
\Crefname{section}{Afsnit}{Afsnitene}
\crefname{figure}{figur}{figurene}
\Crefname{figure}{Figur}{Figurene}
\crefname{table}{tabel}{tabellerne}
\Crefname{table}{Tabel}{Tabellerne}
\crefname{opgave}{opgave}{opgaverne}
\Crefname{opgave}{Opgave}{Opgaverne}
\crefname{delopgave}{delopgave}{delopgaverne}
\Crefname{delopgave}{Delopgave}{Delopgaverne}

\newcommand{\eqbox}[1]{\begin{empheq}[box=\fbox]{align}
	\begin{split}
	#1
	\end{split}
\end{empheq}}

\newcommand{\kb}{\ensuremath{k_\textsc{b}}}

\DeclareSIUnit{\parsec}{pc}
\DeclareSIUnit{\lightyear}{ly}
\DeclareSIUnit{\astronomicalunit}{AU}
\DeclareSIUnit{\year}{yr}
\DeclareSIUnit{\solarmass}{M_\odot}
\DeclareSIUnit{\solarradius}{R_\odot}
\DeclareSIUnit{\solarluminosity}{L_\odot}
\DeclareSIUnit{\solartemperature}{T_\odot}
\DeclareSIUnit{\earthmass}{M_\oplus}
\DeclareSIUnit{\earthradius}{R_\oplus}
\DeclareSIUnit{\jupitermass}{M_J}

% Infobokse og lignende
% http://mirrors.dotsrc.org/ctan/graphics/awesomebox/awesomebox.pdf
% \usepackage{awesomebox}


% Egen infobokse (virker kun med begrænsede symboler)

\usepackage[framemethod=tikz]{mdframed}
\usetikzlibrary{calc}
\usepackage{kantlipsum}

\usepackage[tikz]{bclogo}

\tikzset{
    % lampsymbol/.style={scale=2,overlay}
    % lampsymbol/.pic={\centering\tikz[scale=5]\node[scale=10,rotate=30]{\bclampe}}.style={scale=2,overlay}
    infosymbol/.style={scale=2,overlay}
}

\newmdenv[
    hidealllines=true,
    nobreak,
    middlelinewidth=.8pt,
    backgroundcolor=blue!10,
    frametitlefont=\bfseries,
    leftmargin=.3cm, rightmargin=.3cm, innerleftmargin=2cm,
    roundcorner=5pt,
    % skipabove=\topsep,skipbelow=\topsep,
    singleextra={\path let \p1=(P), \p2=(O) in ($(\x2,0)+0.92*(1.1,\y1)$) node[infosymbol] {\bcinfo};},
    % singleextra={\path let \p1=(P), \p2=(O) in ($(\x2,0)+0.5*(2,\y1)$) node[infosymbol] {\bcinfo};},
]{info}

% Skal bruges som
% \begin{info}[frametitle={Titel}]
%     Tekst
% \end{info}
\usepackage{import}
\begin{document}
\chapter{Numerisk Fysik} \label{chap:numfys_facit}

\begin{opgave}[1]{Trapezmetoden}

\opg Brug formlen nedenfor til at beregne integralet af $f(x)$ fra 3 til 10.

\begin{align}
    \int_3^{10} f(x) dx &\approx \frac{\Delta x}{2} \left( f(x_0) + f(x_n) + 2 \sum_{i=1}^{N-1} f(x_i) \right) \\
    &= \frac{\Delta x}{2} \bigg( f(x_0) + f(x_N) + 2\big(f(x_1)+f(x_2)+...+f(x_{N-1})\big) \bigg) \\
    &= \frac{1}{2} \bigg( f(3) + f(10) + 2\big(f(4)+f(5)+f(6)+f(7)+f(8)+f(9))\big) \bigg) \\
    &= \frac{1}{2} \bigg( 1 + 2 + 2\big(2+4+2+6+5-6)\big) \bigg)\\
    &= \frac{1}{2} \cdot 29\\
    &= \num{14.5}
\end{align}


\end{opgave}

\begin{opgave}[1]{Trapezmetoden 2}


\opg Løs det først analytisk, så vi har et svar at sammenligne med.

\begin{align}
    \int_0^5 \frac{1}{4} x^3 - x^2+3 \dd{x} &= \left[\frac{1}{4\cdot4} x^4 - \frac{1}{3}x^3 + 3x \right]_0^5 \\
    &= \left(\frac{1}{16} 5^4 - \frac{1}{3}5^3 + 3\cdot5 \right) - \left(\frac{1}{16} 0^4 - \frac{1}{3}0^3 + 3\cdot0 \right)\\
    &\approx \num{12,4}
\end{align}

\opg Tegn på figuren ovenfor hvordan hvert areal i trapezmetoden ville se ud for skridt af længde $\Delta x = 1$.


\tikzset{axes/.style={->,>=stealth}}
\tikzset{vector/.style={->,>=stealth}}


\begin{tikzpicture}[scale=0.5, domain=0:5, yscale=1, xscale=2.5, smooth, declare function={fcnf(\x)=0.25*\x^3-\x^2+3;}]

\path (-0.2,0) node(xline){} (5,0);
\path (0,-0.2) node(yline){} (0,10);

%\draw [step=1.0,light grey, very thin] (0,0) grid (5,10);

\def\h{1}

\foreach \n in {1,2,3,4,5}
    \filldraw[blue!30, draw=black,style=dashed]  ({\h*\n},{fcnf(\h*\n)}) -- ({\n*\h}, 0) -- ({\h*\n-\h}, 0) -- (\h*\n-\h,{fcnf(\h*\n-\h)});
    % \draw({\h*\n},{fcnf(\h*\n)}) -- ({\n*\h}, 0) -- ({\h*\n-\h}, 0) -- (\h*\n-\h,{fcnf(\h*\n-\h)});

\foreach \x in {0,1,2,3,4,5}
   \draw (\x cm,1pt) -- (\x cm,-1pt) node[anchor=north] {$\x$};
\foreach \y in {0,1,2,3,4,5,6,7,8,9,10}
    \draw (1pt,\y cm) -- (-1pt,\y cm) node[anchor=east] {$\y$};
    
\draw plot (\x,{fcnf(\x)}) node[right] {$f(x)$};

\draw[axes] (-0.2,0) -- (5,0) node[right] {$x$};
\draw[axes] (0,-0.2) -- (0,10) node[above] {$y$};




\end{tikzpicture}


\opg Før vi bruger trapezmetoden må vi vide hvilke punkter vi bruger den på. Opskriv værdierne af $x$ og $f(x)$ i integralet for skridt af længde $\Delta x = 1$.

\begin{table*}[h!]
      \centering
      \begin{tabular}{|l|c|c|c|c|c|c|c|c|}

          \hline

          %\hdashline
          \textbf{$x$}      & 0 & 1 & 2 & 3 & 4 & 5  \\
          \hline
          \textbf{$f(x)$} & 3 & \num{2,25} & 1 & \num{0,75} & 3 & \num{9,25}  \\

          \hline
      \end{tabular}
\end{table*}{} 

\opg Brug nu trapezmetoden til at estimere integralet.


\begin{align}
    \int_0^5 \frac{1}{4} x^3 - x^2+3 \dd{x} &\approx \frac{\Delta x}{2} \left( f(x_0) + f(x_n) + 2 \sum_{i=1}^{N-1} f(x_i) \right) \\
    &= \frac{1}{2} \bigg( f(0) + f(5) + 2\big(f(1)+f(2)+f(3)+f(4))\big) \bigg) \\
    &= \frac{1}{2} \bigg( 3 + 9.25 + 2\big(\num{2.25} + 1 + \num{0.75} + 3)\big) \bigg) \\
    &= \num{13,125}
\end{align}

\end{opgave}





\begin{opgave}[1]{Euler Metoden}

\opg Euler metoden med en iteration
\begin{align*}
    y(0) &= 1\\
    y(0 + 1) &= y(0) + 1 \cdot y'(0) = 1 + 1 \cdot 1 = 2
\end{align*}
\opg Euler metoden med 5 iteratioenr
\begin{align*}
    y(0) &= 1 \\
    y(0+0.2) &= y(0) + 0.2 \cdot y'(0) = 1 + 0.2 \cdot 1 = 1.2 \\
    y(0.2 + 0.2) &= y(0.2) + 0.2 \cdot y'(0.2) = 1.2 + 0.2 \cdot 1.2 = 1.44 \\
    y(0.4 + 0.2) &= y(0.4) + 0.2 \cdot y'(0.4) = 1.44 + 0.2 \cdot 1.44 = 1.728 \\
    y(0.6 + 0.2) &= y(0.6) + 0.2 \cdot y'(0.6) = 1.728 + 0.2 \cdot 1.728 = 2.0736 \\
    y(0.8 + 0.2) &= y(0.8) + 0.2 \cdot y'(0.8) = 2.0736 + 0.2 \cdot 2.0736 = 2.48832 \\
\end{align*}

\opg 
Løsningen til differentialligningen er $y(x) = e^x$. Dvs at $y(1) = e^1 \approx 2.71828182846$.
Vi ser da at jo mindre $h$ er jo tættere kommer vi på den korrekte løsning, men at vi dermed også skal bruge mere regnekraft. Hvis man forsøger med h = 0.001, og dermed skal igennem 1000 iterationer kommer man frem til resultatet: 2.7169, hvilket er tættere på det rigtige resultat, men stadig ikke så tæt som man kunne forvente efter 1000 iterationer. Heldigvis er det nemt at gøre skridtlængden lille med en computer, men det ville aldrig noget man kunne gøre i hånden. 




\end{opgave}









 

\end{document}
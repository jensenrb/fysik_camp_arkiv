\setlength{\tabcolsep}{1.5 em}
\def\arraystretch{1.35}

\newpage

\section{Nyttige fysiske konstanter og enheder}\label{mat:sec:fysiskekonstanter}

Her er en tabel med nogle af de vigtige konstanter og enheder, som bruges igennem de følgende kapitler. \\[2mm]

\vspace*{-\baselineskip}
\begin{table}[h!]
\centering
\begin{tabular}{llll}
%
%\multicolumn{4}{c}{\Large{\textbf{Fysiske Størrelser, Konstanter og Specielle Enheder}}} \\[2mm]
% \hline
\toprule
%
\textbf{Konstant} & \textbf{Symbol} & \textbf{Værdi} & \textbf{Enhed} \\
%
% \hline
\midrule
%
Lysets fart i vakuum & \si{\clight} & \num{2,99792458e8} & \si{\metre\per\second} \\
Gravitationskonstanten & $G$ & \num{6,6742e-11} & \si{\newton\metre\squared\per\kilo\gram\squared} \\
Plancks konstant & $h$ & \num{6,6260693e-34} & \si{\joule\second} \\
Reduceret Plancks konstant & $\hbar = h/2\pi$ & \num{1,0545718e-34} & \si{\joule\second} \\
Boltzmanns konstant & $\kb$ & \num{1,3806506e-23} & \si{\joule\per\kelvin} \\
Stefan-Boltzmanns konstant & $\sigma$ & \num{5,670373e-8} & \si{\watt\per\metre\squared\per\kelvin\tothe4} \\
Avogadros tal & $N_A$ & \num{6.022141e23} & \si{\per\mole} \\
Gaskonstanten & $R=N_A\kb$ & \num{8.3144598} & \si{\joule\per\mole\per\kelvin} \\
Elementarladningen & $e$ & \num{1,602176565e-19} & \si{\coulomb} \\
Vakuumpermittiviteten & $\epsilon_0$ & \num{8,854187817e-12} & \si{\coulomb\squared\per\newton\per\metre\squared} \\
Vakuumpermeabiliteten & $\mu_0$ & \num{4\pi e-7} & \si{\newton\per\ampere\squared} \\[2mm]
%
% \hline
\midrule
%
\textbf{Størrelse} & \textbf{Symbol} & \textbf{Værdi} & \textbf{Enhed} \\ \hline
%
Bohrradius & $a_0$ & \num{5,2917721067e-11} & \si{\metre} \\
%Jordens radius & $R_\oplus$ & \num{6,371e6} & \si{\metre} \\
%Jordens masse & $M_\oplus$ & \num{5,97219e24} & \si{\kilo\gram} \\
%Solens radius & $R_\odot$ & \num{6,95700e8} & \si{\metre} \\
%Solens masse & $M_\odot$ & \num{1,9891e30} & \si{\kilo\gram} \\
%Solens overfladetemperatur & $T_\odot$ & \num{5,778e3} & \si{\kelvin} \\
%Jupiters masse & $M_J = M_\mathrm{Jup}$ & \num{1.898e27} & \si{\kilo\gram} \\
Jordens radius & \si{\earthradius} & \num{6,371e6} & \si{\metre} \\
Jordens masse & \si{\earthmass} & \num{5,97219e24} & \si{\kilo\gram} \\
Solens radius & \si{\solarradius} & \num{6,95700e8} & \si{\metre} \\
Solens masse & \si{\solarmass} & \num{1,9891e30} & \si{\kilo\gram} \\
Solens overfladetemperatur & \si{\solartemperature} & \num{5,778e3} & \si{\kelvin} \\
Jupiters masse & $\si{\jupitermass} = \textup{M}_\textup{Jup}$ & \num{1.898e27} & \si{\kilo\gram} \\
Protonens masse & $m_p$ & \num{1,672621898e-27} & \si{\kilo\gram} \\
Neutronens masse & $m_n$ & \num{1,674927471e-27} & \si{\kilo\gram} \\
Elektronens masse & $m_e$ & \num{9,10938356e-31} & \si{\kilo\gram} \\[2mm]
%
% \hline
\midrule
%
\textbf{Enhed} & \textbf{Symbol} & \textbf{Værdi} & \textbf{SI-Enhed} \\
%
% \hline
\midrule
%
Astronomisk enhed & \si{\astronomicalunit} & \num{1,49597870700e11} & \si{\metre} \\
Lysår & \si{\lightyear} & \num{9.4605284e15} & \si{\metre} \\
Parsec & \si{\parsec} & \num{3.08567758e16} & \si{\metre} \\
Atomar masseenhed & \si{u} & \num{1.66053904e-27} & \si{\kilo\gram} \\
Ångström & \si{\angstrom} & \num{1,00\dots{}e-10} & \si{\metre} \\
Elektronvolt & \si{\electronvolt} & \num{1.60217662e-19} & \si{\joule} \\
År & \si{\year} & \num{31556925.445} & \si{\second} \\[.5mm]
%
% \hline
\bottomrule
\end{tabular}
\end{table}
